\documentclass[oneside,10pt]{article}
% reminder: US letter: 596pt x 795pt

\newlength{\pagewidthA}
\newlength{\pageheightA}
\setlength{\pagewidthA}{8.5in}
\setlength{\pageheightA}{11in}

\newlength{\pagewidthB}
\newlength{\pageheightB}
\setlength{\pagewidthB}{11in}
\setlength{\pageheightB}{8.5in}

\newlength{\stockwidth}
\newlength{\stockheight}

\usepackage{geometry}

% these seem to have NO effect in preamble!
% geometry first has effect after begin{document}!
\pdfpagewidth=\pagewidthA \pdfpageheight=\pageheightA % to enforce?
\paperwidth=\pagewidthA \paperheight=\pageheightA     % for TikZ
\stockwidth=\pagewidthA \stockheight=\pageheightA % hyperref (memoir)?!


\geometry{inner=1cm,outer=1cm,top=1cm,bottom=1cm}


\makeatletter
\newcommand{\printpagevalues}{%
  % from geometry.sty:
  * paper: \ifx\Gm@paper\@undefined<default>\else\Gm@paper\fi \\%
  * layout: \ifGm@layout<custom>\else<same size as paper>\fi \\%
  \@ifundefined{ifGm@layout}{}{%
  \ifGm@layout
  * layout(width,height): (\the\Gm@layoutwidth,\the\Gm@layoutheight) \\%
  \fi
  * layoutoffset:(h,v)=(\the\Gm@layouthoffset,\the\Gm@layoutvoffset) \\%
  }%
  \pagevalues % from package layouts
}
\makeatother

\newcommand{\generatePageLayouts}{%
  % this command must be called after \begin{document}!

  % geometry needs layoutwidth - cause it ignores the above paper sizes!
  % layoutwidth=148mm ok, layoutwidth=\paperwidth NOT ok
  % paperwidth gets reset again internally in newgeometry: in log: *geometry* verbose mode: * layout(width,height): (614.295pt,794.96999pt)
  % but by using \stockwidth, which here is just a custom length: * layout(width,height): (421.10078pt,597.50787pt)

  \newgeometry{layoutwidth=\pagewidthA,layoutheight=\pageheightA,left=2cm,right=2cm,bottom=2cm,top=2cm}
  \savegeometry{LayoutPageA}

  \newgeometry{layoutwidth=\pagewidthB,layoutheight=\pageheightB,inner=1cm,outer=1cm,top=1cm,bottom=1cm}
  \savegeometry{LayoutPageB}
}


\newcommand{\switchToLayoutPageA}{%
  % doesn't include page sizes; so page size too:
  \pdfpagewidth=\pagewidthA \pdfpageheight=\pageheightA % for PDF output
  \paperwidth=\pagewidthA \paperheight=\pageheightA     % for TikZ
  \stockwidth=\pagewidthA \stockheight=\pageheightA % hyperref (memoir)?!
  \loadgeometry{LayoutPageA} % note; \loadgeometry may reset paperwidth/h!
}
\newcommand{\switchToLayoutPageB}{%
  % doesn't include page sizes; so page size too:
  \pdfpagewidth=\pagewidthB \pdfpageheight=\pageheightB % for PDF output
  \paperwidth=\pagewidthB \paperheight=\pageheightB     % for TikZ
  \stockwidth=\pagewidthB \stockheight=\pageheightB % hyperref (memoir)?!
  \loadgeometry{LayoutPageB} % note; \loadgeometry may reset paperwidth/h!
}




\usepackage{sectsty}
\usepackage[explicit]{titlesec}
\usepackage{fancyhdr}
\usepackage[absolute]{textpos}
\usepackage{xltxtra,fontspec,xunicode}
\defaultfontfeatures{Scale=MatchLowercase}
\setmainfont[BoldFont=Shadowrun Bold,ItalicFont=Shadowrun Italic]{Shadowrun}
\newfontfamily\fatefont{Fate Core Glyphs}
\newfontfamily\orbitronfont{Orbitron}
\newfontfamily\oswaldfont{Oswald}
\newfontfamily\shadowrunfont{Shadowrun}
\newfontfamily\shadowrunbfont{Shadowrun Bold}
\setlength{\parindent}{0cm}
\setlength{\parskip}{.2cm}
\setcounter{tocdepth}{2}
\setcounter{secnumdepth}{0}


\usepackage{pdflscape}
\usepackage{atbegshi}
\usepackage[svgnames,table]{xcolor} 
\usepackage{lipsum}
\usepackage{tabu}
\usepackage{multicol}
\usepackage{changepage}
\usepackage{parcolumns}
\usepackage{array}
\usepackage{comment}
\usepackage{picture}
\usepackage[hidelinks]{hyperref}
\usepackage{float}
\usepackage{layouts}
%\usepackage{showframe}
\usepackage{ifthen}
\usepackage{booktabs}
\usepackage{multirow}
\definecolor{lightgray}{gray}{0.90}


\newcommand\Mybox[1]{%
  \setlength\fboxsep{0pt}\fcolorbox{red}{white}{#1}%
}


\newsavebox{\statBox}
\savebox{\statBox}
(25,25)[bl]{
  \multiput(0,0)(0,25){2}
  {\line(1,0){25}}
  \multiput(0,0)(25,0){2}
  {\line(0,1){25}}
  %\put(20,20){\line(1,0){5}}
  %\put(20,20){\line(0,1){5}}
  \put(0,15){\line(1,1){10}}
}

\newsavebox{\statBoxInf}
\savebox{\statBoxInf}
  (25,25)[bl]{
     \multiput(0,0)(0,25){2}
    {\line(1,0){25}}
  \multiput(0,0)(25,0){2}
    {\line(0,1){25}}
  \put(20,20){\line(1,0){5}}
  \put(20,20){\line(0,1){5}}
}

\newsavebox{\statBoxPlain}
\savebox{\statBoxPlain}
  (25,25)[bl]{
     \multiput(0,0)(0,25){2}
    {\line(1,0){25}}
  \multiput(0,0)(25,0){2}
    {\line(0,1){25}}
}

\newsavebox{\optaccumBox}
\savebox{\optaccumBox}
(10,10)[bl]{
  
  \multiput(0,0)(0,10){2}
  {\line(1,0){10}}
  \multiput(0,0)(10,0){2}
  {\line(0,1){10}}
}

\newsavebox{\accumBox}
\savebox{\accumBox}
  (10,10)[bl]{
\multiput(0,0)(0,10){2}
    {\line(1,0){10}}
  \multiput(0,0)(10,0){2}
    {\line(0,1){10}}
}

\newsavebox{\accumBoxT}
\savebox{\accumBoxT}
  (12,12)[bl]{
\linethickness{2pt}
\multiput(0,0)(0,10){2}
    {\line(1,0){12}}
  \multiput(1,0)(10,0){2}
    {\line(0,1){11}}
}


\newcommand{\selectedMove}[1]{{\fatefont{}B} \textbf{#1}}
\newcommand{\unselectedMove}[1]{{\fatefont{}b} \textbf{#1}}


%\floatstyle{boxed}
\newfloat{dossier}{t}{dos}
\floatname{dossier}{Dossier}

\def\vrulefill{\leavevmode\leaders\vrule width 2pt\vfill\kern\z@}

\titleformat{\section}[block]
{\orbitronfont\fontsize{1.2cm}{1em}\selectfont\bfseries\filright#1}{}{0pt}{}[\vspace{0ex}\titlerule]

\titleformat{\subsection}[block]
{\orbitronfont\fontsize{.7cm}{1em}\selectfont\bfseries\filright#1}{}{0pt}{}[\vspace{0ex}\titlerule]

\titleformat{name=\subsubsection}[block]{\orbitronfont\fontsize{.5cm}{1em}\selectfont\bfseries\filright}{}{0em}{\underline{\MakeUppercase{#1}}}

\titleformat{name=\paragraph}[block]{\orbitronfont\fontsize{.45cm}{1em}\selectfont\bfseries\filright}{}{0em}{\MakeUppercase{#1}}

\titlespacing{\paragraph}{%
  0pt}{%              left margin
  1\baselineskip}{% space before (vertical)
  .5em}%  

\newcommand{\LeftBlockEntry}[2]{                     % Same as \EducationEntry
  \noindent
  \colorbox{Black}{
    \parbox[c][.64cm]{#1}{\color{White}{\orbitronfont\fontsize{18pt}{1em}\bfseries\selectfont
        {\raggedright{} #2}}}
}}
\newcommand{\CenterBlockEntry}[2]{                     % Same as \EducationEntry
  \noindent
  \colorbox{Black}{
    \parbox[c][.64cm]{#1}{\color{White}{\orbitronfont\fontsize{18pt}{1em}\bfseries\selectfont
        {\centering{#2}}}}
}}

\newsavebox{\dossierboxhelper}
\newenvironment{dossierbox}[3]
{
\begin{lrbox}{\dossierboxhelper}
\begin{minipage}[t][#2]{#1}
\CenterBlockEntry{#1}{#3}
\vspace{.1cm}
}
{
\end{minipage}
\end{lrbox}
\usebox{\dossierboxhelper}
}

\newenvironment{moveoptions}
{
\vspace{-.3cm}
\begin{adjustwidth*}{.5cm}{.5cm}
\setlength{\parskip}{.1cm}
\begin{multicols}{2}
}
{
\end{multicols}
\end{adjustwidth*}
\vspace{-.3cm}
}

\newcommand{\moveoption}[1]{\parbox{\linewidth}{\tcirc{} #1}}

\newsavebox{\dossierstatbarbox}
\newcommand{\dossierstatbar}{
\begin{lrbox}{\dossierstatbarbox}
\begin{minipage}[b][\textheight][t]{.215\textwidth}
\begin{dossierbox}{5.2cm}{5cm}{Profile}
\begin{tabular}{|>{\oswaldfont\fontsize{15pt}{1em}\selectfont\bfseries}m{1.8cm}|m{2.9cm}|}
\hline
Street Name& \\[.25cm]\hline
Real Name&  \\[.25cm]\hline
Metatype&  \\[.25cm]\hline
Look&  \\[1cm]
\hline
\end{tabular}
\end{dossierbox}
\begin{dossierbox}{5.2cm}{6.5cm}{STATS}
\begin{picture}(5cm,5cm)
\put( 0cm,4cm){\usebox{\statBox}}
\put( 1cm,4.1cm){ \oswaldfont\fontsize{16pt}{1em}\selectfont\bfseries Oomph }
\put( 0cm, 3cm){\usebox{\statBox}}
\put( 1cm,3.1cm){ \oswaldfont\fontsize{16pt}{1em}\selectfont\bfseries Twitch }
\put( 0cm,2cm){\usebox{\statBox}}
\put( 1cm,2.1cm){ \oswaldfont\fontsize{16pt}{1em}\selectfont\bfseries Mastery }
\put( 0cm,1cm){\usebox{\statBox}}
\put( 1cm,1.1cm){ \oswaldfont\fontsize{16pt}{1em}\selectfont\bfseries Flair }
\put( 0cm,0cm){\usebox{\statBox}}
\put( 1cm,0.1cm){ \oswaldfont\fontsize{16pt}{1em}\selectfont\bfseries Essence }
\put( 3cm,4cm){\usebox{\statBoxPlain}}
\put( 4cm,4.1cm){ \oswaldfont\fontsize{16pt}{1em}\selectfont\bfseries Edge }
\put( 3cm,2cm){\usebox{\statBoxPlain}}
\put( 4cm,2.1cm){ \oswaldfont\fontsize{16pt}{1em}\selectfont\bfseries Armor }
\end{picture}
\end{dossierbox}
\begin{dossierbox}{5.2cm}{3cm}{DAMAGE}
\begin{picture}(5cm,1.1cm)
\multiput(1.25cm,0.6cm)(15,0){7}{\usebox{\accumBox}}
\put(4.95cm,0.6cm){\usebox{\accumBoxT}}
\put(3.25cm,.14cm){\line(0,1){6}}
\put(3.25cm,0.35cm){\line(1,0){52}}
\put(5.08cm,0.35cm){\line(0,1){6}}
\put(3.35cm, 0cm){\oswaldfont\fontsize{12pt}{1em}\selectfont\bfseries GUT CHECK }
\end{picture}
\end{dossierbox}
\begin{dossierbox}{5.2cm}{3cm}{XP}
\begin{picture}(5cm,1.1cm)
\multiput(0.2cm,0.6cm)(15,0){9}{\usebox{\accumBox}}
\put(4.95cm,0.6cm){\usebox{\accumBoxT}}
\put(3.25cm,.14cm){\line(0,1){6}}
\put(3.25cm,0.35cm){\line(1,0){52}}
\put(5.08cm,0.35cm){\line(0,1){6}}
\put(3.35cm, 0cm){\oswaldfont\fontsize{12pt}{1em}\selectfont\bfseries ADVANCE }
\end{picture}
\end{dossierbox}
\end{minipage}%
\end{lrbox}
\usebox{\dossierstatbarbox}
}

\newsavebox{\dossiermovebarbox}
\newenvironment{dossiermovebar}
{
\begin{lrbox}{\dossiermovebarbox}
\begin{minipage}[b][\textheight][t]{0.50\linewidth}
\begin{dossierbox}{17.6cm}{14cm}{ARCHETYPE MOVES}
}
{
\vspace{.2cm}
\end{dossierbox}%
\begin{dossierbox}{17.6cm}{3cm}{WEAPONS}
\begin{adjustwidth*}{0cm}{.2cm}
\vspace{-.1cm}
\begin{tabu}{p{5cm}p{1.5cm}p{1.5cm}p{5cm}p{3cm}}
\rowfont{\oswaldfont\fontsize{16pt}{0em}\selectfont} Weapon & Range & Damage & Ammo & Tags\\
\end{tabu}

\vspace{.2cm}
\setlength{\baselineskip}{.5cm}
\underline{\hspace{17.9cm}}

\underline{\hspace{17.9cm}}

\underline{\hspace{17.9cm}}

\underline{\hspace{17.9cm}}

\end{adjustwidth*}
\end{dossierbox}
\end{minipage}%
\end{lrbox}
\usebox{\dossiermovebarbox}
}


\newsavebox{\dossierrightbarbox}
\newcommand{\dossierrightbar}[1]{
\begin{lrbox}{\dossierrightbarbox}
\ifthenelse{ \equal{#1}{standard} }{
\begin{minipage}[b][\textheight][t]{0.215\linewidth}
\begin{dossierbox}{5.2cm}{3cm}{DEBTS \& FAVORS}
\begin{adjustwidth*}{0cm}{.2cm}
\fontsize{9pt}{1em}\selectfont

\underline{\hspace{3cm}} taught me a valuable lesson in
self-control.

\underline{\hspace{3cm}} helped me find my place here in the
shadows.

I killed someone for \underline{\hspace{3cm}}.

I taught \underline{\hspace{3cm}} secrets normally reserved
for the masters.
\end{adjustwidth*}
\vspace{.5cm}
\end{dossierbox}% 
\begin{dossierbox}{5.2cm}{3cm}{ARMOR}
\begin{adjustwidth*}{0cm}{.2cm}
\vspace{-.2cm}
{\oswaldfont\fontsize{14pt}{0em}\selectfont Type \hspace{2.6cm} Armor Value }
\vspace{.2cm}
\underline{\hspace{5cm}}
\end{adjustwidth*}
\end{dossierbox}%
\begin{dossierbox}{5.2cm}{3cm}{GEAR}
\begin{adjustwidth*}{0cm}{.2cm}
\vspace{.2cm}
\setlength{\baselineskip}{.5cm}
\underline{\hspace{5cm}}
\underline{\hspace{5cm}}
\underline{\hspace{5cm}}
\underline{\hspace{5cm}}
\underline{\hspace{5cm}}
\underline{\hspace{5cm}}
\underline{\hspace{5cm}}
\underline{\hspace{5cm}}
\underline{\hspace{5cm}}
\underline{\hspace{5cm}}
\underline{\hspace{5cm}}
\underline{\hspace{5cm}}
\underline{\hspace{5cm}}
\underline{\hspace{5cm}}
\underline{\hspace{5cm}}
\underline{\hspace{5cm}}
\underline{\hspace{5cm}}
\underline{\hspace{5cm}}
\underline{\hspace{5cm}}
\underline{\hspace{5cm}}
\end{adjustwidth*}
\end{dossierbox}%
\end{minipage}
}{
  This is false.
}
\end{lrbox}
\usebox{\dossierrightbarbox}}



\newcommand{\dossiervehiclebox}{
\begin{dossierbox}{6.9cm}{7cm}{VEHICLE / DRONE }
\begin{tabular}{p{2cm}p{2cm}p{2cm}}
\multicolumn{3}{l}{Name:} \\[.15cm] \hline
\multicolumn{3}{l}{Type:} \\[.15cm] \hline
\multicolumn{3}{l}{Tags:} \\[.15cm] \hline
Power:&Armor:&Frame:\\[.15cm] \hline
Sensor:&Fuel:&Capacity:\\[.15cm] \hline
\multicolumn{3}{l}{Tactical(Drone):}\\[.15cm]\hline
\multicolumn{3}{l}{Notes:} \\[.15cm]
\hline \\[.15cm] 
\hline \\[.15cm]
\hline \\[.15cm]
\hline \\[.15cm]
\hline \\[.15cm]
\hline \\[.15cm]
\hline \\[.15cm]
\end{tabular}
\end{dossierbox}
}









%%% Local Variables: 
%%% mode: latex
%%% TeX-master: "sixth_world"
%%% End: 


\newcommand{\SW}{\textit{Sixth World}}
\newcommand{\DW}{\textit{Dungeon World}}
\newcommand{\SR}{\textit{Shadowrun}}
\newcommand{\tcirc}{\protect{$\bigcirc$}}

\newcommand*\parttitle{}
\newcommand\invisiblepart[1]{%
  \refstepcounter{part}%
  \renewcommand*\parttitle{#1}%
}

\title{Your Title}
\author{Your Name}
\date{}

\setlength{\TPHorizModule}{1cm}
\setlength{\TPVertModule}{1cm}



%%% Local Variables: 
%%% mode: latex
%%% TeX-master: "sixth_world"
%%% End: 


\begin{document}

\generatePageLayouts{}
\switchToLayoutPageA{}


\fancypagestyle{plain}
{
  \fancyhf{}
    \fancyhead{}
    \fancyfoot{}
}	% clear header and footer of plain page because of ToC

\begin{titlepage}

\begin{center}

% Upper part of the page. The '~' is needed because \\
% only works if a paragraph has started.


~\\[5cm]

{\fontsize{75pt}{1em}\selectfont\bfseries
  SIXTH WORLD}\\[.5cm]

{\orbitronfont\fontsize{22.5pt}{1em}\selectfont\bfseries
  A DUNGEON WORLD HACK}\\[0.5cm]

{\orbitronfont\fontsize{22.5pt}{1em}\selectfont\bfseries
  FOR SHADOWRUN}\\[0.5cm]



\vfill

% Bottom of the page
Version: 1.0, Codename: ARES

Authors: Nathan Mitchell

Past Version Authors: Chris Clouser, Tanner Yea

\end{center}

DISCLAIMER

Dungeon World is the property of Sage LaTorra and Adam Koebel, and is available under the Creative Commons Attribution 3.0 Unported
License. See www.dungeon-world.com for details.
The Topps Company, Inc. has sole ownership of the names, logo, artwork, marks, photographs, sounds, audio, video and/or any proprietary
material used in connection with the game Shadowrun. This is a fan-created adaptation, and no challenge is intended toward Topp’s ownership of the Shadowrun intellectual property.





\end{titlepage}
%%% Local Variables: 
%%% mode: latex
%%% TeX-master: "sixth_world"
%%% End: 



\begin{multicols}{2}
\tableofcontents
\end{multicols}
\addtocontents{toc}{\protect\thispagestyle{plain}}

\newcommand{\critterspec}[7]{
\setlength{\parskip}{.1em}
\vspace{.5cm}
\begin{minipage}{\linewidth}
{\large\bfseries #1}

{\itshape #2}

#3

#4\vspace{.25em}
\hrule
\vspace{.25em}
#5 \textit{Instinct:} #6
\begin{adjustwidth*}{.5cm}{.5cm}
#7
\end{adjustwidth*}
\end{minipage}
}


\newpage
\pagestyle{fancy}
\setcounter{page}{1}
\renewcommand{\headrulewidth}{0pt} % remove lines as well
\renewcommand{\footrulewidth}{0pt}
\fancyhf{}
\fancyhead[RO]{
  \orbitronfont\fontsize{.5cm}{0em}\selectfont\bfseries
  \MakeUppercase{ \parttitle{} | SIXTH WORLD }}
\fancyfoot[RO]{\orbitronfont\fontsize{.5cm}{0em}\selectfont\bfseries\thepage}

\invisiblepart{Introduction}
\section{WELCOME TO SIXTH WORLD}


\begin{multicols}{2}

\SW{} is a “hack” of the game \DW{} which
attempts to capture the flavor of the world of the well-known
RPG \SR{}®.

The “Sixth World” is the dangerous and grim future of our
own world, where magic has resurfaced, megacorporations
rule the world, and humanity has perfected incredible new
technological capabilities including advanced cybernetics and
the worldwide virtual reality network called the
Matrix.

This game assumes familiarity with \SR{}, as well as
with \DW{}.


\subsection{WHAT DO I DO?}

You take on the role of a shadowrunner, an individual who
conducts, let’s say, \textit{quasilegal} activities at the behest of the
corporations, governments, and organized crime. You’ll
choose one of the \textbf{archetypes} described later on to experience the excitement and danger of the shadows of the Sixth
World.


\subsection{FICTION FIRST}

Everything that happens in a session of \SW{}  starts
with the fiction, proceeds to rules (if necessary), and ends
with the fiction. Most of the rules of the game are encapsulated in items called \textbf{moves}. That’s simply game terminology for
a small package of instructions telling you how to attempt to
perform certain actions and resolve them using the rules. So
for instance, the move called \textit{Rock \& Roll} contains instructions
on how to fight with someone.


However, it is important to remember that because the game
starts with and ends with the game fiction, you should never
say “I use Rock \& Roll on that guy!”


In fact, this is a cardinal rule for both players and the GM: \textbf{you
never say the name of your move}. You simply determine,
from what you are doing in the game world (running, shooting, jumping, dying, etc.), what move would apply. When
the rolling is done, you conclude with some more fiction (or
perhaps the GM does, depending on the outcome). Thus the
flow of play is:

\vspace{.1cm}
\begin{adjustwidth*}{1em}{0em}
{\orbitronfont \textbf{FICTIONAL ACTION > RULES > FICTIONAL OUTCOME}}
\end{adjustwidth*}
\vspace{.1cm}

For the most part, it’s what you’ve always done when gaming: try something, roll some dice, and see how it
comes out.


Also remember this: if you do something in the game world
that would trigger a move, then \textit{you must make that move}.
You can’t say “I’m diving into the closet to avoid being spotted” and then \textit{not} make the \textit{Stay Frosty} move. Likewise, you
can’t make a move unless the situation actually demands it.
If you’re not fighting someone who’s fighting back, then you
\textit{don’t} get to make the \textit{Rock \& Roll} move.


Also, when a player does something to trigger a move that
seems questionable given the circumstances, it’s nice to remind them of their situation, and give them a chance to revise what’s happening. As the GM, it’s not your job to nail
them with gotcha moments. Instead, point out the potential
issue you see and let them decide.


A good example of this is the Mage’s \textit{Centering} move. It
simply says ``when you take a moment to concentrate and
restore yourself, hold 2 for future spell casting.'' So all the fiction
\textit{requires} is that the mage stop what they’re doing, take a
moment, and gather their strength. Nothing confusing there.
However, if the mage is in the middle of a firefight, and needs
to center themselves, they might just say “okay, I need to get
things together here...I calm myself and draw on the power
of the astral realm.”


When they do that, remind them that they’re in a firefight, and
based on what happened just \textit{before} they needed to center
themselves, they could be exposed to real danger. Suggest,
for instance, that they dive for cover or get behind something
sturdy before they hit the astral gas pump. This isn’t hand-holding, this is just making sure the fiction is working. If they
say “no, no time, I’ll do it now,” you can decide what kind of
opportunity that gives you, and what you’ll do
about it.


On a related note, since the fiction anchors the game, remember that if you want to speak to or ask something of
Valentin, the character being played by Keith, don’t say “Hey
Keith, do you have a spare frag grenade?” Instead, speak to
the character: “Hey, Valentin, do you have a spare
frag?”


Even though character names should be used, you don’t have
to act in first person. What is important is to remain focused
on the characters. So if the GM says, “Valentin, there’s an ork
with a bat coming your way. What do you do?” Keith is perfectly free to say, “Valentin pulls his trenchcoat aside to show
the gleam of his custom Ares Predator.”


Just remember: flow from the fiction to the rules and back to
the fiction, and stay focused on the characters, and everything will be all right!


\subsection{STATS}
Most of the rules of \SW{}  rely on the value of a player
character’s Stats. You’ll hear more about these later on (especially when you get to the Dossiers on page 8), but every
player character in \SW{}  is described by 5
stats:

\vspace{.1cm}

\begin{adjustwidth*}{.5cm}{.5cm}
{\shadowrunbfont Combat:} your skill in all manner of fighting, both armed
and unarmed

{\shadowrunbfont Stamina:} your physical and mental toughness, strength,
and fortitude

{\shadowrunbfont Awareness:} your alertness, reflexes, and ability to react to
dynamic situations

{\shadowrunbfont Craft:} your general educational level, mastery of specific
objects, and skills

{\shadowrunbfont Presence:} your style, appeal, and charisma
\end{adjustwidth*}

Finally, all characters also have two variable
pools of points:

\vspace{.1cm}

\begin{adjustwidth*}{.5cm}{.5cm}

{\shadowrunbfont Edge:} a pool of points used to activate cyberware, use
magical items and abilities, and sometimes save your life when nothing
else will.

{\shadowrunbfont Essence:} your life force and (meta)humanity, this also fuels the powers of magical archetypes (Adept, Mage, and
Shaman)

\vspace{.1cm}
\end{adjustwidth*}

\subsection{ROLLING THE DICE}
In this game, the dice rolling revolves around the concept of
the Move. When you are instructed to roll dice for a move,
your responsibility is simple: roll 2d6, and add the value of
a stat (or sometimes some other value) to the result. When
a roll is needed, it is usually phrased as ”roll+Something,”
where “something” is the value to add to the roll.

\vspace{.1cm}
\begin{adjustwidth*}{.5cm}{.5cm}
\textbf{Example:} \textit{if you are told to roll+Combat, you would roll
2d6, sum the total, and add the value of your Combat stat
to the result.}
\end{adjustwidth*}
\vspace{.1cm}

The total of the roll indicates the outcome of the action taken
by the character:

\begin{adjustwidth*}{.5cm}{.5cm}
\textbf{On a 10+, you achieve a strong success:} you’ve achieved
your aim without complication, and to the fullest extent
possible.

\textbf{On 7-9, you have achieved a weak success:} your achieve
your aim, but with a cost. You will usually be presented with a list of complications to choose from, although
sometimes instead the GM will tell you what complication
occurs.

\textbf{On a total of 6 or less, you have failed:} you don’t get
what you want. In fact, things are probably going to get
worse.
\end{adjustwidth*}

Note that if a move just says “roll,” then you don’t add anything. You just roll 2d6.

In addition to the common 2d6 roll, \SW{}  uses the 
other common polyhedral dice: \textbf{d4}, \textbf{d6}, \textbf{d8}, \textbf{d10}, and \textbf{d12}. 
Twenty-sided dice are not used for mechanics, but can be 
used for some of the random generators at the end of this 
document. 

\subsubsection{ROLL MODIFIERS }
While the basic move roll is 2d6+(something), there are a few 
modifiers and tricks that may apply to a roll. The rules will 
always indicate when to use one of these
modifiers. 

\begin{adjustwidth*}{.5cm}{.5cm}
\textbf{hold n:} when you are told to Hold \textit{n}, or that you gain \textit{n} 
Hold, this means you have a small pool of points that can 
be spent at some future moment of your choosing. You 
will be told on what, specifically, you may spend the Hold. 
Note that if you can spend Hold on a dice roll, you can do 
so \textit{after} you see the results of the roll! 

\textbf{take +n forward/-n forward:} this means take a bonus
(the +) or a penalty (the -) equal to \textit{n} to your
next Move.

\textbf{take +n ongoing/-n ongoing:} this means to take a bonus or penalty equal to \textit{n} to all of your future rolls, until
whatever circumstances caused the ongoing modifier have
changed.

\textbf{boosted:} whenever you are boosted, your result is never
lower than 7 (even if you roll 6 or less). So, when boosted,
you cannot fail, though success may still come at a cost
(not least of which is the fact that while boosted, you can’t
mark XP).

\textbf{glitched:} glitched rolls are the opposite of boosted rolls.
Whenever you are glitched, your result is never higher
than 9, even if you rolled a 10+. You can succeed while
glitched, but it will always come with a cost.

\textbf{b:} this means “take the best of” - you roll multiple dice,
but keep only one of them to determine the final total.
For instance, if you are instructed to roll 2d6b, you would
roll 2d6, and keep the highest die. When written by itself
(without a dice expression) it will be written as
[b].

\textbf{w:} this means “take the worst of” - if you are instructed to
roll 2d6w, then you would roll 2d6 and keep the lowest
die. When written by itself (without a dice expression), it
will be written as [w].

\end{adjustwidth*}

\subsection{ESSENCE}
Every character in \SW{}  has a stat called \textbf{essence},
representing their humanity, life force, and mystical connection with the world. Essence starts at 6, but the installation
of cybernetic augmentation robs a character of some of that
essence, as they become less human and more
machine.

Characters start with 6 essence, although that may be less
if they choose cyberware. Essence can also be lost to some
creatures and to certain injuries, depending on what optional
rules you have in effect.

\subsection{EDGE}

Each Archetype in this game has a variable pool of points
called \textbf{Edge}. Edge is an in-game currency representing a
number of real-world (or at least, game-world) concepts,
from combat experience to how many jobs they’ve pulled off
 to their ability to turn a bad situation into a survivable one to
their general, flat-out awesomeness.

\subsubsection{SPENDING EDGE}
The main way to spend Edge is to gain bonuses to damage
and to rolls. When a player wishes it, they can spend edge
as follows:

\begin{adjustwidth*}{.5cm}{.5cm}
\textbf{To improve damage:} for every point of Edge spent, they
can add a point of damage to their most recent
attack.

\textbf{To improve a roll:} for every two points of Edge spent,
a character can add one point to the result of their most
recent move.
\end{adjustwidth*}

Edge is also used to:

\begin{adjustwidth*}{.5cm}{.5cm}
\tcirc{} Attuning to magical items (see page 40)

\tcirc{} Investing magical fetishes (see page 40)

\tcirc{} Activating cyberware (see page 38)

\tcirc{} Using magical abilities

\tcirc{} Surviving when things are at their darkest (see the Last
\textbf{Chance} move, page 5)
\end{adjustwidth*}

Feel free to think of other ways that Edge can be spent; just
make sure it’s fun.

\subsubsection{EARNING EDGE}
When Edge is spent, it remains spent until the character has
a chance to spend at least a few hours resting in a place
of relative safety, at which point the pool of Edge refreshes.
Starting characters generally have a relatively small pool of
Edge. However, they will earn more Edge in the course of
their adventures. Players gain additional edge in
2 ways:
\begin{adjustwidth*}{.5cm}{.5cm}
\tcirc{} Choosing to gain a point of edge when they make the
\textbf{Advance} move (page 5)

\tcirc{} Being taken out, but not killed, in combat.
\end{adjustwidth*}

Each player is also free to make a case that \textit{another} player’s
character deserves to earn a point of Edge based on their
actions (successful or not), performance, or whatever other
criteria the player things is worthy. If you’re the GM, don’t
be too harsh here: players rewarding each other for having
a good time and getting into the spirit of things is a \textit{good
thing}. Indulge it!

\subsection{XP}
Characters advance by earning \textbf{XP} (typically called “Marking
XP”) as they navigate their shadowruns. Characters can mark
XP in the following circumstances:

\begin{adjustwidth*}{.5cm}{.5cm}
\tcirc{} when they fail a move (this is the most common reason
XP is marked)

\tcirc{} when they finish a run, or a significant portion of a major
run

\tcirc{} when they resolve one of the debts or
favors they have with another character

\tcirc{} when they are manipulated (see page 4) by another
character
\end{adjustwidth*}

Once a character marks 10 XP, they may use the Advance
move (page 5) to “spend” that XP to improve their character.
Possible improvements include gaining new moves, gaining
more Edge (as mentioned above), or improving a core stat.

\subsection{DEBTS \& FAVORS}
Even in the high-tech world world of the 2050’s, nobody
goes it alone in the shadows for long. Sooner or later, you
need to get help from somebody. Sometimes, you can buy
that help with money. Other times, legal tender won’t cover
it and that’s when debts and favors come into
play.

Together, Debts \& Favors form the \textbf{bond} between runners in
a team. If, at the end of a session, you have resolved one of
these bonds, you erase the debt or favor, and you and the
other runner mark XP.

\subsubsection{DEBT}
A debt is something you owe a fellow runner. Maybe they
yanked your ass out of a bad situation down in Aztlan, or
helped spring you from jail, or just lent you some of their own
hard-won experience that saved your bacon.

\subsubsection{FAVOR}
A favor, conversely, is something owed to you by a fellow
runner. Maybe you were the one doing the hot-LZ extraction
in Aztlan, or you took the rap for them on a particular smash
‘n grab job.

Debts and favors are not necessarily reciprocal! A character
might perceive a debt to another that is entirely self-imposed.
Conversely, a character might feel like one of their teammates
owes them something, while that teammate might be completely unaware of that feeling. So, when establishing debts
and favors, don’t assume that a debt on one sheet has to
correspond to a favor on another!

\end{multicols}

\newpage

\invisiblepart{MOVES}
\section{MOVES}
\begin{multicols}{2}

In \SW{}, the place where rules and fiction intersect are
the character’s \textbf{Moves}. Moves are the mechanical structure
used when the fictional actions of a character require some
resolution, and where the outcome of such actions is sufficiently interesting - or in doubt - as to be worth taking a risk
to achieve.

It is tempting to think of moves as a character’s “powers” or
“abilities,” but remember: you should not be looking for a
move to make. Instead, you should describe fictional actions
that fit the circumstances, and when those actions coincide
with a move, that is the point at which you engage the game
mechanics to determine the outcome.

For example, in a situation where Valentin, a street samurai, is
raiding a military compound, his player should not be looking
to see when he can bust out his Rock \& Roll move. Instead,
Valentin’s player should describe what Valentin is doing, and
if what Valentin is doing would fit the criteria for the Rock \&
Roll move, then the player uses those mechanics. Basically, it
is the difference between this:
\begin{adjustwidth*}{.5cm}{.5cm}

\textbf{GM:} \textit{A security guard moves into
  view. What do you do?}

\textbf{Keith (Valentin’s player):} \textit{I should use Rock \& Roll. I’ll lean
around the corner and shoot.}
\end{adjustwidth*}

and this:
\begin{adjustwidth*}{.5cm}{.5cm}

\textbf{GM:} \textit{A security guard moves into view, gun out, looking
for you. What do you do?}

\textbf{Keith:} \textit{I lean around the corner enough to bring my sights
to bear on him, and unload three rounds from my
HK227.}

\textbf{GM:} \textit{That sounds like the Rock \& Roll move, for sure. Roll
2d6 and add your Combat stat.}
\end{adjustwidth*}

There are four general categories of moves in
\SW{}: \textbf{Core}, \textbf{Secondary}, \textbf{Archetype}, and
\textbf{Metatype}.

\begin{adjustwidth*}{.5cm}{.5cm}
\textbf{Core moves} are the most commonly used moves, and
provide mechanics for frequent activities like
fighting, hiding, looking around, and interacting.

\textbf{Secondary moves} are less frequently used, and are usually
situational.

\textbf{Archetype moves} are moves unique to one
of the character archetypes, and reflect their
particular abilities.

\textbf{Metatype moves} are moves that reflect the differing traits
of the five human metatypes in the game.
\end{adjustwidth*}

Core, secondary, and metatype moves are detailed on the
following pages. Archetype moves can be found in the dossier for each archetype.

\subsection{CORE MOVES}

\textbf{CHECK THE SITUATION:} when you \textbf{assess a situation} or
\textbf{determine facts about your environment}, roll+Awareness.
On 10+, you may ask the GM 3 of the following questions.
On 7-9, ask 1 question. Either way, take +1 if you act on the
answers.
\begin{adjustwidth*}{.5cm}{.5cm}

\tcirc{} What is my best escape/access/evasion route?

\tcirc{} Which enemy is most vulnerable?

\tcirc{} Which enemy is the biggest threat?

\tcirc{} What is my enemy’s true position?

\tcirc{} What should I be on the lookout for here?

\tcirc{} Who’s really in control here?
\end{adjustwidth*}

Note: you may ask any question you wish; however, the GM
is only obligated to give answers the questions from the list
above.

\textbf{DROP SCIENCE:} when you \textbf{call on your knowledge of a
particular subject}, roll+Craft. On 10+, the GM tells you
something useful and interesting about the topic. On 7-9, the
GM simply tells you something interesting.

\textbf{FUCK IT UP / MAKE IT RAIN:} when you \textbf{aid or interfere
with someone you have Bond with}, roll+your Bond with
them. On 10+, they are boosted or glitched, your choice. On
7-9, they’re still boosted or glitched, but you are exposed to
danger or retribution.

\textbf{GUT CHECK:} when you \textbf{check off your 8th wound box},
roll+Stamina. On 10+, you stay on your feet, and if the damage you just received would take you beyond 8 boxes, ignore
any excess. On 7-9, as above, but (choose 2):
\begin{adjustwidth*}{.5cm}{.5cm}

\tcirc{} you are glitched

\tcirc{} you’ll pass out in a few moments

\tcirc{} you’re making it worse; First Aid moves to help you
take -1
\end{adjustwidth*}

On a failure, you collapse unconscious. If you were taken
down by stun damage, you are merely unconscios. Otherwise, you require first aid to stabilize you.

\textbf{MANIPULATE:} when you \textbf{have leverage over someone}
(something they need, want, or wish to hide) \textbf{and wish to
get something from them}, roll+Presence. If the
person is an:
\begin{adjustwidth*}{.5cm}{.5cm}

\textbf{NPC:} On 10+, they’ll ask you for something in return,
but will give you what you need now. On 7-9, they’ll
need to see some concrete assurance you’ll do what
they ask before they help you.

\textbf{PC:} on a 10+, both of the following apply. On 7-9, only
1 applies (you choose):
\begin{adjustwidth*}{.5cm}{.5cm}
\tcirc{} If they comply, they get to mark XP.

\tcirc{} If they refuse, they have to Stay Frosty.
\end{adjustwidth*}
\end{adjustwidth*}

\textbf{MAKE THEM SWEAT:} when you \textbf{impose your will on someone by threat of force}, roll+Combat. On 10+, they comply
without argument. On 7-9, they comply, but (choose
1):
\begin{adjustwidth*}{.5cm}{.5cm}

\tcirc{} They look for payback

\tcirc{} They do only the bare minimum

\tcirc{} They tell someone else about it
\end{adjustwidth*}

\textbf{ROCK \& ROLL:} when you \textbf{attack an
  enemy in melee or at
range}, roll+Combat. Determine the result based on the type
of attack, as follows:
\begin{adjustwidth*}{.5cm}{.5cm}

\textbf{Melee Attacks:} on 10+, you hit and deal damage. On 7-9,
you deal damage, but your target attacks you as
well.

\textbf{Ranged Attacks:} on 10+, you hit and deal damage. On
7-9, you hit, but (choose 1):
\begin{adjustwidth*}{.5cm}{.5cm}

\tcirc{} You need to expose yourself to danger

\tcirc{} You burn up ammunition. Mark off 1 ammo.

\tcirc{} You only graze the target (-2 damage)
\end{adjustwidth*}
\end{adjustwidth*}

\textbf{STAY FROSTY:} when you \textbf{act despite imminent danger,
fear, or risk}, you must roll. The stat you add depends on
how you’re addressing the risk. If you’re:
\begin{adjustwidth*}{.5cm}{.5cm}

\tcirc{} staying alert and reacting quickly,
roll+Awareness

\tcirc{} counting on combat experience and willingness to do
harm, roll+Combat

\tcirc{} hoping you’re tough enough mentally or physically to
weather the storm, roll+Stamina

\tcirc{} banking on your skill or knowledge,
roll+Craft

\tcirc{} flashing a smile or banking on charm,
roll+Presence
\end{adjustwidth*}

On 10+, you succeed. On 7-9, you succeed, but the GM will
present you with a choice: a worse outcome, hard bargain,
or ugly choice.

\textbf{TAKE A BULLET:} when you stand in defense of another,
roll+Stamina. On 10+, the attack hits you intead. On 7-9, you
take half the damage.

\subsection{SECONDARY MOVES}

\textbf{ADVANCE:} when \textbf{you have downtime, and have marked
10 XP}, you can spend time reflecting on your experiences
and honing your skills. When you Advance, choose one of
the following:
\begin{adjustwidth*}{.5cm}{.5cm}

\tcirc{} Advance a stat (each stat may be advanced one time,
fill the small triangle on the dossier when you’ve
advanced a stat)

\tcirc{} Gain a new move from your dossier

\tcirc{} Gain a move from another Archetype’s
dossier

\tcirc{} Gain 1 Edge

\end{adjustwidth*}

You may only choose one benefit each time you advance.
However, you can choose a benefit multiple times, subject
to the limits specified above. Once you have advanced, clear
your XP track.

\textbf{LAST CHANCE:} when \textbf{you are facing death and out of
options}, permanently sacrifice at least 1 Edge and roll+the
amount sacrificed. On 10+, you miraculously make it through,
and it’s not as bad as it looked. On 7-9, you make it through,
but you must agree to a painful bargain. On 6 or less...it’s all
over. Edge sacrificed for this move is gone until you earn it
back; it does not refresh with rest as usual.

\textbf{CITATION NEEDED:} when you \textbf{research a topic, person,
business, or location}, roll+Craft. On 10+, you spend 1 day
searching, and locate a useful detail about the topic of the
research. On 7-9, you locate a useful detail, but
(choose 1):
\begin{adjustwidth*}{.5cm}{.5cm}

\tcirc{} you end up in a rabbit warren of information; spend 1
additional day digging through it

\tcirc{} your search raises a flag in someone else’s systems (the
GM determines whose)

\tcirc{} the information is in hardcopy, and you need to go to it;
spend 1 additional day on the search
\end{adjustwidth*}

\textbf{FIRST AID:} when you \textbf{try to keep a teammate from dying}
from their wounds, roll+Craft. On 10+, you stabilize your
teammate. On 7-9, you stabilize them, but (choose
1):
\begin{adjustwidth*}{.5cm}{.5cm}

\tcirc{} you can’t move them to cover

\tcirc{} you expose yourself to danger (take 2
damage)

\tcirc{} their wounds force you to Stay Frosty
\end{adjustwidth*}

On a failure, your teammate cannot be saved.

\textbf{GO SHOPPING:} when you \textbf{hit the market to buy legal or illegal items}, roll+Presence. On 10+, you find what you need:
if it’s a legal item, you’ll have it in 1 day; illegal items take
2 days. On 7-9, you can get it, but you must wait 1d4 additional days.

\textbf{HIT THE BOOKS:} when you \textbf{spend time training, practicing,
or studying your abilities}, you gain Prep. You gain 1 Prep for
every 2 days spent in training or practice. When that training
and preparation pays off, you can spend 1 Prep to get +1 to
any roll. You can only spend 1 Prep per roll.

\textbf{OVERWATCH:} when you’re \textbf{providing cover for an ally and
a threat appears}, roll+Awareness. On 10+, your ally gets
the drop on the threat. On 7-9, they’re alerted, and take +1
on their next move. On a miss, the threat gets the drop on
your ally.

\textbf{POP PILLS:} when you \textbf{indulge in a drug}, roll+Stamina. On a
10+, you experience the effects as normal. On 7-9, you experience the effects but you got a weak batch, so the effects
last half as long.
If you roll snake eyes when you pop pills, you become addicted to the drug. If you go 3 sessions without a hit, roll 2d6w. If
you roll a 4 or higher, you are no longer addicted; otherwise,
you’re still hooked. If you are an addict and roll snake eyes
while popping pills, you overdose and take 8 Stun.

\textbf{PULL STRINGS:} when you \textbf{hit up a contact for info or assistance}, roll+Presence. On 10+, the contact provides useful
information (related to their own knowledge) or assistance.
On 7-9, the contact provides information or assistance, but
(choose 1):
\begin{adjustwidth*}{.5cm}{.5cm}

\tcirc{} Has to get back to you; wait 1 day

\tcirc{} Isn’t happy about it; take -1 forward to the next time
you Pull Strings with this contact

\tcirc{} Requires a favor in return
\end{adjustwidth*}

If you fail, your contact doesn’t want to see you for a while,
and will not return calls or meet with you for 1d6+1 days.
Repeated failures of this move can permanently sever your
relationship.

\textbf{SUPPRESSION FIRE:} when you \textbf{suppress an area to pin the
enemy down down}, roll+Combat and mark off 2 Ammo. On
10+, the targets are suppressed and cannot move or return
fire. On 7-9, the targets are suppressed, but deal 2 damage
first.






\subsection{METATYPE MOVES}

There are five primary metahuman types (or “metatypes”) in
the \SW{}: \textbf{Human}, \textbf{Dwarf}, \textbf{Elf}, \textbf{Ork}, and \textbf{Troll}, each
with their own unique moves. When you choose your metatype, you also choose one move from the list as your metatype move.

While there are regional differences in the appearance and
nature of metatypes, such as the trollish Oni in Japan and the
elvish Dryad in England, all metahumans have access to the
same moves.

Additionally, if there are other metatypes or species you wish
to add to the game, don’t hesitate: just name the metatype,
and come up with a move or two for it (or just lift one from
the list here).



\subsubsection{HUMAN}
\begin{adjustwidth*}{.5cm}{.5cm}

\textbf{PROFESSIONAL:} choose an area of knowledge or training.
When you Drop Science about that area of expertise, you
are boosted.

\textbf{PRIVILEGE:} when interacting with humans, take +1 to Presence moves.
\end{adjustwidth*}

\subsubsection{DWARF}

All dwarves have natural thermographic vision.

\begin{adjustwidth*}{.5cm}{.5cm}

\textbf{TONIGHT WE DRINK:} if you’re drinking with someone, you
may manipulate someone using Stamina instead of
Rep.

\textbf{NEVER SICK:} you are immune to disease and
poisons.

\textbf{SAVVY:} when you repair or improve machines, you are
boosted.
\end{adjustwidth*}

\subsubsection{ELF}
All elves have natural low-light vision.

\begin{adjustwidth*}{.5cm}{.5cm}

\textbf{UNCANNY GRACE:} once per fight, when you take damage,
you can elect to take -2 forward and reduce damage
by half.

\textbf{ETHEREAL:} when manipulating someone via charm or seduction, you are boosted.
\end{adjustwidth*}

\subsubsection{ORK}

All orks have natural low-light vision.

\begin{adjustwidth*}{.5cm}{.5cm}

\textbf{‘ARD BASTARD:} take +1 to gut checks

\textbf{STREETFIGHTER:} the first time you attack an enemy with a
nonlethal weapon (fists, feet, batons, etc), you
are boosted.

\textbf{FEARLESS:} take +1 to stay frosty in the face of fear.
\end{adjustwidth*}

\subsubsection{TROLL}
All trolls have natural thermographic vision.

\begin{adjustwidth*}{.5cm}{.5cm}

\textbf{DERMAL BONE PLATING:} you have +1 armor.

\textbf{YOU’LL JUST MAKE IT ANGRY:} you gain 1 additional wound
box.

\textbf{JUGGERNAUT:} your fists should be licensed weapons. You
deal lethal damage in unarmed combat.
\end{adjustwidth*}


\subsection{CROSS-ARCHETYPE MOVES}
Archetypes are, in effect, the character classes in \SW{}.
However, the class boundaries are somewhat fungible—you
can “multiclass” to a certain extent.

When you make the Advance move, you have the option of
selecting a move from another archetype. You can choose
moves freely from other archetypes, subject to the following
two restrictions:
\begin{adjustwidth*}{.5cm}{.5cm}

1. You may choose no more than 3 moves from another
archetype.

2. If your character is a non-magical archetype, they may
\textbf{not} select moves that require Essence to be spent. They
may select moves with optional Essence requirements.
Of course, both restrictions are entirely subject to GM and
group discretion.
\end{adjustwidth*}

Restriction \#2, for example, can be modified easily if the group
wishes all characters in their game to have some magical potential. One potential alternative is to permit open “multiclassing,” but limit essence recovery options for non-magical
archetypes (for example, recovering only half your essence
each day, and not being allowed to take the Center or Commune moves).


\end{multicols}

\invisiblepart{CHARACTER CREATION}
\section{CHARACTER CREATION}
\begin{multicols}{2}

Creating a character is a multi-step process (don’t worry,
though, it’s pretty easy). The overall process is described
here; more detail is provided in each Archetype’s dossier.
You’ll record the details you create on the dossier page or the
supplemental “extra info” page located on page 28.

\paragraph{1.  Choose your Archetype}

There are 10 Archetypes to choose from: \textbf{Adept}, \textbf{Face}, \textbf{Ex-
Cop}, \textbf{Hacker}, \textbf{Mage}, \textbf{Mercenary}, \textbf{Rigger}, \textbf{Shaman}, \textbf{Street
Doc}, and \textbf{Street Samurai}. You can learn more about them in
the Archetypes section, page 7.

\paragraph{2.  Choose your Metatype}

There are 5 metatypes: \textbf{Human}, \textbf{Dwarf}, \textbf{Elf}, \textbf{Ork}, and \textbf{Troll}.
Each metatype offers a choice of Metatype Moves. Choose
one move from the \textbf{Metatype Moves} section, page 6.

\paragraph{3.  Choose your Look}

Each character archetype will present options for look; you
are free to make up your own as well.

\paragraph{4.  Choose your Name and Street Name}

Pick a real name and street name. You may use the lists provided in the \textbf{GM Resources} section on page 72, or create
your own.

\paragraph{5.  Assign your Stats}

All characters have the following stats:
\begin{adjustwidth*}{.5cm}{.5cm}

\textbf{Combat:} your skill in all manner of fighting, both armed
and unarmed

\textbf{Stamina:} your physical and mental toughness, strength,
and fortitude

\textbf{Awareness:} your alertness, reflexes, and ability to react to
dynamic situations

\textbf{Craft:} your general educational level, mastery of specific
subjects, and skills

\textbf{Presence:} your style, appeal, and charisma.
\end{adjustwidth*}

All core stats start with a modifier of +0.

\paragraph{6.  Spend your Build Points}

You have \textbf{4 build points} to distribute among your stats. To
increase a stat by 1 point costs 1 Build Point (e.g., it is a
straight 1-for-1 cost).

You may increase a stat to a maximum of +2 as a starting
character. Additionally, if you wish, you may lower one stat
to -1 in order to gain an additional Build Point to spend elsewhere.

\paragraph{7.  Set your Essence and Edge}

Depending on your archetype, you start with a varying
amount of Essence and Edge. Note this amount on your character sheet.

\paragraph{8.  Choose Equipment}

Each archetype will present various weapon, spell, cyberware, and equipment options. Choose from the suggested
items, or if you want to create your own equipment, use the
equipment creation rules starting on page 60 to customize
your kit.

If you choose cyberware, and one of the options provides a
capability you already have (such as thermographic vision),
you may exchange it for any equivalent ability or other item;
just check with the GM.

\paragraph{9.  Choose Contacts}

Everybody knows somebody. You will be presented with a
list of potential contacts your character might know as a result of their experiences both before and after they became
shadowrunners.

\paragraph{10.  Establish Debts and Favors}

In your life before and after becoming a shadowrunner, you’ve
worked with a lot of people, and ended up owing, or being
owed, by them. These relationships include at least one of
your fellow shadowrunners, and are called \textbf{debts} and \textbf{favors}.
When you are instructed to create your debts and favors with
fellow runners, you’ll see a list of sample statements to help
you create them. You don’t have to use these; they’re simply
suggestions.

To create a debt or favor, place the name of one of the other
characters in the blank space in one of the statements presented. You can place the same name more than once (that
is, in more than one sentence), but you must establish at least
one debt or favor to start with.

Collectively, debts and favors are known as \textbf{bonds}. Later,
during play, you may end up resolving a bond with someone.
If you do, both of you mark XP.

\paragraph{11.  Starting Moves}

Your character knows all the Core and Secondary Moves.
You character also knows one or more of his or her Archetype moves. If you are given an option to choose additional
moves, check off the box next to them on the
character sheet.

\paragraph{12.  Advancement}

Each time you fail a roll - that is, you roll a 6 or less - you mark
XP. When you mark 10 XP, and you have downtime, you can
make the \textbf{Advance} move (page 5).


\end{multicols}

\newpage
\switchToLayoutPageB{}
\setlength{\paperwidth}{795pt}
\setlength{\paperheight}{614pt}
\setlength{\textheight}{500pt}
\setlength{\topmargin}{-52pt}

\invisiblepart{DOSSIER : ADEPT}

\section{THE ADEPT}
\begin{multicols}{3}
\setlength{\parskip}{.05cm}

\texttt{>>>When the gift awakened in me, I looked inward. I stud-
ied myself. I saw my limitations - and overcame them. I rec-
ognized my flaws - and accepted them. I reached inward un-
til I held the very heart of my own power, and when I found
it, I switched it on.}

\texttt{Fast, deadly, balanced, I’m an island of focus in the mael-
strom of combat. Some people cannot grasp my true ca-
pabilities. Others don’t understand why I directed your gifts
inward, instead of outward in flashy displays. But
I know why.}

\texttt{Because in the end, when the machine fails, and the magic
dies, I will still have peace.<<<}

\textbf{The Adept} is a magic-user whose power is focused
inward, unlocking their full physical potential. Realized
in the form of performance, speed, and endurance at
or exceeding the peak of human capability, mastery
of martial combat, and total control of self, the adept’s
inner calm and perfected body are the envy of
many.

\subsection{CREATING AN ADEPT}

\paragraph{1.  Choose your Metatype}

You may choose \textbf{Human}, \textbf{Dwarf}, \textbf{Elf}, \textbf{Ork}, or
\textbf{Troll}. Each metatype offers a selection of meta-
type moves. Choose one metatype move from
the options presented.

\paragraph{2.  Choose your look}

\textit{Wise eyes, wary eyes, glowing eyes}

\textit{No hair, cropped hair, long braid}

\textit{Clean skin, tattooed skin, hard skin}

\textit{Perfect body, heavy body, lithe body}

\paragraph{3.  Choose your name and street name}

Make up a name and street name or pick a real
name and street name from the lists and name
generators starting in the \textbf{GM Resources} section.

\paragraph{4.  Assign your stats}

You have 5 stats: Awareness, Combat, Stamina,
Craft, and Presence. Important stats for you are
Awareness, Craft, and Presence.

You have 4 \textbf{Build Points} to distribute among
your stats. To increase a stat by 1 point costs 1
Build Point. You may increase a stat to a maxi-
mum of +2 as a starting character. If you wish,
you may lower 1 stat to -1 in order to have an
additional point to spend.

\paragraph{5.  Choose your Equipment}

Choose from the lists below, or customize your
own gear using the rules in \textbf{Creating Gear} on
page 60.

\textbf{Armor:} \textit{leather armor, Arcana armor}

\textbf{Weapons:} \textit{paired Ares Predators, katana, bo
staff, paired combat knives, compound bow}

\paragraph{6.  Set your Essence and Edge.}

You start with 6 Essence and 3 Edge.

\paragraph{7.  Choose 2 Contacts}

Temple master, gunsmith, underground fight
club organizer, tea shop owner, yakuza soldier,
fetishmonger

\paragraph{8.  Establish debts and favors}

Place one of your fellow runners’ names in at
least one of the blanks in the \textbf{Debts \& Favors}
section of your playbook. Each time a name
appears in a debt or favor, it counts as 1 Bond
with that character. The more people you have
Bond with, the better.

\paragraph{9.  Starting Funds}

You start play with 3d6 x 250¥ immediately
available.

\paragraph{10.  Starting Moves}

You know all the Core and Secondary Moves.
You also know the \textbf{Mystic Body} move, and
one other Adept move.

\end{multicols}

\newpage

\begin{dossier}
\dossierstatbar{THE ADEPT}
\hspace{.5cm}%
\vrule width 2pt
\hspace{.3cm}%
\begin{dossiermovebar}
\fontsize{9pt}{1em}\selectfont
\setlength{\parskip}{.2cm}
\selectedMove{Mystic Body:} when taking damage
from magical effects, your body counts as natural
mystic armor rated at your Essence divided by
three (rounded down). This move makes you eligible
to take additional Adept moves, though you may not
take more than your Essence stat. If your essence
is reduced, you may lose one or more Adept moves.

\unselectedMove{Enhanced Ability:} when you concentrate on enhancing your abilities, spend 1 Edge 
and roll+the stat you wish to enhance. On 10+, increase that stat by 1 point until the end of 
the current scene or encounter. On a 7-9, increase any stat by 1, but reduce another stat by 1 
for the equivalent time period. 

\unselectedMove{Killing Hands:} when you deal damage while unarmed, spend 1 Edge to deal lethal 
  damage instead of stun. 

  \unselectedMove{Danger Sense:} when you open your mind to the world of subtle mundane and magical 
    information in your environment, spend 1 Edge and roll+Awareness. On 10+, you cannot 
    be surprised. On 7-9, take +1 to Stay Frosty. 

    \unselectedMove{Submission Hold:} when you would deal damage to an enemy in melee, you may 
      instead forgo damage and subdue your opponent. If the opponent would normally be dangerous to touch (such as a fire elemental), you may spend 1 Edge to negate that danger. 

      \unselectedMove{Perfect Control:} when you wish to control your emotions, reactions, and nonverbal 
        cues, roll+Presence. On 10+, you can
        demonstrate any emotion or reaction you
        wish, indistinguishable from a genuine
        response. On 7-9, you maintain control,
        but (choose 1): 
\begin{moveoptions}
        \moveoption{ it is exhausting; take 1 stun }

        \moveoption{ your control will only last for a
        short time }

        \moveoption{ your effort is detectable by magical means }
\end{moveoptions}

        \unselectedMove{The Sight:} when you take time to study an enemy, roll+Awareness. On 10+, take +1 
          forward or take +2 damage forward to your next melee attack. On 7-9, take +1 forward. 

          \unselectedMove{Astral Projection:} when you project your spirit into astral space, spend 1 Edge and 
            roll+Craft. On 10+, you project successfully. On 7-9, you project, but your connection is ten- 
            uous; take -1 ongoing while in astral space. While in astral space, roll+Craft for all actions. 

            \unselectedMove{Iron Skin:} when you take damage, you may spend Edge 1-for-1 to reduce the dam- 
              age. 

              \unselectedMove{Unbreakable Defense:} when you Rock \& Roll, take +1 armor forward. 

                

\end{dossiermovebar}%
\end{dossier}

%%% Local Variables: 
%%% mode: latex
%%% TeX-master: "sixth_world"
%%% End: 

\invisiblepart{DOSSIER : EX-COP}

\section{THE EX-COP}
\begin{multicols}{3}
\setlength{\parskip}{.05cm}

\texttt{>>>Years on the job, and now what am I doing? Running
the shadows. Shit, I used to throw skels like myself in jail ev-
ery day. On the other hand, the pay is better than anything
I made on the force, I get to meet interesting people, and it
beats corporate rent-a-cop work.}

\texttt{Some of these folks, they think because they’ve got the wires,
or the mojo, they can walk circles around me. And yeah,
maybe so, if I ever let them have a level playing field. But I
still think like a cop, and I know the system. People still on the
job are happy to help an old buddy.}

\texttt{And while the badge may not be entirely official anymore,
there’s always the gun.<<<}

\textbf{The Ex-Cop} comes from Lone Star, Knight Errant, the
military police, or any one of many law enforcement
agencies in the confused landscape of the 2050’s.
Possessed of a keen investigative mind, brutally effec-
tive combat skills, experience with the best and worst
of humanity, and connections deep into “the system,”
the ex-cop is a valuable asset.


\subsection{CREATING AN EX-COP}

\paragraph{1.  Choose your Metatype}

You may choose \textbf{Human}, \textbf{Dwarf}, \textbf{Elf}, \textbf{Ork}, or
\textbf{Troll}. Each metatype offers a selection of meta-
type moves. Choose one metatype move from
the options presented.

\paragraph{2.  Choose your look}

\textit{Cold eyes, tired eyes, wary eyes}

\textit{Close cropped hair, shaggy hair, bald}

\textit{Cheap suit, street clothes, hawaiian shirt}

\textit{Heavy body, fit body, injured body}

\paragraph{3.  Choose your name and street name}

Make up a name and street name or pick a real
name and street name from the lists and name
generators starting in the \textbf{GM Resources} section.

\paragraph{4.  Assign your stats}

You have 5 stats: Awareness, Combat, Stamina,
Craft, and Presence. Important stats for you are
Craft, Presence, and Combat.

You have 4 \textbf{Build Points} to distribute among
your stats. To increase a stat by 1 point costs 1
Build Point. You may increase a stat to a maxi-
mum of +2 as a starting character. If you wish,
you may lower 1 stat to -1 in order to have an
additional point to spend.

\paragraph{5.  Choose your Equipment}

Choose from the lists below, or customize your
own gear using the rules in \textbf{Creating Gear} on
page 60.

\textbf{Armor:} \textit{armor vest, form-fitting armor}

\textbf{Service Pistol:} \textit{Ruger Super Warhawk, Colt
Manhunter}

\textbf{Additional Weapon:} \textit{HK 227,
  Remington 990}

\paragraph{6.  Choose your cyberware}

You may start with one of the following cyber-
ware kits (descriptions of these items are on
page 45):

\textbf{Kit 1 (3 essence):} \textit{smartlink, bone lacing}

\textbf{Kit 2 (3 essence):} \textit{cybereyes with low-light
and flare compensator, level 1 skillwires}


\paragraph{7.  Set your Essence and Edge.}

To determine your starting Essence, subtract the
essence cost of your cyberware (if any) from 6.

You start with 3 Edge.

\paragraph{8.  Choose 3 Contacts}

Confidential informant (CI), precinct secretary,
gang leader, prosecutor, journalist, former part-
ner, defense attorney

\paragraph{9.  Establish debts and favors}

Place one of your fellow runners’ names in at
least one of the blanks in the \textbf{Debts \& Favors}
section of your playbook. Each time a name
appears in a debt or favor, it counts as 1 Bond
with that character. The more people you have
Bond with, the better.

\paragraph{10.  Starting Funds}

You start play with 3d6 x 250¥ immediately
available.

\paragraph{11.  Starting Moves}

You know all the Core and Secondary Moves.
You also know the \textbf{Gumshoe} move, and
one other Ex-Cop move.

\end{multicols}

\newpage


\begin{dossier}
\dossierstatbar{THE EX-COP}
\hspace{.5cm}%
\vrule width 2pt
\hspace{.3cm}%
\begin{dossiermovebar}
\fontsize{9pt}{1em}\selectfont
\setlength{\parskip}{.05cm}


\selectedMove{Gumshoe:} when you examine the scene of an event, or interrogate someone about an
event, roll+Craft. On 10+, pick two of the following to learn (relevant to what you’re investi-
gating). On 7-9, pick one:
\begin{moveoptions}
\moveoption{ Scene: when the events happened; whether magic was involved; how many individuals
were involved; if this is the primary scene of the
event}

\moveoption{ Person: if they’re connected to the event; whether they’re hiding something; what they
stood to lose or gain; a useful personal detail
(e.g, a tic, handedness, etc.)}
\end{moveoptions}

\unselectedMove{Work the System:} when you use your ex-LEO status to get help, roll+Presence. On
10+, you have an old pal jam somebody up or cut them a break. On 7-9, you get the de-
sired result, but (choose 1):
\begin{moveoptions}
\moveoption{ the person knows who helped or hindered
them}

\moveoption{ your buddy got in trouble}

\moveoption{ your name got mentioned to the wrong ears}
\end{moveoptions}

\unselectedMove{Takedown:} when you take control of a person physically, roll+Combat. On 10+, they
are under your complete control, and you are both unharmed. On 7-9, you gain control of
them, but either you or your target must take 2
damage.

\unselectedMove{Interrogation:} when you attempt to make someone sweat, you may roll+Craft instead
of +Presence.

\unselectedMove{Deep Cover:} when you Stay Frosty to blend in to a criminal environment, you are
boosted.

\unselectedMove{Good Cop, Bad Cop:} when you aid someone you have bond with during an interroga-
tion, roll+Craft instead of +Bond.

\unselectedMove{Gun Cage:} when you need some specialized weapons or hardware fast, you can bor-
row it from a buddy on the force. When you return it, roll+Presence (subtracting 1 for every
day beyond the first that you’ve had the gear). On 10+, everything’s fine. On 7-9, you’ve
raised suspicions and can’t use this move for your next run. On 6 or less, you owe your
buddy a favor before you get trusted with the keys
again.

\unselectedMove{The Feds:} you have a contact in federal law enforcement. Roll+Presence. On 10+, pick
2. On 7-9, pick 1.
\begin{moveoptions}
\moveoption{ You get a tip-off on a big operation so
you can steer clear}

\moveoption{ You gain interesting and useful
information about your current run}

\moveoption{ You get access to federal data on an
individual}

\moveoption{ You are listed as a “consultant” on a
case}
\end{moveoptions}

\unselectedMove{Doorkicker:} when you lead the team in an assault on the enemy, roll+Combat. On 10+,
designate up to 3 enemies who are surprised. On
7-9, designate up to 2 enemies.

\unselectedMove{Hostage Negotiator:} when you negotiate in a dangerous situation, you may
roll+Combat instead of +Presence. You must still have leverage to negotiate.

\end{dossiermovebar}%
\end{dossier}

%%% Local Variables: 
%%% mode: latex
%%% TeX-master: "sixth_world"
%%% End: 

\invisiblepart{DOSSIER : FACE}

\section{THE FACE}
\begin{multicols}{3}
\setlength{\parskip}{.05cm}

\texttt{>>>I could have been on the trid - I’ve got the looks. And
half the megacorps in Seattle would kill to get me in an in-
terview. But why tie myself down like that? I have a particular
set of talents that makes me incredibly valuable in shadow-
running circles, and to be completely honest, I’m hooked on
the adrenaline.}

\texttt{It’s a rush to be someone else, to read someone’s tics and
cues, and to run a con so effective that the mark never even
figures out it happened. It’s good when it goes right. So
good.}

\texttt{On the other hand, you have to be careful who you con. You
don’t con your team. Why? I sometimes ask myself the same
thing. But then...well, lemme make a long story short. You see
this scar...?<<<}

\textbf{The Face} is the professional front of the shadowrun-
ning team. When a deal is being negotiated, the Face
is front and center. However, the Face is also a pro-
fessional con, and a master of disguise, misdirection,
and interpersonal relations. A team without the Face is
at a disadvantage in dealing with potential employers
and rivals, and with a few phone calls, the Face makes
getting into and out of any operation easier.



\subsection{CREATING A FACE}

\paragraph{1.  Choose your Metatype}

You may choose \textbf{Human}, \textbf{Dwarf}, \textbf{Elf}, \textbf{Ork}, or
\textbf{Troll}. Each metatype offers a selection of meta-
type moves. Choose one metatype move from
the options presented.

\paragraph{2.  Choose your look}

\textit{Wise eyes, jeweled eyes, laughing eyes}

\textit{Normal skin, perfect skin, synthetic skin}

\textit{Great smile, smoky stare, rugged good looks,
regal bearing}

\textit{Fit body, compact body, androgynous body}

\paragraph{3.  Choose your name and street name}

Make up a name and street name or pick a real
name and street name from the lists and name
generators starting in the \textbf{GM Resources} section.

\paragraph{4.  Assign your stats}

You have 5 stats: Awareness, Combat, Stamina,
Craft, and Presence. Important stats for you are
Awareness, Presence, and Craft.

You have 4 \textbf{Build Points} to distribute among
your stats. To increase a stat by 1 point costs 1
Build Point. You may increase a stat to a maxi-
mum of +2 as a starting character. If you wish,
you may lower 1 stat to -1 in order to have an
additional point to spend.

\paragraph{5.  Choose your Equipment}

Choose from the lists below, or customize your
own gear using the rules in \textbf{Creating Gear} on
page 60.

\textbf{Armor:} \textit{armorweave clothing, form fitting
armor, light armor jacket}

\textbf{Weapon:} \textit{Colt L36, Beretta 101T, stun baton,
taser}

\paragraph{6.  Choose your cyberware}

You may start with one of the following cyber-
ware kits (descriptions of these items are on
page 45):

\textbf{Kit 1 (2 essence):} \textit{cybereyes with thermo-
graphic vision, voice modulator}

\textbf{Kit 2 (3 essence):} \textit{FeatherTouch system, level
1 skillwires}


\paragraph{7.  Set your Essence and Edge.}

To determine your starting Essence, subtract the
essence cost of your cyberware (if any) from 6.

You start with 4 Edge.

\paragraph{8.  Choose 4 Contacts}

Club owner, Yakuza boss, car dealer, journalist,
senator’s aide, money launderer, mafia capo,
arms dealer, wealthy socialite


\paragraph{9.  Establish debts and favors}

Place one of your fellow runners’ names in at
least one of the blanks in the \textbf{Debts \& Favors}
section of your playbook. Each time a name
appears in a debt or favor, it counts as 1 Bond
with that character. The more people you have
Bond with, the better.

\paragraph{10.  Starting Funds}

You start play with 3d6 x 350¥ immediately
available.

\paragraph{11.  Starting Moves}

You know all the Core and Secondary Moves.
You also know the \textbf{Razor Insight} move, and
one other Face move.

\end{multicols}

\newpage

\begin{dossier}
\dossierstatbar{THE FACE}
\hspace{.5cm}%
\vrule width 2pt
\hspace{.3cm}%
\begin{dossiermovebar}
\fontsize{8.5pt}{1em}\selectfont
\setlength{\parskip}{.05cm}


\selectedMove{Razor Insight:} when you have a casual conversation with someone, roll+Awareness. On
10+, you learn three of the following things. On
7-9, you learn 2.
\begin{moveoptions}
\moveoption{ Something they love}

\moveoption{ Something they lost}

\moveoption{ Something they fear}

\moveoption{ Something they took}

\moveoption{ Something they need}
\end{moveoptions}
If you use this information when fast talking, manipulating, or making them sweat, you are
boosted.

\unselectedMove{Fast Talk:} when you need to convince somebody of something fast, roll+Presence. On
10+, your quick thinking gets you through. On 7-9,
they’re convinced, but (choose 1)

\begin{moveoptions}
\moveoption{ they check up on your story later}

\moveoption{ they get in serious trouble for believing
you}

\moveoption{ one of your contacts somehow ends up
involved...in a bad way}
\end{moveoptions}

\unselectedMove{Work the Angles:} when
you manipulate someone, take +1.

\unselectedMove{Come Hither:} when you attempt to seduce someone, roll+Presence. On 10+, they’re
into you, and you can get a favor from them or get access to some of their personal stuff.
On 7-9, they’re into you and will provide minor help, but it will take some more time and
TLC to get a favor from them.

\unselectedMove{Build a Legend:} when you create a false identity, spend 1 day working on it and
roll+Craft. On 10+, your legend is solid and will hold up to any scrutiny. On 7-9, it holds up
for now, but (choose 1):

\begin{moveoptions}
\moveoption{ you’ve only got 1d4+Craft days before its
blown}

\moveoption{ you run into someone who knows you...as
someone else.}

\moveoption{ you have to do something unpleasant to
maintain your cover.}
\end{moveoptions}

\unselectedMove{Crazy Smooth:} when you
Fast Talk, you are boosted.

\unselectedMove{I Know A Guy:} when you need an illegal good or service, roll+Presence. On 10+, you
know someone who can get it for you immediately, and discreetly. On 7-9, they can get it,
but (choose 1):

\begin{moveoptions}
\moveoption{ it takes 1 additional day}

\moveoption{ it costs twice as much as predicted}

\moveoption{ your fence has to drop your name to get}
it
\end{moveoptions}

\unselectedMove{Honeyed Words:} when you make someone sweat, you may roll+Presence instead of
Combat.

\unselectedMove{Chameleon:} when you attempt to blend in to a social environment, roll+Presence. On
10+, nobody questions your presence. On 7-9, you catch the eye of someone who becomes
curious about what you’re doing there.

\unselectedMove{Irresistible:} even if you anger, insult, or otherwise tick off a contact, they just can’t stay
mad at you. They only avoid you for half as long as normal.

\end{dossiermovebar}%
\end{dossier}

%%% Local Variables: 
%%% mode: latex
%%% TeX-master: "sixth_world"
%%% End: 

\invisiblepart{DOSSIER : HACKER}

\section{THE HACKER}
\begin{multicols}{3}
\setlength{\parskip}{.05cm}

\texttt{>>>These chromers and spellworms are missing the point.
They’re in this for money, looking to retire someday? Hah.
They’ve got no idea where the power is. Real power lies in
a world most of them take for granted. But it’s a world I live
and breathe. You want paydata? I know where it is. You want
me to shut some shit down? I can do that. You want me to
hack Renraku? Give me a dataline. I’ll do it. I dream in code,
babe. I can see the girl in the red dress.}

\texttt{And don’t tell anyone, but this? I do it for fun. You should see
me when I’m serious.<<<}

\textbf{The Hacker} is the master of the worldwide virtual re-
ality network of the Matrix. Able to bend the Matrix
their will, the Hacker is a critical member of the team.
From finding crucial data on targets, to locating floor-
plans of facilities, to shutting down security systems
and sabotaging response efforts, the hacker’s value is
indisputable.


\subsection{CREATING A HACKER}

\paragraph{1.  Choose your Metatype}

You may choose \textbf{Human}, \textbf{Dwarf}, \textbf{Elf}, \textbf{Ork}, or
\textbf{Troll}. Each metatype offers a selection of meta-
type moves. Choose one metatype move from
the options presented.

\paragraph{2.  Choose your look}

\textit{Cybereyes, glasses, unfocused eyes}

\textit{No hair, unkempt hair, mohawk, ponytail}

\textit{Pale skin, bad skin, tattooed skin}

\textit{Thin body, heavy body, compact body, flabby
body}

\paragraph{3.  Choose your name and street name}

Make up a name and street name or pick a real
name and street name from the lists and name
generators starting in the \textbf{GM Resources} section.

\paragraph{4.  Assign your stats}

You have 5 stats: Awareness, Combat, Stamina,
Craft, and Presence. Important stats for you are
Awareness, Craft, and Combat.

You have 4 \textbf{Build Points} to distribute among
your stats. To increase a stat by 1 point costs 1
Build Point. You may increase a stat to a maxi-
mum of +2 as a starting character. If you wish,
you may lower 1 stat to -1 in order to have an
additional point to spend.

\paragraph{5.  Choose your Equipment}

Choose from the lists below, or customize your
own gear using the rules in \textbf{Creating Gear} on
page 60.

\textbf{Armor:} \textit{trenchcoat, light armor jacket}

\textbf{Weapon:} \textit{Fichetti Needler, Ares Lightfire 70,
Combat Axe, Remington 990}

\paragraph{6.  Build your deck}

Choose one of the cyberdecks below, or con-
struct your cyberdeck using the \textbf{Creating Cyber-
decks} rules located on page 62.

\textbf{Cyberdeck:} \textit{Fuchi Cyber-4, Fuchi Cyber-7}

\paragraph{7.  Write your Programs}

Choose 3 programs from the list on page 38,
or using the rules in \textbf{Writing Programs} (page
66), create the software you wish to run on
your deck. You have 8 points to spend purchas-
ing routines for your programs. Each routine you
purchase costs 1 point.



\paragraph{8.  Choose your cyberware}

You may start with one of the following cyber-
ware kits (descriptions of these items are on
page 45):

\textbf{Kit 1 (3 essence):} \textit{cybereyes with low-light,
synaptic hardening, datajack}

\textbf{Kit 2 (2 essence):} \textit{headware cyberdeck, data-
jack}


\paragraph{9.  Set your Essence and Edge.}

To determine your starting Essence, subtract the
essence cost of your cyberware (if any) from 6.

You start with 3 Edge.

\paragraph{10.  Choose 2 Contacts}

GhostSyndicate, electronics dealer, military
hacker, gang member, former professor, ma-
trix guru, white hat, script kiddie, poker dealer,
money launderer

\paragraph{11.  Establish debts and favors}

Place one of your fellow runners’ names in at
least one of the blanks in the \textbf{Debts \& Favors}
section of your playbook. Each time a name
appears in a debt or favor, it counts as 1 Bond
with that character. The more people you have
Bond with, the better.

\paragraph{12.  Starting Funds}

You start play with 3d6 x 150¥ immediately
available.

\paragraph{13.  Starting Moves}

You know all the Core and Secondary Moves.
You also know the \textbf{Born Digital} and \textbf{Sling
Code} moves.

\end{multicols}

\newpage

\begin{dossier}
\dossierstatbar{THE HACKER}
\hspace{.5cm}%
\vrule width 2pt
\hspace{.3cm}%
\begin{dossiermovebar}
\fontsize{9pt}{1em}\selectfont
\setlength{\parskip}{.1cm}

\selectedMove{Born Digital:} while in
the Matrix, when you:
\begin{moveoptions}
\moveoption{ Stay Frosty, add your deck’s Mask rating
to the roll}

\moveoption{ Take damage, subtract your deck’s
Hardening rating from the damage}

\moveoption{ Rock \& Roll, roll+Craft instead of
+Combat}

\moveoption{ Deal damage, add your deck’s CPU to the
damage.}
\end{moveoptions}

\selectedMove{Sling Code:} when you hack a Matrix node or device, roll+Craft. On 10+, choose 3. On
7-9, choose 2:
\begin{moveoptions}
\moveoption{ The node or device is unaware of the
intrusion}

\moveoption{ You leave no trace of your presence}

\moveoption{ You don’t trigger IC}

\moveoption{ You learn a useful detail about another
node connected to this one}
\end{moveoptions}

Once in control of a node, you can issue commands
appropriate to it.

\unselectedMove{Overwatch:} when you defend a device or node against a matrix attack, roll+Aware-
ness. On 10+, the attack is ineffective. On 7-9, halve the damage or duration of the attack’s
effect.

\unselectedMove{IC Killer:} when you
inflict damage to IC, inflict +1 damage.

\unselectedMove{Interference:} when you use a device or device node you have hacked to interfere with
the enemy, Hold 1 to grant to a teammate when they
Rock \& Roll or Stay Frosty.

\unselectedMove{Deathmatch:} when you deal damage to enemy hackers, add your Combat stat to the
damage.

\unselectedMove{Multitasker:} you can
hack multiple systems or devices
simultaneously. Roll+Awareness. On 10+, you suffer no penalties to hack two systems. On 7-9, take -1 ongoing to the second
system.

\unselectedMove{Compile Agent:} you can integrate up to 6 storage worth of programs currently in
your cyberdeck’s storage into an independent matrix entity called a Agent. Only one Agent
can be active at a time. Agents have the following
characteristics:
\begin{moveoptions}
\moveoption{ CPU: allocate at least 1 point from your
cyberdeck’s CPU to the Agent’s CPU}

\moveoption{ Wounds: a Agent’s wounds are equal to its
size in storage units}

\moveoption{ Moves: Agents use the Sling Code and Born
Digital moves, substituting CPU for Craft.}

\moveoption{ Other Stats: any other stats a Agent depend on its constituent programs (e.g., if a con-
stituent program has the Armor routine, the Agent
has Armor 1)}
\end{moveoptions}

\unselectedMove{Tracer:} when you would deal damage to an enemy hacker in Matrix combat, you can
instead forgo damage to plant a tracer tag on their avatar. This tracer is active for 1+Craft
days.

\unselectedMove{Subversion:} when you would defeat an intrusion countermeasure, instead of destroying it you can subvert it’s programming and send it at an enemy. Roll+Craft. On 10+, it will
attack the enemy twice before crashing. On 7-9, it will attack once.

\end{dossiermovebar}%
\end{dossier}

%%% Local Variables: 
%%% mode: latex
%%% TeX-master: "sixth_world"
%%% End: 

\invisiblepart{DOSSIER : MAGE}

\section{THE MAGE}
\begin{multicols}{3}
\setlength{\parskip}{.05cm}

\texttt{>>>It’s not easy to study these formulae. Trust me, it’s like
learning a language spoken by creatures with ten mouths,
twelve eyes, and a tonal language based on what the color
nine smells like. If you haven’t got the gift, well...if you’re
lucky, it’ll look like gibberish. If you’re unlucky, it might just
bust your head. But do you know what it’s like to turn invis-
ible, to throw lightning from your hands, or to heal injuries
with a word? To be the artillery when a run goes south hard?
You know what it’s like?}

\texttt{It’s a little like being a god.<<<}

\textbf{The Mage}’s magical talent is focused on the Arcana art
of spellcasting - employing esoteric formulas, incanta-
tions, and the precepts of magical theory to shape
reality itself. If you want an Arcana artillery company,
someone to cloak the entire team in magical invisi-
bility, or someone to provide astral overwatch for the
whole team, look to the mage.


\subsection{CREATING A MAGE}

\paragraph{1.  Choose your Metatype}

You may choose \textbf{Human}, \textbf{Dwarf}, \textbf{Elf}, \textbf{Ork}, or
\textbf{Troll}. Each metatype offers a selection of meta-
type moves. Choose one metatype move from
the options presented.

\paragraph{2.  Choose your look}

\textit{Blank eyes, unnatural eyes, piercing eyes}

\textit{Long hair, bald, wild hair}

\textit{Robes, street clothes, dress clothes}

\textit{Thin body, weak body, muscular body}

\paragraph{3.  Choose your name and street name}

Make up a name and street name or pick a real
name and street name from the lists and name
generators starting in the \textbf{GM Resources} section.

\paragraph{4.  Assign your stats}

You have 5 stats: Awareness, Combat, Stamina,
Craft, and Presence. Important stats for you are
Craft, Awareness, and Stamina.

You have 4 \textbf{Build Points} to distribute among
your stats. To increase a stat by 1 point costs 1
Build Point. You may increase a stat to a maxi-
mum of +2 as a starting character. If you wish,
you may lower 1 stat to -1 in order to have an
additional point to spend.

\paragraph{5.  Choose your Equipment}

Choose from the lists below, or customize your
own gear using the rules in \textbf{Creating Gear} on
page 60.

\textbf{Armor:} \textit{trenchcoat, light armor jacket, armor
charm}

\textbf{Weapon:} \textit{Beretta 101T, Ruger Super War-
hawk, Staff}

\paragraph{6.  Craft your Spells}

Choose 3 of the following 5 spell categories:

\textit{Combat, Detection, Illusion, Health, Manipulation}

You know 2 spells in one of your chosen catego-
ries, 1 in the second category, and 1 in the final
category. Either choose from the lists of spells on
page 40, or create your spells according to the
\textbf{Spellcrafting} rules on page 67.

\paragraph{7.  Set your Essence and Edge.}

You start with 6 Essence and 3 Edge.

\paragraph{8.  Choose 2 Contacts}

Wage Mage, Corporate Exec, Fetishmonger,
Paranormal Animal Expert, Bartender, Street
Cop, Professor of Magical Theory

\paragraph{9.  Establish debts and favors}

Place one of your fellow runners’ names in at
least one of the blanks in the \textbf{Debts \& Favors}
section of your playbook. Each time a name
appears in a debt or favor, it counts as 1 Bond
with that character. The more people you have
Bond with, the better.

\paragraph{10.  Starting Funds}

You start play with 3d6 x 250¥ immediately
available.

\paragraph{11.  Starting Moves}

You know all the Core and Secondary Moves.

You also know the \textbf{Cast a Spell} and \textbf{Counterspell} moves.

\end{multicols}

\newpage


\begin{dossier}
\dossierstatbar{THE MAGE}
\hspace{.5cm}%
\vrule width 2pt
\hspace{.3cm}%
\begin{dossiermovebar}
\fontsize{9pt}{1em}\selectfont
\setlength{\parskip}{.01cm}

\selectedMove{Cast a Spell:} When you cast a spell, choose the spell's force and roll. What stat you add 
depends on the spell type: 
\begin{moveoptions}
\moveoption{ \textbf{Combat:} roll+Combat }

\moveoption{ \textbf{Health:} roll+Stamina }

\moveoption{ \textbf{Detection:} roll+Awareness }

\moveoption{ \textbf{Manipulation:} roll+Craft }

\moveoption{ \textbf{Illusion:} roll+Presence }
\end{moveoptions}

For every 3 points of force, take -1 to the roll. The force can not exceed twice your essence.

On 10+, you cast the spell without trouble. On 7-9, you cast the spell, but (choose 1): 
\begin{moveoptions}

\moveoption{ it causes drain; take the spell's force divided by 2 as stun. If the force is greater than your current essence, the damage is physical. }

\moveoption{ it causes astral feedback; you are glitched for the next spell you cast }

\moveoption{ you must expose yourself to danger or an attack to cast the spell }
\end{moveoptions}

\selectedMove{Counterspell:} to disrupt or end a spell cast by another towards yourself or an ally, roll+Craft. 
On 10+, the spell is dispelled. On 7-9, the disruption was draining; take 2 stun.

\unselectedMove{Center:} when you take a moment to concentrate and restore yourself, hold 2 for future spell casting or counterspelling.

\unselectedMove{Specialist:} choose a spell category.You take +1 when casting spells from that category. 

    \unselectedMove{Astral Trace:} when you observe a magical effect for which you cannot determine the 
      source, roll+Awareness. On a 10+, the GM answers three of the following. On 7-9, two: 
\begin{moveoptions}

      \moveoption{ In what direction does the source of this magic lie? }

      \moveoption{ Approximately how far away is the source? }

      \moveoption{ What is the general nature of the source (metahuman, astral, para-animal)? }

      \moveoption{ How powerful is the source? }
\end{moveoptions}

      \unselectedMove{Imbue Focus:} when you craft a magical focus, permanently reduce your Edge by 1, and roll+Craft. On 10+, 
        the focus is successfully created. On 7-9, the focus is created, but you must spend twice as 
        much Edge to craft it. Imbuing a focus takes 1 day. 

        \unselectedMove{Hermetic Library:} you have permission to access an extensive library of hermetic lore. 
          When you or a teammate uses the Citation Needed move to research magical history or 
          theory, the move is boosted. 

\unselectedMove{Mana Barrier:} to construct a mana barrier, roll+Craft. On 10+ the barrier is formed. On 7-9, the barrier is formed, but you suffer 2 stun as drain. If created, the barrier's force is equal to your roll-6.

\unselectedMove{Quickening:} when sustaining a spell, you may spend Edge equal to the spell's force to create a self-sustaining spell. This eliminates it's sustaining penalty, but ties up the Edge spent until the spell is dismissed or disrupted.

\unselectedMove{Absorption:} when you successfully counterspell, roll+Stamina. On 10+, hold +2 for your next spell casting move. On 7-9, hold +1. 


\begin{comment}
             \unselectedMove{Initiate:} when you hit the books, you may also spend the Prep earned to: 
\begin{moveoptions}

              \moveoption{ reduce a spell’s Essence cost by 1 (this can reduce the cost to 0) }

                \moveoption{ boost a Cast a Spell move }

                \moveoption{ restore 1 essence }
\end{moveoptions}
\end{comment}



\end{dossiermovebar}%
\end{dossier}

%%% Local Variables: 
%%% mode: latex
%%% TeX-master: "sixth_world"
%%% End: 

\invisiblepart{DOSSIER : MERCENARY}

\section{THE MERCENARY}
\begin{multicols}{3}
\setlength{\parskip}{.05cm}

\texttt{>>>I’ve fought in a dozen little brush wars - and some big
ones - over the years. I’ve seen a lot of shit go down. Once I
got out, though, I wasn’t good for much except killing peo-
ple and breaking things. Upside: those are pretty marketable
skills in 2050. Seems like everybody in the damn country
wants somebody dead or something destroyed.}

\texttt{So I did my time with a few crews. Some pros. Some...not. I
try to maintain a code, though, and after a while I decided
that freelance work was where it’s at. That was a learning
experience. Some of these supposedly shit-hot runners need
to learn a few essentials, like “don’t set up the ambush so you
shoot your own guys” and what “enfilade” means. Makes me
cringe sometimes.}

\texttt{Still, I’ve got a good team, I set my own hours, and I get to
decide whether melting down a busload of nuns is worth the
pay.<<<}

\textbf{The Mercenary} served in one of the many military
forces found in the Sixth World, doing time in conflicts
large and small, and brought from that solid tactical
abilities and a respectable repertoire of combat tal-
ents. Hardened mentally and physically from years in
service, the merc is highly skilled in combat and has
the added benefit of leadership experience that can
save the team’s bacon when things get hairy.



\subsection{CREATING A MERCENARY}

\paragraph{1.  Choose your Metatype}

You may choose \textbf{Human}, \textbf{Dwarf}, \textbf{Elf}, \textbf{Ork}, or
\textbf{Troll}. Each metatype offers a selection of meta-
type moves. Choose one metatype move from
the options presented.

\paragraph{2.  Choose your look}

\textit{Dead eyes, cold eyes, soft eyes}

\textit{Boonie hat, crew cut, ponytail, fauxhawk}

\textit{Combat fatigues, street clothes, nice suit}

\textit{Scarred skin, tough skin, soft skin}

\paragraph{3.  Choose your name and street name}

Make up a name and street name or pick a real
name and street name from the lists and name
generators starting in the \textbf{GM Resources} section.

\paragraph{4.  Assign your stats}

You have 5 stats: Awareness, Combat, Stamina,
Craft, and Presence. Important stats for you are
Combat, Stamina, and Presence.


You have 4 \textbf{Build Points} to distribute among
your stats. To increase a stat by 1 point costs 1
Build Point. You may increase a stat to a maxi-
mum of +2 as a starting character. If you wish,
you may lower 1 stat to -1 in order to have an
additional point to spend.

\paragraph{5.  Choose your Equipment}

Choose from the lists below, or customize your
own gear using the rules in \textbf{Creating Gear} on
page 60.

\textbf{Armor:} \textit{ballistic vest, armor jacket, combat
armor}

\textbf{Weapon (choose 3):} \textit{Ares Predator, Browning
Max Power, AK-97K, Ingram Smartgun, Colt
M22A2, AK-97, tomahawk, combat knife}


\paragraph{6.  Choose your cyberware}

You may start with one of the following cyber-
ware kits (descriptions of these items are on
page 45):

\textbf{Kit 1 (3 essence):} \textit{level 1 wired reflexes, hand
razors}

\textbf{Kit 2 (3 essence):} \textit{cybereyes with low-light/
vision magnification, active camouflage}


\paragraph{7.  Set your Essence and Edge.}

To determine your starting Essence, subtract the
essence cost of your cyberware (if any) from 6.

You start with 3 Edge.

\paragraph{8.  Choose 2 Contacts}

Former CO, Terrorist Cell Member, Arms Dealer,
Veterans Clinic Doctor, Old War Buddy, Street
Pharmacist, Therapist


\paragraph{9.  Establish debts and favors}

Place one of your fellow runners’ names in at
least one of the blanks in the \textbf{Debts \& Favors}
section of your playbook. Each time a name
appears in a debt or favor, it counts as 1 Bond
with that character. The more people you have
Bond with, the better.

\paragraph{10.  Starting Funds}

You start play with 3d6 x 150¥ immediately
available.

\paragraph{11.  Starting Moves}

You know all the Core and Secondary Moves.
You also know the \textbf{Go Tactical} move and
one other Mercenary move.

\end{multicols}

\newpage

\begin{dossier}
\dossierstatbar{THE MERCENARY}
\hspace{.5cm}%
\vrule width 2pt
\hspace{.3cm}%
\begin{dossiermovebar}
\fontsize{9pt}{1em}\selectfont
\setlength{\parskip}{.2cm}

\selectedMove{Go Tactical:} when you Check the Situation during combat, roll+Combat instead of
+Awareness. On a 10+, instead of asking the GM questions, you may instead choose to Hold
3. On a 7-9, you may choose to Hold 1.
\begin{moveoptions}
\moveoption{ You can then spend that Hold 1-for-1
  to grant a bonus to any ally at any point during
  the combat.}
\end{moveoptions}
\unselectedMove{ Combat Instincts:} Boost your first Rock \& Roll move in a fight.

\unselectedMove{ Leatherneck:} when you take damage, roll+Stamina. On 10+, reduce the damage by 3.
On 7-9, reduce the damage by 1.

\unselectedMove{ CQC Expert:} when you Rock \& Roll using a melee weapon or while unarmed,
deal +1d4 damage.

\unselectedMove{ Veteran:} when you Stay Frosty, you take +1.

\unselectedMove{ Adapt and Overcome:} when you fail a move, instead of marking XP you may
roll+Craft. On 10+, take +2 forward to your next move. On 7-9, take +1 forward.

\unselectedMove{ Contracts Available:} you have contacts with a mercenary force or guild. Roll+Presence.
On 10+, they can pass you a contract worth 10,000¥. On 7-9, they can pass you a contract
worth 5,000¥.

\unselectedMove{ Field Trial:} when you use your military connections to acquire military-only equipment,
roll+Presence. On 10+, you’re able to borrow the equipment for 5 days. On 7-9, you borrow
the equipment, but (choose 1):
\begin{moveoptions}
\moveoption{ There’s an unscheduled inventory inspection before you can return it}

\moveoption{ You need to pony up a sizeable “security deposit”}

\moveoption{You got a hangar queen. The equipment
  requires 1 day of maintenance, or it will fail
  at a most inopportune moment.}
\end{moveoptions}
\unselectedMove{ Inspiring:} when you roll a 10+ when you Stay Frosty, one ally who saw you can take +1
to their next move.

\unselectedMove{ Dodge This:} when you manage to get out of an enemy’s line of sight, roll+Craft. On
10+, you get the drop on that enemy when you reappear. On 7-9, you take +1 forward
against that enemy when you reappear.

\end{dossiermovebar}%
\end{dossier}

%%% Local Variables: 
%%% mode: latex
%%% TeX-master: "sixth_world"
%%% End: 

\invisiblepart{DOSSIER : COVERT OPS}

\section{THE COVERT OPS SPECIALIST}
\begin{multicols}{3}
\setlength{\parskip}{.05cm}

\texttt{>>> They call us ``ghosts''. I walk the streets feeling the eyes
  look over me and then instantly forgetting
  me. Unnoticeable. But that's fine. I prefer it
  that way, really. }

\texttt{Smash and grab runs are for amateurs. Not
my style. The people who hire me don't want
flash. They want discretion. Intellegence, theft,
or even occasional wetwork - it doesn't matter to
me. I'm in it for two things: the nuyen and the rush.}

\texttt{When you walk into a building and walk out
again with no one the wiser, that's when you're
truly a ghost. <<<}

\textbf{The Covert Ops Specialist} has perfected
the art of silent entry. Locked doors, security
systems, armed patrols -  these mean nothing to a
skilled Covert Ops Specialist. More than just a
professional thief, these individuals provide
critical support to a team trying to be
subtle. And when its time to take the gloves off,
the Covert Ops Specialist is no stranger to a
surgical assassination.


\subsection{CREATING A COVERT OPS SPECIALIST}

\paragraph{1.  Choose your Metatype}

You may choose \textbf{Human}, \textbf{Dwarf}, \textbf{Elf}, \textbf{Ork}, or
\textbf{Troll}. Each metatype offers a selection of meta-
type moves. Choose one metatype move from
the options presented.

\paragraph{2.  Choose your look}

\textit{Quick eyes, clear eyes, sharp eyes}

\textit{Short hair, ponytail, crew cut, no hair}

\textit{Body suit, street clothes, nice suit}

\textit{Long body, thin body, athletic body}

\paragraph{3.  Choose your name and street name}

Make up a name and street name or pick a real
name and street name from the lists and name
generators starting in the \textbf{GM Resources} section.

\paragraph{4.  Assign your stats}

You have 5 stats: Awareness, Combat, Stamina,
Craft, and Presence. Important stats for you are
Combat, Awareness, and Craft.


You have 4 \textbf{Build Points} to distribute among
your stats. To increase a stat by 1 point costs 1
Build Point. You may increase a stat to a maxi-
mum of +2 as a starting character. If you wish,
you may lower 1 stat to -1 in order to have an
additional point to spend.

\paragraph{5.  Choose your Equipment}

Choose from the lists below, or customize your
own gear using the rules in \textbf{Creating Gear} on
page 60.

\textbf{Armor:} \textit{chameleon suit, form-fitting armor}

\textbf{Weapon (Primary):} \textit{Ares Predator,
   Ranger Arms}

\textbf{Weapon (Silent):} \textit{Taser, Narcoject
Rifle, Crossbow}


\paragraph{6.  Choose your cyberware}

You may start with one of the following cyber-
ware kits (descriptions of these items are on
page 45):

\textbf{Kit 1 (2 essence):} \textit{Cybereyes with
  low-light/camera, Cyberears with Noise Filter/Recorder}

\textbf{Kit 2 (4 essence):} \textit{Wired Reflexes
  1, active camouflage}


\paragraph{7.  Set your Essence and Edge.}

To determine your starting Essence, subtract the
essence cost of your cyberware (if any) from 6.

You start with 3 Edge.

\paragraph{8.  Choose 2 Contacts}




\paragraph{9.  Establish debts and favors}

Place one of your fellow runners’ names in at
least one of the blanks in the \textbf{Debts \& Favors}
section of your playbook. Each time a name
appears in a debt or favor, it counts as 1 Bond
with that character. The more people you have
Bond with, the better.

\paragraph{10.  Starting Funds}

You start play with 3d6 x 150¥ immediately
available.

\paragraph{11.  Starting Moves}

You know all the Core and Secondary Moves.
You also know the \textbf{Ghosting} move and
one other Covert Ops move.

\end{multicols}

\newpage

\begin{dossier}
\dossierstatbar{COVERT OPS SPEC}
\hspace{.5cm}%
\vrule width 2pt
\hspace{.3cm}%
\begin{dossiermovebar}
\fontsize{9pt}{1em}\selectfont
\setlength{\parskip}{.2cm}

\selectedMove{Ghosting:} when you attempt to avoid detection,
roll+Awareness. On a 10+, your presence is
completely undetected. On a 9-7, you escape
detection, but suspicions are raised and you take
a -1 to your future Ghosting attempts. The penalty
will stack until you achieve 10+ at Ghosting or
leave the secure 

\unselectedMove{ Act of God:} when you attempt to
snipe a target from long range, roll+Combat. On a
10+, choose 2. On a 9-7, choose one:
\begin{moveoptions}
\moveoption{You make the shot without exposing
  your position}

\moveoption{You hit the target precisely where you
wanted to}

\moveoption{You don't expend any ammunition}
\end{moveoptions}

\unselectedMove{Cat Burglar:} while climbing, you
are boosted when Staying Frosty.

\unselectedMove{Plan B:} when you fail a Ghosting move, instead of marking XP you may
roll+Combat. On a 10+, take +2 forward towards
your next Rock \& Roll move. On 7-9, take +1 forward.

\unselectedMove{Standard Architecture:} when you
attempt to navigate within a building,
roll+Craft. On 10+, you are able to find your way
without any trouble. On 7-9, you know the way, but
encounter unexpected trouble on route.

\unselectedMove{Shadow Operator} when you achieve a
10+ at Rock \& Roll under poor lighting
conditions, you may opt to deal no damage to your
opponent but instead silently disable them.

\unselectedMove{Security Bypass} when attempting
to bypass a security lock, roll+Craft. On a 10+,
you successfully disable the lock. On a 7-9, you
disable the lock, but choose one:
\begin{moveoptions}
\moveoption{disabling the lock sets off a remote
  alarm}

\moveoption{in your attempts, you damage the
  device visibly}

\moveoption{it takes a long time to disable the lock}

\end{moveoptions}

\unselectedMove{Sight Unseen} when you
successfully infiltrate and then exit a  building
without being detected, gain +1 hold.

\unselectedMove{One of Many} if you hide in a
crowd, take +1 to your Ghosting move.

\unselectedMove{Magic Bullet} when executing an
Act of God, add the following as a possible result
to select from:
\begin{moveoptions}
\moveoption{In addition to your primary target, you strike
  a second, close-by, target}
\end{moveoptions}

\end{dossiermovebar}%
\end{dossier}

%%% Local Variables: 
%%% mode: latex
%%% TeX-master: "sixth_world"
%%% End: 

\invisiblepart{DOSSIER : RIGGER}

\section{THE RIGGER}
\begin{multicols}{3}
\setlength{\parskip}{.05cm}

\texttt{>>>When it comes right down to it, I don’t really live any-
where. Unless you count the driver’s seat. My crew might call
me the “lookout” or the “getaway driver” but when things
have gone bad, I’ve never seen them not be happy that I own
an armored truck with a couple of Vindicators on it.}

\texttt{Seriously, have you seen it? Man, she’s sweet. Purrs like a
kitten, too.}

\texttt{Anyway, with all this Matrix-this and magic-that and
mass-transit-other, you’d think driving wasn’t such a big
thing. Well, that’s a load of bullshit. See, runners don’t take
the fuckin’ subway, choombatta. There ain’t a bus that goes
to the top of Ares Macrotech Tower. You want discreet tac-
tical insertion into a hot LZ? Or a luxury ride in a tricked
out limo? Or how about a good old fashioned \#18 (that one
involves crashing a cement truck through a wall to-- well,
anyway, good times...).}

\texttt{Long story short, you want a ride? You talk to me.<<<}

\textbf{The Rigger} is a cybered-up, shit-hot driving machine.
When a team needs transportation, recon, or a flying
drone to blow the enemy into bloody rags, they turn
to their rigger. Riggers have the capability to oper-
ate any vehicle at its peak, as well as operate drone
vehicles of various kinds. Getting into and out of an
op, and providing a little robotic fire support, is the
rigger’s specialty.



\subsection{CREATING A RIGGER}

\paragraph{1.  Choose your Metatype}

You may choose \textbf{Human}, \textbf{Dwarf}, \textbf{Elf}, \textbf{Ork}, or
\textbf{Troll}. Each metatype offers a selection of meta-
type moves. Choose one metatype move from
the options presented.

\paragraph{2.  Choose your look}

\textit{Goggles, alert eyes, obvious cybereyes}

\textit{Kaiser helmet, cowboy hat, pirate bandana}

\textit{Biker clothes, flight suit, street clothes, punk}

\textit{Heavy body, built body, lean body}

\paragraph{3.  Choose your name and street name}

Make up a name and street name or pick a real
name and street name from the lists and name
generators starting in the \textbf{GM Resources} section.

\paragraph{4.  Assign your stats}

You have 5 stats: Awareness, Combat, Stamina,
Craft, and Presence. Important stats for you are
Awareness, Craft, and Stamina.

You have 4 \textbf{Build Points} to distribute among
your stats. To increase a stat by 1 point costs 1
Build Point. You may increase a stat to a maxi-
mum of +2 as a starting character. If you wish,
you may lower 1 stat to -1 in order to have an
additional point to spend.

\paragraph{5.  Create your Vehicle and Drones}

Pick a mix of vehicles and drones (you may have
up either 2 drones and 1 vehicle or 1 drone and
2 vehicles) from those listed in the \textbf{Vehicles} sec-
tion on page 38, or build them according to
the \textbf{Gear Creation} rules on page 62.

\paragraph{6.  Choose your Equipment}

Choose from the lists below, or customize your
own gear using the rules in \textbf{Creating Gear} on
page 60.

\textbf{Armor:} \textit{ballistic vest, lined coat}

\textbf{Weapon (choose 2):} \textit{Enfield AS-7, Browning
Max Power, Ares Predator, AK-97K, combat
axe}


\paragraph{7.  Choose your cyberware}

You have a \textbf{Control Rig} installed. This allows you
to link to your vehicles and drones. The Control
Rig is always active, and includes a datajack.

You may choose one of the following two kits.
Costs below do not include the cost of the Con-
trol Rig ((descriptions of these items are on page
38):

\textbf{Kit 1 (2 essence):} \textit{cybereyes with flare com-
pensator and low-light, tactical computer}

\textbf{Kit 2 (3 essence):} \textit{cyberears with noise damp-
er and radio, bone lacing}


\paragraph{8.  Set your Essence and Edge.}

To determine your starting Essence, subtract the
essence cost of your cyberware (if any) from 4.

You start with 3 Edge.

\paragraph{9.  Choose 2 Contacts}

Chop shop worker, go ganger, fence, trucker,
arms dealer, mechanic, bartender, cargo pilot,
car thief


\paragraph{10.  Establish debts and favors}

Place one of your fellow runners’ names in at
least one of the blanks in the \textbf{Debts \& Favors}
section of your playbook. Each time a name
appears in a debt or favor, it counts as 1 Bond
with that character. The more people you have
Bond with, the better.

\paragraph{11.  Starting Funds}

You start play with 3d6 x 400¥ immediately
available.

\paragraph{12.  Starting Moves}

You know all the Core and Secondary Moves.
You also know the \textbf{Wheelman} or
\textbf{Drone Rigger} move and one other Rigger move.

\end{multicols}

\newpage



\begin{dossier}
\dossierstatbar{THE RIGGER}
\hspace{.5cm}%
\vrule width 2pt
\hspace{.3cm}%
\begin{dossiermovebar}
\fontsize{9pt}{1em}\selectfont
\setlength{\parskip}{.2cm}


\unselectedMove{ Wheelman:} while jacked into a vehicle you own, when you: 
\begin{moveoptions}
  \moveoption{Stay Frosty, roll+Craft}

  \moveoption{ Check the Situation, add your
    vehicle’s Sensor rating to the roll }

    \moveoption{ Fail a move involving the vehicle, mark off 1 Fuel. }
\end{moveoptions}
    \unselectedMove{ Drone Rigger:} while jacked into a drone, when you: 
\begin{moveoptions}
      \moveoption{Rock \& Roll or Stay Frosty, roll+Craft}

      \moveoption{Check the Situation, roll+the drone’s Sensor rating}

        \moveoption{Fail a move involving the drone, mark off 1 Fuel. }

        \moveoption{ Take an action of your own (not involving the drone), take -2. }
\end{moveoptions}
        \unselectedMove{ Autonomous Mode:} when you put a drone in autonomous mode, indicate which 
          mode setting you want, and roll+Craft. On 10+, hold 2 to be spent on the drone’s moves. 
            On 7-9, hold 1. Drone mode settings (and the rolls they use for moves) are: 
\begin{moveoptions}
            \moveoption{ Sentry: the drone can make the Rock \& Roll move; roll+Tactical }

            \moveoption{ Recon: the drone can make the Check the Situation move; roll+Sensor }

            \moveoption{ Evasion: the drone can make the Stay Frosty move; roll+Power }
\end{moveoptions}
            \unselectedMove{ Split Personality:} when you launch a drone, roll+Awareness. On 10+, you don’t take 
                the normal -2 penalty to non-drone moves while controlling it. On 7-9, the penalty is 
                reduced to -1. 

                \unselectedMove{ Feedback:} when a vehicle or drone you are currently jacked into takes damage, 
                  roll+Stamina. On 10+, the feedback is filtered out completely. On 7-9,  you get a little bit of 
                  a zap: take 1 stun. On a failure, you get a wallop: take 1 wound. 

                  \unselectedMove{ Fly, my pretties!:} You can control two drones at a time instead of one. 

                    \unselectedMove{ Jury Rig:} when you have to make fast repairs to a vehicle or machine, roll+Craft. On 
                      10+, you get it running again and fast. On 7-9, you get it running, but (choose 1): 
\begin{moveoptions}
  \moveoption{ it will only run for 1d10 minutes }

  \moveoption{ afterwards, it will be a total loss. }

  \moveoption{ one of its qualities is reduced by 1, permanently }
\end{moveoptions}
                      \unselectedMove{ Garage:} when you have downtime or legwork time, you can upgrade one of your 
                        vehicles or drones. For every day of time you spend upgrading, you can improve one of the 
                        vehicle’s quality by 1 point or add or change a weapon. You can only upgrade each quality 
                          once. 

                          \unselectedMove{ Percussive Maintenance:} when you smack the hell out of a recalcitrant device, 
                            roll+Craft. On 10+, the device springs to life. On 7-9, the device works for only a moment, 
                            but you know what you need to do to fix it. Take +1 forward to Jury Rig. 

                            \unselectedMove{ Paint the Target:} when you point out a drone or vehicle’s weakness to your team- 
                                mates, they take +1 forward to attacks against it. 


\end{dossiermovebar}%
\end{dossier}

%%% Local Variables: 
%%% mode: latex
%%% TeX-master: "sixth_world"
%%% End: 

\invisiblepart{DOSSIER : SHAMAN}

\section{THE SHAMAN}
\begin{multicols}{3}
\setlength{\parskip}{.05cm}

\texttt{>>>My partner over there likes blasting lightning from his
hands. That’s cool, you know? I mean seriously - it’s cool.
And scary. I’d be jealous, but...I have this other trick. See, in-
stead of channeling power through my hands and poring
over dusty tomes, I just have a quick look-see into the unseen
world around us, locate a friend, and ask ‘em for a hand.}

\texttt{You’re looking at me like you’ve got no idea what I’m talking
about. Lemme break it down for you. All around you, right
now, is the world of astral energy. It’s like our world, but...
not. Okay, not really at all but let’s not get off-topic. Dwelling
there are spirits. Some are called elementals, but what’s nec-
essary to grok is this: I can talk to ‘em, and I can bring them
here, and I can make them do things.}

\texttt{So remember to thank me the next time a being of pure fire
appears and saves your ass from getting geeked.<<<}

\textbf{The Shaman} is a master of conjuring: summoning the
spirits that dwell in the astral realm and compelling
them to do the shaman’s bidding. The shaman’s spir-
its provide many services, from devastating combat
abilities to protection from hostile intent to informa-
tion and reconnaissance impossible for a mundane.



\subsection{CREATING A SHAMAN}

\paragraph{1.  Choose your Metatype}

You may choose \textbf{Human}, \textbf{Dwarf}, \textbf{Elf}, \textbf{Ork}, or
\textbf{Troll}. Each metatype offers a selection of meta-
type moves. Choose one metatype move from
the options presented.

\paragraph{2.  Choose your look}

\textit{Heterochromic eyes, wise eyes, sunglasses}

\textit{Long hair, dreadlocks, shaved head}

\textit{Street clothes, anachronistic clothes, biker gear}

\textit{Wiry body, thin body, round body}

\paragraph{3.  Choose your name and street name}

Make up a name and street name or pick a real
name and street name from the lists and name
generators starting in the \textbf{GM Resources} section.

\paragraph{4.  Assign your stats}

You have 5 stats: Awareness, Combat, Stamina,
Craft, and Presence. Important stats for you are
Craft, and Stamina.

You have 4 \textbf{Build Points} to distribute among
your stats. To increase a stat by 1 point costs 1
Build Point. You may increase a stat to a maxi-
mum of +2 as a starting character. If you wish,
you may lower 1 stat to -1 in order to have an
additional point to spend.

\paragraph{5.  Choose your Totem}

Choose a totem from the list on page 71, or
make up one of your own.

\paragraph{6.  Choose your Equipment}

Choose from the lists below, or customize your
own gear using the rules in \textbf{Creating Gear} on
page 60.

\textbf{Armor:} \textit{Leather jacket, defensive charm, riot
shield}

\textbf{Weapon:} \textit{Ruger Super Warhawk, Colt Man-
hunter, AK-97, combat axe, crossbow}

\paragraph{7.  Bond with your Spirits}

You start the game able to summon 3 spirits. Ei-
ther choose 3 spirits from the examples on page
43, or create these spirits using the rules in the
\textbf{Creating Spirits} section starting on page 69.

\paragraph{8.  Set your Essence and Edge.}

You start with 6 Essence and 3 Edge.

\paragraph{9.  Choose 2 Contacts}

Wage mage, ork underground, gang thug, street
cop, herbalist, university professor, diner owner,
fetishmonger, art dealer, hedge wizard, houngan

\paragraph{10.  Establish debts and favors}

Place one of your fellow runners’ names in at
least one of the blanks in the \textbf{Debts \& Favors}
section of your playbook. Each time a name
appears in a debt or favor, it counts as 1 Bond
with that character. The more people you have
Bond with, the better.

\paragraph{11.  Starting Funds}

You start play with 3d6 x 150¥ immediately
available.

\paragraph{12.  Starting Moves}

You know all the Core and Secondary Moves.

You also know the \textbf{Conjure} and
\textbf{Commune} moves.

\end{multicols}

\newpage

\begin{dossier}
\dossierstatbar{THE SHAMAN}
\hspace{.5cm}%
\vrule width 2pt
\hspace{.3cm}%
\begin{dossiermovebar}
\fontsize{9pt}{1em}\selectfont
\setlength{\parskip}{.1cm}

\selectedMove{ Conjure:} When you summon a spirit, choose the spirit's force and roll. What stat you add
depends on the spirit’s nature:
\begin{moveoptions}
\moveoption{ \textbf{Destroyer:} roll+Combat}

\moveoption{ \textbf{Teacher:} roll+Craft}

\moveoption{ \textbf{Protector:} roll+Stamina}

\moveoption{ \textbf{Seducer:} roll+Presence}

\moveoption{ \textbf{Watcher:} roll+Awareness}
\end{moveoptions}

For every 3 points of force, take -1 to the
roll. The force can not exceed twice your essence.

On 10+, the being is summoned as expected, and may perform a number of Spirit Moves
equal to it's force. On 7-9, the being is summoned, but (choose 1):
\begin{moveoptions}
\moveoption{ it causes drain; take the spirit's force divided by 2 as stun. If the force is greater than your current essence, the damage is physical. }

\moveoption{ you only manage to summon it at half
  the desired force}

\moveoption{ You must expose yourself to danger or an attack to summon the spirit}
\end{moveoptions}
On a failure, the spirit does not manifest, and
you take 1 stun as drain. If you roll a natural 2
(that is, “snake eyes”), the spirit is summoned in an uncontrolled state, and the GM will control
its actions until it is exhausted or banished.

\selectedMove{Commune:} when you mentally commune with your totem, you may gain its boons and
flaws.

\unselectedMove{Banish:} when you attempt to banish a spirit, roll+Stamina. On 10+, you reduce the spirit’s
available moves by 1. On 7-9, you reduce the spirit’s moves by 1, but it deals 1 damage to
you (ignoring armor). If you reduce the spirit’s available moves to 0, it vanishes immediately.

\unselectedMove{ Binding:} when you know a free spirit’s true name and attempt to bind it, roll+Presence.
On a 10+, the true spirit falls under your control and can be called upon later. On a 7-9, the
spirit is controlled, but only for the remainder of the scene.

\unselectedMove{ Favored Spirit:} choose 1 spirit type (Watcher, Teacher, Protector, Destroyer, Seducer).
Take +1 when conjuring spirits of that type.

\unselectedMove{ Aura Mask:} you may conceal your magical nature. Roll+Craft. On 10+, you appear to
be a mundane individual to anyone or anything that examines you. On 7-9, you appear
mundane, but must spend 1 Edge to do so.

\unselectedMove{ Spirit Master:} whenever you
would summon a spirit of force greater than 1, you
may instead conjure multiple spirits, dividing the force among them.

\unselectedMove{ Ally:} choose one of your spirits. This spirit becomes your ally, and when summoned,
always performs one Spirit Move for free for the Shaman. The Spirit also develops a
telepathic link with the Shaman, becoming a new contact. If you ever roll snake eyes while
summoning your ally, it becomes a free spirit.

\unselectedMove{ Great Spirit:} when you conjure a
spirit, if you may take -2 to the roll and the spirit is summoned as a Great Spirit. The Great Spirit is immune
to non-magical attacks, and it has 2 more spirit points increase its Moves for as long as it is
summoned. If you take drain from the summoning,
increase the amount by 1.

\unselectedMove{Spirit Hunter:} when you battle a spirit, you can spend 1 Edge to 
                  ignore the spirit’s armor. 

\end{dossiermovebar}%
\end{dossier}

%%% Local Variables: 
%%% mode: latex
%%% TeX-master: "sixth_world"
%%% End: 

\invisiblepart{DOSSIER : STREET DOCTOR}

\section{THE STREET DOC}
\begin{multicols}{3}
\setlength{\parskip}{.05cm}

\texttt{>>>Medicine, they say, is a calling. You’re in it to help peo-
ple. Well, that’s true, as far as it goes. I liked what I did, un-
til one day I realized I just couldn’t do it anymore. It had
changed, or maybe I did.}

\texttt{But when you’ve spent your time doing it, that’s what you
know. And remember that thing I said about wanting to help
people? Well there’s a whole lot of people who need help,
and they live just below our noses, right where we can’t see.
I set out to help them - street medicine. These days, street
medicine will get you tied up in ugly business sooner or later.
I ended up crossing some people. I needed money. I found
out about shadowrunning. I also found out that plenty of
teams love a good scalpel.}

\texttt{It’s not always fun, combat medicine. In fact, “fun” is not even
in the top 10 words I’d use to describe it. But I figure it’s bet-
ter than leaving someone to see whether blood loss or the
waste management crew gets to them first. So I’m still help-
ing people. They’re not always good people. In fact, they’re
usually career criminals.}

\texttt{Hey, nobody’s perfect.<<<}

\textbf{The Street Doc} brings medical expertise to the shad-
ows, helping their team survive and recover from the
inevitable injuries that they will incur in their particular
line of work. Modern technology might make basic
first aid a matter of a slap patch and a pain pill, but
when you get caught by a frag grenade, basic first aid
is not what you need. You need the Doc.


\subsection{CREATING A STREET DOC}

\paragraph{1.  Choose your Metatype}

You may choose \textbf{Human}, \textbf{Dwarf}, \textbf{Elf}, \textbf{Ork}, or
\textbf{Troll}. Each metatype offers a selection of meta-
type moves. Choose one metatype move from
the options presented.

\paragraph{2.  Choose your look}

\textit{Clear eyes, old eyes, sharp eyes}

\textit{Close cut hair, stylish hairdo, bandana}

\textit{Fit body, heavy body, compact body}

\textit{Business attire, street clothes, EMT jumpsuit}

\paragraph{3.  Choose your name and street name}

Make up a name and street name or pick a real
name and street name from the lists and name
generators starting in the \textbf{GM Resources} section.

\paragraph{4.  Assign your stats}

You have 5 stats: Awareness, Combat, Stamina,
Craft, and Presence. Important stats for you are
Craft and Presence.

You have 4 \textbf{Build Points} to distribute among
your stats. To increase a stat by 1 point costs 1
Build Point. You may increase a stat to a maxi-
mum of +2 as a starting character. If you wish,
you may lower 1 stat to -1 in order to have an
additional point to spend.

\paragraph{5.  Choose your Equipment}

Choose from the lists below, or customize your
own gear using the rules in \textbf{Creating Gear} on
page 60.

\textbf{Armor:} \textit{ballistic vest, armor jacket}

\textbf{Weapon:} \textit{Narcoject rifle, Browning
Max Power, HK227, stun baton, combat knife}

\textbf{Med Kit:} \textit{You have a medkit with 6 Supply.}


\paragraph{6.  Choose your cyberware}

You may start with one of the following cyber-
ware kits (descriptions of these items are on
page 45):

\textbf{Kit 1 (3 essence):} \textit{cyberears with ultrasound
and radio, level 1 skillwires}

\textbf{Kit 2 (3 essence):} \textit{obvious cyberarm with
ReadiMed and shocktrodes}


\paragraph{7.  Set your Essence and Edge.}

To determine your starting Essence, subtract the
essence cost of your cyberware (if any) from 6.

You start with 3 Edge.

\paragraph{8.  Choose 2 Contacts}

ER doctor, morgue staffer, medical examiner,
DocWagon driver, organlegger, black market or-
gan dealer, blood bank worker, pharmacist

\paragraph{9.  Establish debts and favors}

Place one of your fellow runners’ names in at
least one of the blanks in the \textbf{Debts \& Favors}
section of your playbook. Each time a name
appears in a debt or favor, it counts as 1 Bond
with that character. The more people you have
Bond with, the better.

\paragraph{10.  Starting Funds}

You start play with 3d6 x 400¥ immediately
available.

\paragraph{11.  Starting Moves}

You know all the Core and Secondary Moves.
You also know the \textbf{Combat Medic} and
\textbf{Stay With Me} moves.

\end{multicols}

\newpage

\begin{dossier}
\dossierstatbar{THE STREET DOC}
\hspace{.5cm}%
\vrule width 2pt
\hspace{.3cm}%
\begin{dossiermovebar}
\fontsize{9pt}{1em}\selectfont
\setlength{\parskip}{.2cm}


\selectedMove{  Combat Medic:} when you provide medical aid to a person, roll+Craft and mark off 1
Supply from your kit. On 10+, the patient heals 2d4b damage. On 7-9, the patient heals 1d4
damage.

\selectedMove{  Stay With Me:} when you attempt to stabilize a teammate who is bleeding out, roll+Craft
and mark off 2 supply from your kit. On 10+, choose 3. On 7-9, choose 2:
\begin{moveoptions}
\moveoption{they can be moved without a stretcher}

\moveoption{it takes fewer supplies than expected - mark off only 1 supply}

\moveoption{you do not expose yourself to danger to help them.}

\moveoption{they will not have a chronic injury}
\end{moveoptions}
Your patient does not die if you fail this move, and you may take -1 and try again. A second
failure, however, results in the death of the patient.

\unselectedMove{ Grace Under Fire:} when you are working on a patient during a fight but not actively
fighting, you have +1 armor.

\unselectedMove{ First Do No Harm:} if you refuse to do harm (you never deal lethal damage), then your
Grace Under Fire move grants +2 armor instead.

\unselectedMove{ We All Bleed Red:} when you take time to treat an injured enemy, mark off 1 supply
and roll+Presence. On 10+, they’re stable, and you can ask two questions which they will
answer truthfully. On 7-9, you can ask only one question.

\unselectedMove{ Good Drugs:} when a patient contracts a disease or is poisoned, roll+Awareness. On
10+, you have the correct antidote or antitoxin on hand, and can halt the progress of the
disease or poison. Mark off 1 supply from your medkit. On 7-9, you are only able to slow
the effects. Mark off 1 supply from your medkit.

\unselectedMove{ Pharmacy Is Open:} when you use a contact to obtain medical supplies (amounting to
+1 supply), and roll+Presence. On 10+, choose 2. On 7-9, choose 1:
\begin{moveoptions}
\moveoption{you get +2 supply instead of +1}

\moveoption{it takes 1 day to get the supplies instead of 2}

\moveoption{nobody notices the supplies are missing}

\moveoption{you receive an interesting piece of information as well}
\end{moveoptions}
\unselectedMove{ Mobile Surgery:} you own a vehicle that contains a small but complete surgical suite,
capable of treating serious injuries. It has a base supply value of 10. Supplies from the mo-
bile surgery can be used to replenish your Med Kit.

\unselectedMove{ You Got This:} whenever you walk someone through a medical procedure (such as first
aid), roll+Presence. On 10+, they are boosted. On 7-9, they take +1.

\unselectedMove{ Megalexicosis:} when you spout a stream of medical technobabble to confuse, intimi-
date, convince, or distract someone, you may roll+Craft instead of +Presence.



\end{dossiermovebar}%
\end{dossier}

%%% Local Variables: 
%%% mode: latex
%%% TeX-master: "sixth_world"
%%% End: 

\invisiblepart{DOSSIER : STREET SAMURAI}

\section{THE STREET SAMURAI}
\begin{multicols}{3}
\setlength{\parskip}{.05cm}

\texttt{>>>I’m not close to a lot of people. It might be my blank
silver cybereyes, or the dermal plating under my skin...or
maybe just the fact that whenever I look at someone, they
assume I have some sort of crosshair hovering over them.}

\texttt{They’re right about the crosshairs.}

\texttt{Anyway, I don’t have a lot of friends. But when the lead starts
flying, all that changes.}

\texttt{I’m chipped and wired, choombatta. I’m harder than steel,
faster than lightning, hit like an avalanche, and shoot like I
invented it. It cost me, of course. Injuries. Pain. Shitloads of
money.}

\texttt{Was it worth it? Replacing my meat with machines? The pain
of recovery, the terrible itch as it integrated, the gradual dis-
tancing of people I loved. Was it worth it, to be this good?}

\texttt{Hell yes.<<<}

\textbf{The Street Samurai} is a combat master. Often one of
toughest and most skilled combatants on the team,
the street samurai is a warrior for hire whose super-
human talents were bought with cybernetic upgrades,
relentless training, and no small amount of spilled
blood. The Street Samurai may be a hired gun, but
they take the word “samurai” very seriously, and ad-
here to a code of their own devising. On the mean
streets of the Sixth World, ther samurai is a feared—
and respected—enemy.

\subsection{CREATING A STREET SAMURAI}

\paragraph{1.  Choose your Metatype}

You may choose \textbf{Human}, \textbf{Dwarf}, \textbf{Elf}, \textbf{Ork}, or
\textbf{Troll}. Each metatype offers a selection of meta-
type moves. Choose one metatype move from
the options presented.

\paragraph{2.  Choose your look}

\textit{Glowing eyes, silvered eyes, hard eyes}

\textit{Cropped hair, wild hair, topknot}

\textit{Tattooed skin, scarred skin, camo skin}

\textit{Bulky body, lithe body, skinny body}

\paragraph{3.  Choose your name and street name}

Make up a name and street name or pick a real
name and street name from the lists and name
generators starting in the \textbf{GM Resources} section.

\paragraph{4.  Assign your stats}

You have 5 stats: Awareness, Combat, Stamina,
Craft, and Presence. Important stats for you are
Combat, Stamina, and Craft.


You have 4 \textbf{Build Points} to distribute among
your stats. To increase a stat by 1 point costs 1
Build Point. You may increase a stat to a maxi-
mum of +2 as a starting character. If you wish,
you may lower 1 stat to -1 in order to have an
additional point to spend.

\paragraph{5.  Choose your Equipment}

Choose from the lists below, or customize your
own gear using the rules in \textbf{Creating Gear} on
page 60.

\textbf{Armor:} \textit{form-fitting armor, ballistic vest, lined
coat}

\textbf{Weapon:} \textit{choose four weapons from the list
of melee and small arms}


\paragraph{6.  Choose your cyberware}

You start with the following cyberware items:


\begin{adjustwidth*}{.5cm}{.5cm}
\tcirc{} Cybereyes with low-light, thermographic,
and flare compensation capability

\tcirc{} Wired Reflexes 1 \textbf{or} Dermal Plating (+1)
\end{adjustwidth*}

These items are state of the art, fully integrated
with your biology and do not cost essence. In
addition, choose one kit from the options below
(descriptions of these items are on page 38):


\textbf{Kit 1 (5essence):} \textit{Bone lacing, skillwires,
cranial cushion}

\textbf{Kit 2 (3 essence):} \textit{cyberarm with light pistol
and smartlink}

\textbf{Kit 3 (4 essence):} \textit{Smartlink, noise filter, tacti-
cal computer, hand razors}

You can customize items using the rules in the
\textbf{Creating Cyberware} section on page 63.

\paragraph{7.  Set your Essence and Edge.}

To determine your starting Essence, subtract the
essence cost of your cyberware (if any) from 6.

You start with 4 Edge.

\paragraph{8.  Choose 2 Contacts}

Arms dealer, cybersurgeon, bartender, street
clinic nurse, private investigator, dockworker, pi-
lot, cab driver, retired runner, survival nut

\paragraph{9.  Create your Code}

The word “samurai” means something on these
streets. Create (with the help of the GM and the
other players, if you like) the code of honor that
you follow.


\paragraph{10.  Establish debts and favors}

Place one of your fellow runners’ names in at
least one of the blanks in the \textbf{Debts \& Favors}
section of your playbook. Each time a name
appears in a debt or favor, it counts as 1 Bond
with that character. The more people you have
Bond with, the better.

\paragraph{11.  Starting Funds}

You start play with 3d6 x 250¥ immediately
available.

\paragraph{12.  Starting Moves}

You know all the Core and Secondary Moves.
You also know the \textbf{Weapons Free} move and
one other Street Samurai move.

\end{multicols}

\newpage



\begin{dossier}
\dossierstatbar{THE STREET SAM}
\hspace{.5cm}%
\vrule width 2pt
\hspace{.3cm}%
\begin{dossiermovebar}
\fontsize{9pt}{1em}\selectfont
\setlength{\parskip}{.2cm}
\selectedMove{Weapons Free:} when you fire up your augmentations, roll+Combat. On 10+, you may
spend 1 Edge and activate all of your cyberware. On 7-9, it costs 2 Edge.

\unselectedMove{More Power:} when you attempt to bend, break through, or otherwise destroy some-
thing, roll+Stamina. On 10+, you easily achieve your goal. On 7-9, you break it, but (choose
1):
\begin{moveoptions}
\moveoption{ It takes longer than expected}

\moveoption{ It makes a lot of noise}

\moveoption{ You take 1 stun in the process}
\end{moveoptions}

\unselectedMove{Get Medieval:} when you deal damage to an enemy in melee, take +2 forward against
that enemy.

\unselectedMove{Shake it Off:} when you fight through the pain, roll+Stamina. On 10+, remove 2 boxes
of stun damage. On 7-9, remove 1.

\unselectedMove{Situational Awareness:} you are never surprised. If an enemy would get the drop on
you, you may act first.

\unselectedMove{State of the Art:} select one cyberware item that normally requires activation. That item
gains the always on tag.

\unselectedMove{The Only Thing Faster is Light:} whenever you Rock \& Roll, on a 12+ you may deal
your damage to a second target within range.

\unselectedMove{Pain Editor:} when you have to make a Gut Check, you are boosted. Additionally, when
you reach 9 or more wounds, you may choose to accept a chronic injury rather than bleed-
ing out. If you already have all of the chronic injuries, you cannot use this move.

\unselectedMove{Honorable:} when you uphold a tenet of your code, roll+Presence. On a 10+, hold
2. On 7-9, hold 1. You may spend this hold to pull strings, manipulate, or make someone
sweat.

\unselectedMove{Deadeye:} when you attack a surprised or defenseless enemy in ranged combat, you
can deal damage or, name your target and roll+Combat:
\begin{moveoptions}
\moveoption{ Head: on 10+, you deal your damage and they fall to the ground, stunned. 7-9: they fall
to the ground, stunned.}

\moveoption{ Arms: on 10+, you deal your damage, and they drop whatever they’re holding. 7-9: they
drop whatever they’re holding.}

\moveoption{ Legs: on 10+, you deal your damage, and they are slowed or immobilized. 7-9: they are
slowed or immobilized.}
\end{moveoptions}
\end{dossiermovebar}%
\end{dossier}

%%% Local Variables: 
%%% mode: latex
%%% TeX-master: "sixth_world"
%%% End: 


\invisiblepart{DOSSIER : GEAR ADDENDUM}
\begin{dossier}

\begin{minipage}[b][\textheight][t]{8cm}
\begin{dossierbox}{8cm}{15cm}{GEAR}
\begin{adjustwidth*}{0cm}{.2cm}
\vspace{-.1cm}
\begin{tabu}{p{2cm}|p{5.5cm}}
\rowfont{\oswaldfont\fontsize{16pt}{0em}\selectfont}
Item & Effect / Notes\\
\hline\\[.20cm]
\hline\\[.20cm]
\hline\\[.20cm]
\hline\\[.20cm]
\hline\\[.20cm]
\hline\\[.20cm]
\hline\\[.20cm]
\hline\\[.20cm]
\hline\\[.20cm]
\hline\\[.20cm]
\hline\\[.20cm]
\hline\\[.20cm]
\hline\\[.20cm]
\hline\\[.20cm]
\hline\\[.20cm]
\hline\\[.20cm]
\hline\\[.20cm]
\hline\\[.20cm]
\hline\\[.20cm]
\hline\\[.20cm]
\hline
\end{tabu}
\end{adjustwidth*}
\end{dossierbox}
\end{minipage}
\hspace{.65cm}%
\vrule width 2pt
\hspace{.3cm}%
\begin{minipage}[b][\textheight][t]{17cm}
\begin{dossierbox}{8cm}{12.2cm}{CYBERWARE}
\begin{adjustwidth*}{0cm}{.2cm}
\vspace{-.1cm}
\begin{tabu}{p{2cm}|p{5.5cm}}
\rowfont{\oswaldfont\fontsize{16pt}{0em}\selectfont}
Implant & Effect / Notes\\
\hline\\[.20cm]
\hline\\[.20cm]
\hline\\[.20cm]
\hline\\[.20cm]
\hline\\[.20cm]
\hline\\[.20cm]
\hline\\[.20cm]
\hline\\[.20cm]
\hline\\[.20cm]
\hline\\[.20cm]
\hline\\[.20cm]
\hline\\[.20cm]
\hline\\[.20cm]
\end{tabu}
\end{adjustwidth*}
\end{dossierbox}%
\begin{dossierbox}{8cm}{12.2cm}{PROGRAMS}
\begin{adjustwidth*}{0cm}{.2cm}
\vspace{-.1cm}
\begin{tabu}{p{2cm}|p{5.5cm}}
\rowfont{\oswaldfont\fontsize{16pt}{0em}\selectfont}
Program & Routines\\
\hline\\[.20cm]
\hline\\[.20cm]
\hline\\[.20cm]
\hline\\[.20cm]
\hline\\[.20cm]
\hline\\[.20cm]
\hline\\[.20cm]
\hline\\[.20cm]
\hline\\[.20cm]
\hline\\[.20cm]
\hline\\[.20cm]
\hline\\[.20cm]
\hline\\[.20cm]
\end{tabu}
\end{adjustwidth*}
\end{dossierbox}%
\begin{dossierbox}{17cm}{3cm}{WEAPONS}
\begin{adjustwidth*}{0cm}{.2cm}
\vspace{-.1cm}
\begin{tabu}{p{5cm}p{1.5cm}p{1.5cm}p{5cm}p{3cm}}
\rowfont{\oswaldfont\fontsize{16pt}{0em}\selectfont} Weapon & Range & Damage & Ammo & Tags\\
\end{tabu}

\vspace{.2cm}
\setlength{\baselineskip}{.72cm}
\underline{\hspace{17.3cm}}

\underline{\hspace{17.3cm}}

\underline{\hspace{17.3cm}}

\underline{\hspace{17.3cm}}

\underline{\hspace{17.3cm}}

\underline{\hspace{17.3cm}}
\end{adjustwidth*}
\end{dossierbox}
\end{minipage}

\end{dossier}

%%% Local Variables: 
%%% mode: latex
%%% TeX-master: "sixth_world"
%%% End: 

\invisiblepart{DOSSIER : SPELLS \& SPIRITS ADDENDUM}
\begin{dossier}

\begin{minipage}[b][9.7cm][t]{25.5cm}
\begin{dossierbox}{25.5cm}{15cm}{SPELLS}
\begin{adjustwidth*}{0cm}{.2cm}
\vspace{-.1cm}
\begin{tabu}{p{5cm}|p{9cm}|p{2cm}|p{8cm}}
\rowfont{\oswaldfont\fontsize{16pt}{0em}\selectfont}
Spell & Description & Min. Force & Tags\\
\hline& & & \\[.20cm]
\hline& & & \\[.20cm]
\hline& & & \\[.20cm]
\hline& & & \\[.20cm]
\hline& & & \\[.20cm]
\hline& & & \\[.20cm]
\hline& & & \\[.20cm]
\hline& & & \\[.20cm]
\hline& & & \\[.20cm]
\hline& & & \\[.20cm]
\hline
\end{tabu}
\end{adjustwidth*}
\end{dossierbox}
\end{minipage}
\begin{minipage}[b][][t]{25.5cm}
\begin{dossierbox}{25.5cm}{15cm}{SPIRITS}
\begin{adjustwidth*}{0cm}{.2cm}
\vspace{-.1cm}
\begin{tabu}{p{5cm}|p{9cm}|p{10cm}}
\rowfont{\oswaldfont\fontsize{16pt}{0em}\selectfont}
Spirit & Description & Tags\\
\hline& & \\[.20cm]
\hline& & \\[.20cm]
\hline& & \\[.20cm]
\hline& & \\[.20cm]
\hline& & \\[.20cm]
\hline& & \\[.20cm]
\hline& & \\[.20cm]
\hline& & \\[.20cm]
\hline
\end{tabu}
\end{adjustwidth*}
\end{dossierbox}
\end{minipage}


\end{dossier}

%%% Local Variables: 
%%% mode: latex
%%% TeX-master: "sixth_world"
%%% End: 

\invisiblepart{DOSSIER : JOURNAL}
\begin{dossier}

\begin{minipage}[b][9.3cm][t]{25.5cm}
\begin{dossierbox}{25.5cm}{15cm}{NOTES}
\begin{adjustwidth*}{0cm}{.2cm}
\vspace{-.1cm}
\begin{tabu}{p{25.4cm}}
\\[.20cm]
\hline\\[.20cm]
\hline\\[.20cm]
\hline\\[.20cm]
\hline\\[.20cm]
\hline\\[.20cm]
\hline\\[.20cm]
\hline\\[.20cm]
\hline\\[.20cm]
\hline\\[.20cm]
\hline\\[.20cm]
\end{tabu}
\end{adjustwidth*}
\end{dossierbox}
\end{minipage}
\begin{minipage}[b][][t]{25.5cm}
\begin{dossierbox}{25.5cm}{15cm}{DEBTS \& FAVORS}
\begin{adjustwidth*}{0cm}{.2cm}
\vspace{-.1cm}
\begin{tabu}{p{25.4cm}}
\\[.20cm]
\hline\\[.20cm]
\hline\\[.20cm]
\hline\\[.20cm]
\hline\\[.20cm]
\hline\\[.20cm]
\hline\\[.20cm]
\hline\\[.20cm]
\hline
\end{tabu}
\end{adjustwidth*}
\end{dossierbox}
\end{minipage}


\end{dossier}

%%% Local Variables: 
%%% mode: latex
%%% TeX-master: "sixth_world"
%%% End: 

\invisiblepart{DOSSIER : VEHICLES \& DRONES}
\begin{dossier}

\begin{minipage}[b][\textheight][t]{.3\linewidth}
\dossiervehiclebox
\\
\vspace{2cm}
\\
\dossiervehiclebox
\end{minipage}%
\hspace{1.5cm}%
\begin{minipage}[b][\textheight][t]{0.3\linewidth}
\dossiervehiclebox
\\
\vspace{2cm}
\\
\dossiervehiclebox
\end{minipage}%
\hspace{1.5cm}%
\begin{minipage}[b][\textheight][t]{0.3\linewidth}
\dossiervehiclebox
\\
\vspace{2cm}
\\
\dossiervehiclebox
\end{minipage}




\end{dossier}

%%% Local Variables: 
%%% mode: latex
%%% TeX-master: "sixth_world"
%%% End: 


\newpage
\switchToLayoutPageA{}


\invisiblepart{COMBAT}
\section{COMBAT}
\begin{multicols}{2}

Shadowrunners tend to get themselves into lots of trouble,
the kind that ends with some high-intensity interpersonal conflict resolution. In other words, combat. As you’ll find
when you read through the rest of this document, most of
combat (in fact, pretty much everything the player characters
do, ever) is handled through the application of various moves
as they intersect with the fiction. This section explains a few
specific quirks of combat in \SW{}.

Remember: although you’re reading a section titled “Combat,” there’ no point at which the game switches to “combat
rounds,” and nobody rolls initiative. In other words, there’s
no true division between combat and everything else that
happens in \SW{}. Since everything flows from the
game fiction and returns to the game fiction, combat is just
another part of the regular flow of the game.



\subsubsection{ARMOR}
Because a shadowrunner leads a dangerous life, a big premium is put on not getting hit or at least not taking all the
damage. The obvious way to do so is to wear armor. In \SW{}, armor reduces incoming damage on a 1 for 1 basis.
The tradeoff, of course, is that you can’t spend all day walking
around in combat armor—it’s hot, itchy, intimidating, and
cops tend to notice.

Some metatypes and archetypes offer moves that let you reduce damage, or otherwise avoid some of the less pleasant
outcomes of damage. For example, the \textit{‘Ard Bastard} move
(an ork metatype move) lets the character take +1 to gut
checks, and the troll move \textit{You’ll Just Make It Angry} grants an
additional wound box..


\subsubsection{SURPRISE}
The \textit{Rock \& Roll} move and most other damage-dealing moves
assume that your target can fight back. If that’s not a possibility (that is, if your target is surprised, helpless, etc.), the fiction
can’t trigger the \textit{Rock \& Roll} move. You just put a round in
their head and move on.

When you get the drop on someone in combat, you don’t
need to use a move to deal damage to them—you can simply deal your damage (or kill them outright, depending on
the situation). Likewise, if someone gets the drop on you in
combat, expect to eat some lead.


\subsubsection{FIRE MODES}
Weapons in the game can fire in semi-automatic, burst, or
full-auto modes, depending on their specific capabilities.
Semi-auto is the “default” assumption; in that mode you only
use up ammunition when you roll 7-9 on the \textit{Rock \& Roll}
move, and choose to burn extra ammo.

Firing in \textbf{burst} or \textbf{auto} modes when using \textit{Rock \& Roll} allows
you to add +1 damage to your attack; however, it always
uses 1 ammo (even if you roll 10+).

Finally, full-auto mode is very useful for suppression fire, and
lets you take +1 when you use the Suppression Fire move.


\subsubsection{RELOADING}
Most of the weapons indicate some ammo capacity using the
ammo tag - this indicates how much ammunition a weapon
can carry in its magazine or clip before it must be reloaded.
If a weapon has 3 ammo, for instance, you have ammunition in the gun until you have marked off all three ammo.
Ammo is an abstraction - 1 ammo does not represent a single
round, but simply “some ammunition.” The game assumes
(for the most part) that a character fires multiple shots in a
single move.

During combat, assume that combatants are reloading their
weapons when appropriate, keeping them topped up. Mechanically, this is handled by the fact that \textit{Rock \& Roll} doesn’t
cost ammo unless you roll a 7-9, and choose to burn up extra
ammo (or if you use burst or full-auto weapons).

When you mark off all your ammo, you’ll need to reload.
There is no specific move to reload a weapon. If taking the
time to reload would not expose you to danger, then you
can reload simply by saying so. On the other hand, if you’re
reloading despite an imminent risk, that’s a job for the Stay
Frosty move.


\subsubsection{LIGHT AND SOUND}
You’ll note in the Metahuman Moves section that some metahumans have the ability to see either in low-light, or see into
the infrared (and you’ll also note in the Cyberware section that
cyberware can grant similar abilities). At the GM’s discretion,
he or she may establish that the area the characters are in has
low visibility due to one of the following factors, and impose
modifiers on players’ rolls. There are four
visibility options:
\begin{adjustwidth*}{.5cm}{.5cm}

\textbf{Darkness:} both low-light and thermographic vision allow
normal vision in dark environments. Characters with normal vision must use a light or take -1 ongoing as long as
it remains dark. \textbf{Note:} low-light vision is ineffective in truly
complete darkness, and no vision type is effective in supernatural darkness.

\textbf{Smoke/Fog:} characters with normal or low-light vision
take -1 ongoing while the smoke or fog persists; characters with thermographic vision suffer no vision
difficulties.

\textbf{Glare/Flash:} in circumstances of very bright light, all characters without some sort of compensation (sunglasses, or
flare compensators for things like flash-bang grenades)
take -1 ongoing until they recover or compensate from the
bright light.
\end{adjustwidth*}

As with vision, it’s important to be able to hear in combat.
In a very noisy environment (a factory, an active airstrip, etc.)
or in the event of intensely sharp or loud noises (flash-bangs,
explosions, even sustained gunfire), the GM may impose -1
forward or -2 forward penalties. Certain cyberware (such as
frequency filters or dampers) or protective equipment like
earplugs can eliminate these penalties.


\end{multicols}

\invisiblepart{DAMAGE AND HEALING}
\section{DAMAGE AND HEALING}
\begin{multicols}{2}

Inevitably, when you play with guns, magic, and sensitive secrets, somebody is going to get shot. Or burned, or hit with a
brick, or drenched in elemental acid summoned from beyond
the realm of mortal ken, or thrown out a window, or...well,
you get the point.

In any case, damage will be given and taken, and quite possibly end with someone being little more than yesterday’s
garbage.

\subsection{DEALING DAMAGE}

When you make a move that has the potential to deal damage, the move will usually say, as a possible result, “deal
your damage” or “you deal damage.” Damage in the game
is usually variable, based on the damage dice for the weapon
being used (see the \textbf{Equipment} section for information on
weapons). This is the amount of damage that is applied to
your target.

\begin{adjustwidth*}{.5cm}{.5cm}

\textbf{Example:} \textit{Johnny Chopz hits a ghoul with his trusty katana. The katana deals 2d6b damage (meaning roll 2d6, and
take the best result). Johnny’s player rolls 2d6, getting 3,
5. Thus, the attack deals 5 damage to the ghoul. Bad news,
creep.}
\end{adjustwidth*}

If a move indicates that you deal half damage, roll the damage as normal, and then divide the result in half (rounding up) 
to get your final damage amount. 
The most common situation in which you’ll deal half damage 
  is if you’re shooting at a vehicle with small arms. Vehicles 
    take only half damage (before armor) from small arms, and no 
    damage from melee weaponry. 

\begin{adjustwidth*}{.5cm}{.5cm}

\textbf{Example:} \textit{Johnny is being chased down by a go-ganger,
and turns to shoot at the onrushing psycho with his Ares
Predator. When he rocks \& rolls with the ganger, he’s able
to deal his damage (1d8+1) and wants to hit the vehicle,
not the ganger. He rolls 5 damage. Halving that yields 3
damage (5 ÷ 2, rounded up) means that a bullet just gets
through the armor, but it ain’t gonna help. If he’d pulled
out his katana and stood his ground...well, what would
happen is that he’d end up with a motorcycle wheel up
his nose.}
\end{adjustwidth*}

\subsection{GETTING HURT}
When a character takes damage in the game, it is recorded by
marking \textbf{wound boxes} the character’s playbook. Most weapons in the game deal physical damage; when taking damage
from this kind of weapon, mark off a number of boxes on
the Wound Track equal to the damage taken. Getting dealt 3
damage, for instance, would mean that (all else things being
equal) the player would mark 3 Wounds on their
playbook.

If a weapon specifies that it deals stun damage, you still check
off boxes on the Wound track. However, if a weapon dealing
stun damage is the one that takes you out, you are knocked
unconscious. All characters have a maximum of 8 wounds/
they can take. Once they reach 8, the next wound will put
them on the ground, thoroughly incapacitated (whether
unconscious, or worse). To differentiate between stun and
wound damage, put a single diagonal line in the box for stun,
and an X for wounds.

\subsubsection{WOUNDS TRUMP STUN}

If you have marked off stun damage on your damage track,
and you subsequently take an actual wound, the wounds
“push” the stun toward the right-hand end of the track—to
indicate this, you can add a second line in the already-stunned
boxes to make an X, and then mark off additional boxes of
stun to the right.
\begin{adjustwidth*}{.5cm}{.5cm}

\textbf{Example:} \textit{Uncle Slam just got nailed by a stun baton, and
took 2 stun damage. His wound track looks like
this:}

\textit{As the fight develops, his opponent pulls out a knife, and
manages to slip the point through a gap in Uncle Slam’s
apparently-not-so-patent armor. Uncle Slam takes 1 wound
from the attack. His wound track now looks like this:}
\end{adjustwidth*}

If taking a wound pushes your stun damage into the 8th box,
you will have to make a gut check.
\subsubsection{EXTRA WOUND BOXES}
Some moves (such as the \textit{You’ll Just Make It Angry} move)
 or equipment (like Bone Lacing) grant an additional wound
box. In the archetype dossiers starting on page 8, these
additional boxes are shown with dotted lines. If you do have
an extra box, just darken the lines so you know where to start
filling in wounds. No matter what equipment or moves you
have, you can never have more than 10 wound boxes.

\subsubsection{GUT CHECKS}

When a character takes damage in the game, it is assumed
that, until the last couple boxes, while they may ultimately
prove to be significant injuries, they’re minor enough to ignore for the moment. There are two exceptions:

\begin{adjustwidth*}{.5cm}{.5cm}
\textbf{Wound \#8:} when you check off that last box of your
Wound track, you must make the Gut Check move.

\textbf{Major Trauma:} if you take 6 or more damage (after applying armor) in a single hit, you have just taken Major Trauma. You will need to make the Gut Check move.
\end{adjustwidth*}

\subsubsection{BLEEDING OUT}
Once a character takes their 9th wound (that is, takes any
damage after reaching their 8th wound box), they are \textbf{Bleeding Out.} This basically means they’re incapacitated, unable to
perform any sort of action, and badly hurt (it doesn’t actually
mean there’s blood everywhere; “bleeding out” just sounds
cool).

A character who is Bleeding Out must be stabilized, either
via the \textbf{First Aid} move, or via equipment such as the Trauma
Patch.

\subsubsection{CHRONIC INJURY}
If a character reaches the Bleeding Out stage, and survives
their precarious situation, they will be left with a \textbf{Chronic Injury.} This is a long-term (and possibly permanent) reminder
of their brush with death. Chronic Injuries reduce the affected
Stat by 1 point. When your character receives a
chronic injury, choose one of the following and
check off the small box on their dossier:
\begin{adjustwidth*}{.5cm}{.5cm}

\textbf{Shaky (-1 Combat):} your injury interferes with your ability
to fight. Perhaps your hands are unsteady, or maybe your
mind is too traumatized to focus, or you even lost a limb
or part of a limb.

\textbf{Fragile (-1 Stamina):} your injury weakens your body, making you less able to endure the hard life of
shadowrunning.

\textbf{Sluggish (-1 Awareness):} you suffered an
injury that hampers your ability to perceive and
react to the world -- perhaps you were partially blinded or deafened, or you have
nerve damage that prevents you from reacting as quickly
as you once did.

\textbf{Dazed (-1 Craft):} your injury dulls your mind, making it
harder to recollect facts and focus on
intellectual matters.

\textbf{Disfigured (-1 Presence):} your injury left you with nasty
scars that are immediately obvious and shocking to the
people you interact with.

\textbf{Dulled (-1 Edge):} whatever happened to you out there,
you’re not as sharp as you used to be. Maybe it’s just some
glitched out cyberware, or maybe you lost a little bit of
what it takes to do this job.

\textbf{Faded (-1 Essence):} whether it fed the unnatural thirst of
some paranormal creature, fueled a dark ritual, or just got
hacked away by someone meaner and faster than you, you
lost a piece of yourself.
\end{adjustwidth*}

You can’t have the same chronic injury twice. If you are already Faded, and you take a second chronic injury, you’ll have
to choose something else. However, if you heal a chronic injury and recover the lost stat point, you could elect to take it
again in the future.


\subsection{GETTING BETTER}
\subsubsection{HEALING STUN}
Stun damage is fairly simple to heal. at the end of an encounter, scene, or situation (in other words, once the character
has a chance to take a breather), their stun damage is healed.

\subsubsection{HEALING WOUNDS}
Generally, as long as a character has not received more than
8 wounds, and has not failed a gut check, they are not incapacitated by injury (though they may be feeling very much
the worse for wear). Recovery from this level of injury is really
a matter of time, and perhaps a small amount of attention
from their, ah...let’s say, primary care provider. Mechanically,
injury of this nature will be healed during downtime, assuming that they get approximately two days of rest and basic
medical care for each wound box they have.

\begin{adjustwidth*}{.5cm}{.5cm}

\textbf{Example:} \textit{Navy got hurt on her last run, but she was on
her feet and processing oxygen at the end of it, so she
considered it a job well done. She finished the run with 4
wound boxes checked. This means that she will need to
have roughly 8 days of rest and medical care to heal those
injuries, at which point, she’s good as new.}
\end{adjustwidth*}
\subsubsection{HEALING CHRONIC INJURIES}
Chronic Injuries are not necessarily permanent injuries, unless
the player wishes them to be. However, they can only be
healed or ameliorated by major or long-term treatment. A
chronic physical injury may be fixed via cybernetic replacement, for instance, which is a major surgical intervention.
Chronic psychological injury may require therapy over a long
term as well.

It is up to the GM and players to negotiate the specific plan
for removal of a Chronic Injury. It may be that recovery may
evolve into a shadowrun of its own, but that is not required:
spending funds to pay for therapy, new cyberware, surgery,
or the like is sufficient if you want to keep the story of the
recovery as background events.

\subsection{GETTING BURIED}
With the rules covering stabilization, chronic injury, armor,
and so forth it’s actually fairly hard to
all-the-way die in \SW{}. However, it can happen
in a few different ways.
\begin{adjustwidth*}{.5cm}{.5cm}

\textbf{Failed to Stabilize:} if the person attempting to provide
First Aid to Bleeding Out character fails their move, the
wounded character cannot be stabilized, and dies at the
end of the encounter.

\textbf{Continued Damage:} if a character takes 6 damage beyond
that 8th wound box (armor still counts!) they’re too badly mangled to be saved. Players, understand that this can
happen; GM’s, be really careful with this one.

\textbf{Overwhelming Kaboom:} if a character is hit with an attack of such overwhelming power that surviving it strains
all credulity, they’re killed immediately. For example, if a
character is, say, hit by an antiship missile, or falls into a
crucible of molten iron...just forget it, they’re
gone.
\end{adjustwidth*}
\subsubsection{THE LAST RUN}
The faded chronic injury can potentially reduce a character to zero essence. Mechanically, dropping to zero
does not have an immediate effect: instead, you have
until your next advance to regain that lost point of essence (so you have at least 1 essence).

If you elect not to regain a point of essence, you’re signaling to the GM that it’s time for your character’s\textbf{ Last
Run.} The nature of this final adventure is up to you and
the GM, but it will be your final walk in the shadows.

\end{multicols}

\invisiblepart{MAGIC}
\section{MAGIC}
\begin{multicols}{2}
In the \SW{}, the magic has returned to the world,
and dormant powers have reawakened. Magic is fueled by
Essence, one of the variable point pools each character has.

\subsection{ESSENCE}
Three archetypes in the game - the \textbf{Adept}, the \textbf{Mage}, and
the \textbf{Shaman} - are magically gifted, or \textbf{Awakened}, which means that they
depend on their Essence to use their
magical abilities.
\begin{adjustwidth*}{.5cm}{.5cm}
\textbf{The Adept:} adepts turn their magical ability inward to improve themselves, sometimes to superhuman levels. An
adept spends edge to temporarily modify his or her
capabilities (for example the Enhanced Ability or Killing
Hands moves). An adept's essence infuses their body with mystical protection, but also limits their maximum potential.

\textbf{The Mage:} when a mage casts a spell, the mage's essence serves as a limit to how powerful the spell can be. The more powerful the spell, the more of a chance something will go wrong - potentially injuring the mage in the process.

\textbf{The Shaman:} when a shaman summons
a spirit or elemental, the shaman's essence serves as a limit to the how powerful the spirit can be. More powerful spirits can perform more tasks for their summoner. However, dealing with powerful spirits is risky - the summoning could injure the shaman or even turn an angry spirit loss upon them.
\end{adjustwidth*}

\subsubsection{MAGIC, CYBERWARE, AND INJURY}

A character's essence is a combination of their life force and humanity, together providing a mystical connection to the astral planes and allowing the character, whether they are an adept, mage, or shaman, to perform magic. Anything that interferes with this connection can reduce a character's essence and consequently reduce their magical abilities. Receiving a serious injury or implanting cyberware into one's body are the most common ways a character can lose essence. As a person loses parts of their body or has them replaced by machines, their life force is diminished. 

While mundanes typically pay little attention to a loss of essence, this can have serious consequences for a magic user. Adepts can lose access to magically enhanced abilities, while mages and shamans will have a much harder time casting spells or summoning spirits. So while that implanted sword arm might look enticing to a hardcore melee adept, it is critically important to balance the benefits of the chrome against the loss of one's magical abilities.

\subsection{MANA BARRIERS}

\textbf{Mana barriers} are protective wards shaped by a mage's will. Mana barriers block both astral and physical objects, including spells. They also act as solid barriers towards astral  perception and projection. However, barriers are typically limited in scope, either placed across narrow choke points, like hallways, or as full domes several meters in diameter. Larger mana barriers are possible, but only with special training and when working within a group of mages. These large barriers are typically only employed to protect high value corporate or government sites, in conjunction with more traditional security.

From a game perspective, mana barriers possess a single stat: force. The barrier's force determines the following qualities:

\begin{adjustwidth*}{.5cm}{.5cm}

\textbf{Wound Boxes \& Armor} A mana barrier has its force X 2 as wound boxes. It has its force $\div$ 2 as armor.

\textbf{Duration} A mana barrier will last for a number of days equal to its force.

\end{adjustwidth*}

If you can't wait for a barrier to naturally dissipate, it is possible to destroy one using brute force. Barriers don't fight back, so a Rock \& Roll move doesn't apply, but the combination of high armor and numerous wound boxes makes destroying a mana barrier potentially time consuming. Alternatively, an awakened character can attempt to pass through a barrier without destroying it. A word of warning however: dealing damage to a barrier will immediately alert the mage who created it. Passing through a barrier is more subtle, but still carries risks of detection.

\textbf{Jump the Fence:} when you \textbf{pass through a mana barrier as an awakened character}, roll+Stamina. On 10+, you pass through unnoticed.  On 7-9, you pass through, but the barrier's creator is alerted to the attempt. On a failure, you fail to pass through and the barrier's creator is alerted.

\subsection{ASTRAL SPACE}
Much like the Matrix, Astral Space is a sort of alternate universe adjacent to our own. It is where spells, spirits, magical
creatures, wards and more reside.

When an individual \textbf{perceives} the Astral, they can see the
entities existing in Astral Space. All three arcane archetypes
can astrally perceive. In addition, they can perceive emotional
auras of living beings, as well as background magical nature
of the area. When an individual \textbf{projects} themselves into astral space, they transfer their consciousness from their physical body to the astral plane, and can fully interact with other
Astral entities and traverse great distances. The Shaman and
Mage can astrally project.

The following effects occur while perceiving or
projecting:
\begin{adjustwidth*}{.5cm}{.5cm}
\textbf{Perceiving:} while astrally perceiving, take -2 ongoing to
any moves in the physical world.

\textbf{Projecting:} you cannot take action in the physical world
(your body is unconscious and helpless). When you make
moves in astral space, always roll +Craft, instead of the
usual stat.
\end{adjustwidth*}

\subsubsection{ASTRAL QUESTS}
The Astral also serves as a huge deposit of magical information, though most of the deepest knowledge is hidden in the
metaplanes. Metaplanes are the planes beyond the Astral,
the real sources of all magic. Every metaplane has a \textbf{citadel}, a
core of pure magical energy that can alter the magical world.
Accessing it can let you destroy a spirit permanently, learn
some information such as the true name of a spirit, or learn
an individual’s true aura. Note, however, an astral quest may
only have a single goal.

Astral Quests are also dangerous in that you are stuck in a
metaplane until you either complete your Quest or fail. You
can’t give up, and you can never go back, only forward.

\subsubsection{DOMAINS}
To go on an Astral Quest, you must visit various metalocations known as domains, similar to Nodes in the Matrix (in
fact, mapping these \textbf{domains} is a useful tool to keep play on
track and engaging). The number and nature of these domains depends on the quest you are undertaking, but each
one presents a challenge the character must complete in order to move on to the next domain. This could be fierce combat, a riddle, a puzzle or any variety of things.

Minor quests usually have 3 or 4 domains, while major quests
can have up to 10 or more, all of which lead, ultimately, to
the Citadel, where the quester will find the object or information they seek. Moving from domain to domain is as simple
as willing yourself there once the task in the current domain
is completed.

\subsubsection{THE DWELLER}
The first domain you encounter is always the Domain of the
\textbf{Dweller}, a mystical being who blocks the entrance to the
metaplanes. The Dweller knows everything about the quester,
and will always question the nature your quest before granting passage. The Dweller is an enigmatic trickster, but if you
go on quests often, you’ll get to know this being quite well.

\end{multicols}


\invisiblepart{THE MATRIX}
\section{THE MATRIX}
\begin{multicols}{2}
The \textbf{Matrix}, a world-spanning high-fidelity virtual reality network, is the domain of the Hacker. A hacker’s job is unique,
and the conflicts they face usually take place in the gleaming
virtual world of the matrix. However, this conflict is no less
important—or deadly—than the one their street sam buddy
is going through. With security hackers, rogue software, and
deadly black IC out there, a piece of Matrix code can be every
bit as lethal as a 7.62mm bullet.

\subsection{BUILDING SYSTEMS}
Including matrix and hacking challenges for the Hacker is
one of the things the GM should keep in mind as gameplay
evolves; a hacker with nothing to hack is a sad panda indeed.
One way to do so is outlining a \textbf{system}. This is different from
hacking devices individually or wireless hacking (see “Wired or Wireless?”).

\subsubsection{NODES}
A matrix system is made up of a series of Nodes. Each node
represents a particular secured (or, if the hacker is lucky,
non-secured) region of the network that can be penetrated
and controlled. GM’s are encouraged to draw simple maps of
connected nodes, or create a list of different nodes and brief
notes about them for to use when the Hacker starts slinging
code.

Different nodes have different purposes, challenges, and payoffs:
\begin{adjustwidth*}{.5cm}{.5cm}
\textbf{Security Node:} this node houses and dispatches intrusion
countermeasures.

\textbf{Datastore:} this node contains data, and may have encryption or even a data bomb failsafe to render data useless if
intrusions are detected

\textbf{Credentials Node:} contains user credentials or grants permissions which can help the hacker avoid detection or access secured areas

\textbf{Process Node:} runs a process on the network, slowing
down the activity of other system software

\textbf{Control Node:} this is a node to which multiple device
nodes are connected; it serves as a master controller for
the attached devices.

\textbf{Device Nodes:} a single device connected to the network.
Devices range from cameras to security drones to maglocks; almost everything is connected. Devices are frequent targets for intrusion attempts. Most simple devices
have minimal privilege on the network, but that is often
enough.
\end{adjustwidth*}
\subsubsection{ARMORED NODES}
Many matrix nodes have only one layer of security: once 
you hack in, the node is yours. However, more secure systems have additional defenses. These nodes, called \textbf{armored 
nodes}, are both hardened against intrusion and contain intrusion countermeasures.

Mechanically, Armored Nodes have both Wounds (how many
is up to the GM), and embedded Intrusion Countermeasures
(see \textbf{Threats}, page 46) which fight back against intruding
hackers.

It’s possible to have nodes that have only Wounds, but no
defensive IC. In this case, the node is effectively defenseless,
and the Hacker simply deals damage to the node.

\subsubsection{ALERT LEVELS}
A System has four \textbf{Alert Levels}, representing both how aware
the system is that it has been compromised, and how actively
it will attempt to locate, identify, and stop the
intrusion.
\begin{adjustwidth*}{.5cm}{.5cm}
\textbf{Green:} the system is unaware that it has been compromised.

\textbf{Yellow:} the system has detected a possible intrusion. Routine notifications are dispatched, but no direct countermeasures are taken.

\textbf{Orange:} the system is aware of an intrusion and is actively
trying to locate, disable, and trace the hacker. Nonlethal
countermeasures are approved.

\textbf{Red:} the system is aware of a serious intrusion. Lethal
countermeasures are approved.
\end{adjustwidth*}
\subsection{HACKING}
When a Hacker encounters a node or device, he or she must
first hack into the node using the \textbf{Sling Code} move. Once
inside, the Hacker can transit through the node, or take advantage of any actions or bonuses the node provides (unless
it is an Armored Node or is protected by IC, in which case it
will not be nearly so trivial to use the node’s
functions).

\subsubsection{WIRED OR WIRELESS?}
Although node maps evoke a particular style of Matrix
runs, namely using the “wired connection” paradigm of
older editions of Shadowrun, you can easily use wireless
hacking, or a mix of the two. For wireless hacking, all
devices are a node. They may contain multiple nodes
inside, as well, or be standalone., but they’re also usually
accessible via a wireless connection (or if not, accessible
via connection to another device that is).

Devices such as firearms, cyberware, and other items carried by individuals are also fair game for hacking. In such
case, assume them to be armored nodes. You’ll need
to indicate how many wounds the device has, and how
much damage it can do to a hacker, if any.

A sample device might be:
\begin{adjustwidth*}{.5cm}{.5cm}
\textbf{Commlink} [6 wounds, 1d4 stun dmg]
\end{adjustwidth*}
An armored node or device can only deliver its damage
in matrix combat; the commlink above didn’t suddenly
 become a taser.

\end{multicols}


\invisiblepart{LEGWORK \& DOWNTIME}
\section{LEGWORK \& DOWNTIME}
\begin{multicols}{2}
While most of the interesting parts of \SW{} happen
in the middle of a shadowrun, most shadowrunning teams, if
they have the opportunity, will take time to do some research
on their run and the people associated with it, and gather
necessary equipment, before they stick their head in the alligator’s mouth.

Likewise, after a run, shadowrunners might take some time
to go to ground, heal up their wounds, spend some of their
ill-gotten nuyen, and generally maintain a low profile while
the aftermath of their latest job blows over. The
cycle of activity in \SW{}, then, can usually be
described as

\begin{adjustwidth*}{1cm}{2cm}
{\orbitronfont \textbf{LEGWORK > THE RUN > DOWNTIME}}
\end{adjustwidth*}

(Please note this is descriptive, not prescriptive: your games
don’t have to resemble this in the least, if you don’t want
them to!)

In \SW{}, the research portion of the run is called legwork, and the time after a run—and before the work starts on
the next run—is generally referred to as downtime. While
legwork has some optional rules to structure it, downtime is
much less rules-oriented, and is handled much like downtime
in other games: narratively, as a chance for players to talk
about what’s going on without rolling dice, and to set the
stage leading up to the next run.

\subsection{LEGWORK}
Shadowrunners do not (always) charge headlong into danger,
guns and spells blazing. In fact, those who do generally only
do it once.

Instead, a savvy runner does legwork before a run, getting as
much information as possible within the time they have. This
section outlines how to play through the legwork process,
letting the players create details that give them advantages,
while giving you a few wrenches to throw in the works in return. The methodology below was originally described in the
“Dirty Dungeons” segment of John Wick’s Play Dirty gaming
advice videos, and is an option for lending more mechanical
weight behind the legwork that goes into a shadowrun. There
are 3 basic steps:

\textbf{1.  PROVIDE THE ANCHOR}

Give the players a premise they have to deal with. This can be
anything from “extract scientist X from the corporate facility
at Y” to “a Humanis Policlub group is preparing a terrorist
attack and we want it stopped.”

\textbf{2.  START THE LEGWORK}

During the actual legwork, characters search for information,
speak to contacts and other NPCs, purchase or otherwise acquire equipment, get assets into position, and discover details that will help flesh out the mission. Details discovered
in this fashion are awarded through moves taken during the
legwork phase.

When a detail is uncovered, the player establishes the nature of the detail: what it is and why it’s valuable.
Details found this way can be anything from floor plans to
passkeys to security procedures, whatever a player might
think is useful. Problematic details (too much of an advantage,
one-shot-mission-solvers, mission-evaders, and the like),
however, should be discussed immediately, and replaced
with something else that’s more reasonable and
believable.

When a character discovers or establishes a detail, add a point
to the Mission Pool (it’s probably best to use poker chips or
pennies or something to track Mission Points). Continue gathering details and building the Mission Pool until the players
are satisfied or any game-imposed time limits run
out.

\textbf{3.  GATHER COMPLICATION POINTS}

While the players are prepping their info, they are also building up a number of Complication Points you’ll have available. Every Legwork move specifies how much time is spent,
and for every day of “game world time” spent on Legwork,
you add one point to your Complication Pool—the longer
they spend getting ready, the more likely it is that the details
might change a bit.

\subsubsection{MISSION POINTS}
At any point during the run, a player may draw one point
from the Mission Pool and spend it to boost their next move.
Players must use the Mission Point on their next move (they
can’t hold onto it until later - once drawn from the pool, it’s
use it or lose it). Additionally, once a Mission Point is used, it
is removed from the mission pool. Mission Pools do not refresh (the only way to get another mission pool is, of course,
to get another mission).

\subsubsection{COMPLICATION POINTS}
When the characters gather information for a run, it is important for the GM to remember that all of the information they
gather is true. Detail gathering is an opporunity for players to
declare what they know to be true about a mission, and not
an opportunity for the GM to feed them erroneous information. On the other hand, if everything always went exactly to
plan, it wouldn’t be a shadowrun!

To introduce these little wrinkles, the GM may spend complication points to throw a small wrench into the works, by
declaring a change or inaccurracy in one of the
details discovered during msision prep.

\begin{adjustwidth*}{.5cm}{.5cm}
\textbf{Example:}\textit{ during mission prep, the characters discovered
that security patrols on the 6th floor of their target building happen in two shifts, but there is a 5 minute gap in
coverage they could exploit. As they approach the entry
point from an adjacent building, the GM elects to spend
a complication point to introduce a twist - a new guard
is being trained, and he and his supervisor happen to be
right near the window where the team was going to make
their entry.}
\end{adjustwidth*}

Complication Points are an opportunity to use a GM Move 
to alter a detail the characters discovered legwork (in the 
example above, the GM has revealed an unwelcome truth 
about the security patrols), with the added concession that 
you have spent a limited resource in order to do
so. 

In that vein, a caution to the GM: use care when introducing 
complications. Remember that much of the detail provided 
by the players will be plenty exciting - and get plenty complicated - simply by playing to see what happens, Because 
success with a cost is a constant companion in \SW{}, 
the characters’ own actions are going to complicate things, so 
  you should let the details they have help them
  out. 

Finally, remember that Complication Points can only be spent
to alter a mission detail, and they must be spent if you wish
to do so. Spend carefully, and only when it will make things
more interesting – never just to screw the characters.
Like Mission Points, Complication Points, once spent, are
gone.

\subsubsection{LEGWORK MOVES}
This section’s title is a bit of a misnomer. \SW{} doesn’t
specify a fixed set of approved “legwork moves,” nor any
“legwork only” moves. Nevertheless, several moves (both
secondary moves as well as some archetype moves) involve
preparation, information gathering, training, and similar activities, Moves that feature prominently in preparation and
legwork include:
\begin{adjustwidth*}{.5cm}{.5cm}
\tcirc{} Citation Needed

\tcirc{} Pull Strings

\tcirc{} Hit the Books

\tcirc{} Go Shopping

\tcirc{} Build a Legend (Face)

\tcirc{} I Know A Guy (Face)

\tcirc{} Contracts Available (Mercenary)

\tcirc{} Field Trial (Mercenary)

\tcirc{} Gun Cage (Ex-cop)

\tcirc{} Pharmacy is Open (Street Doc)
\end{adjustwidth*}

\subsubsection{OTHER ACTIVITIES}
Other activities that can be done during legwork (or during
downtime) include writing programs (page 66), spellcrafting (page 67), working on gear (page 60), or bonding
with new spirits (page 69). The rules for each of those activities specify the time the character must spend to successfully complete the activity.

\subsection{DOWNTIME}
Downtime is, in effect, “free time” for the characters. This is
the time spent dealing with their lives outside of shadowrunning: recovering from injury, paying their rent, working out,
getting drunk, or spending time with family (believe it or not,
not every shadowrunner is a hyperparanoid loner drifter with
nothing to lose).

Time spent in downtime is handled in a narrative fashion.
If something done during downtime specifies an amount of
time required, that time is spent, but that serves mainly to
indicate the overall passage of time in the world, rather than
racing toward an oncoming deadline.

On the other hand, the world does live and breathe. If an
event is coming, it will happen when it happens, and will not
necessarily wait for the characters’ schedules to line up. (On
 the upside, unless the event is “bombs fall, everybody dies,”
    then world events that happen during downtime should only
   serve to make the runners’ lives more interesting).

\subsubsection{DOWNTIME MOVES}
Although downtime is largely a move-free time, moves can
occur then. One move that must occur during downtime is
the Advance move (page 5), where characters can to reflect on their experience and improve themselves.

\end{multicols}

\newpage


\invisiblepart{EQUIPMENT}
\section{EQUIPMENT}
\begin{multicols}{2}
In this section you’ll find example equipment (weapons, cyberdecks, vehicles, etc.) available in the \SW{}. This isn’t
an exhaustive list of what’s available; rather, they’re just samples of some classic items to help you get playing quickly.
Also, although it’s not exactly the correct word, in this document the term equipment refers to pretty much any resource
the character has (so spells and spirits are also considered
“equipment” for the sake of simplicity).

\SW{} also offers rules to create customized
and personalized versions of the following:
\begin{adjustwidth*}{.5cm}{.5cm}

\tcirc{} weapons

\tcirc{} cyberdecks

\tcirc{} vehicles and drones

\tcirc{} spells

\tcirc{} programs

\tcirc{} spirits
\end{adjustwidth*}

If you want to create and customize your own stuff, check out
the \textbf{Creating Gear} section starting on page 60. That section explains \SW{}’s “template-based” customization
system.

Of course, you should also feel free to simply make up new
equipment or add in things you think are missing—just because there isn’t a set of creation rules for something doesn’t
mean it doesn’t exist!

\subsection{EQUIPMENT TAGS}
Equipment—like many items in \SW{}—is described in
terms of \textbf{tags}, which are short keywords that indicate various
capabilities or qualities. Certain tags apply to multiple kinds
of equipment (such as obvious, supply, or armor). Tags that
only apply to specific kinds of equipment are described in the
listing of that kind of item. The following tags apply to multiple types of equipment.
\begin{adjustwidth*}{.5cm}{.5cm}

\textit{2-hand:} this item must be used with both hands

\textit{armor +n:} grants a +n bonus to existing armor

\textit{armor n:} grants n Armor (for vehicles or drones, indicates
armor rating, and is abbreviated arm)

\textit{arcane:} can only be used by magical archetypes

\textit{area:} affects multiple targets

\textit{+bonus:} grants a bonus to a particular move; e.g. +1 to
Stay Frosty

\textit{conceal:} this weapon or item is easily hidden and will not
be spotted by enemies

\textit{damage n:} the amount of damage a weapon or other item
deals. Abbreviated dmg

\textit{heal n:} restores n wounds

\textit{ignores armor:} bypasses the target’s armor

\textit{loud:} noisy and audible to anyone with functioning hearing;
for weapons, it means the weapon cannot be
suppressed

\textit{messy:} deals damage in a particularly gruesome way

\textit{obvious:} cannot be concealed, or is immediately visible to
any observer

\textit{range:} the range(s) at which the weapon or other attack is
effective. Ranges are \textbf{touch (t)}, \textbf{close (c)}, \textbf{short
(s)}, \textbf{medium (m)}, and \textbf{long (l)}.

\textit{shock:} the weapon deals electrical shock

\textit{special (description):} if the effect of the item requires explanation, use this tag.

\textit{stun:} this weapon or attack deals Stun damage only

\textit{subtle:} not easily noticed (as opposed to conceal, which
means it is unnoticeable)

\textit{supply n:} the amount of supplies or uses you can get out
of an item. Each use of the item consumes 1 supply (unless
otherwise stated).
\end{adjustwidth*}


\subsection{WEAPONS}

\textbf{WEAPON TAGS}
\begin{adjustwidth*}{.5cm}{.5cm}

\textit{2-hand:} this item must be used with both hands

\textit{AP n:} this weapon ignores n points of armor; note that
each point of AP requires the payment of the 25\% customization premium

\textit{auto:} this weapon can fire in full auto mode. Abbreviated
fa.

\textit{burst:} this weapon fires in burst mode. Mark off 1 additional Ammo to deal +1 damage. Abbreviated bf.

\textit{chem:} this weapon delivers a chemical agent of some kind
to the target; depending on the delivery mechanism, armor may be ignored.

\textit{forceful:} when this weapon deals damage, it also deals 1
stun

\textit{fuzed:} this weapon cannot be used at less than the shortest
range increment listed

\textit{reload:} after using this weapon, it takes more than a moment to reload it.

\textit{semiauto:} this weapon fires one shot every time the trigger
is pulled. Abbreviated sa.

\textit{stabilized:} this weapon cannot be fired except from a bipod, tripod, or supported position.

\textit{suppressed:} this weapon makes little to no noise when
fired

\textit{thrown:} this item can be throw. If thrown, the range is
short.

\textit{vented:} the weapon has recoil venting, granting +1 to
Suppression Fire
\end{adjustwidth*}

\subsubsection{MELEE WEAPONS}
\textbf{Staff} [\textit{range c, stun, 1d6+2
  damage, 100¥}]

\textbf{Combat Axe} [\textit{range c, messy, 1d6+2
  dmg, 1,250¥}]

\textbf{Combat Knife} [\textit{range c, 2d4b dmg,
  1 AP, 300¥}]

\textbf{Compound Bow} [\textit{range s/m/l, 2-hand, dmg 1d6+1, ammo
1, 500¥}]

\textbf{Crossbow} [\textit{range c/s/m, 2-hand, dmg 1d6, suppressed, reload, 400¥}]

\textbf{Fists/Feet} [\textit{range c, 1d6 dmg,
  stun}]

\textbf{Katana} [\textit{range c, 2d6b damage,
  1,000¥}]

\textbf{Spiked Glove} [\textit{range c, 1d4 wound
  + 1 stun, 50¥}]

\textbf{Stun Baton} [\textit{range c, 1d4 dmg, stun, shock, ignores armor,
750¥}]

\textbf{Tomahawk} [\textit{range c, messy, thrown,
  1d6 damage, 200¥}]


\subsubsection{HOLD-OUT PISTOLS}
\textbf{Streetline Special} [\textit{range s, sa, dmg 2d4b, ammo 3, con- 
ceal, 250¥}] 

\textbf{ Fichetti Needler} [\textit{range s, dmg 2d4b, AP 1, conceal, ammo 
 3, 400¥}] 

\textbf{Walther PP} [\textit{range s, sa/bf, dmg 1d4+1, ammo 1, conceal, 
  325¥}] 


\subsubsection{LIGHT PISTOLS}
\textbf{Colt L36} [\textit{range s/m, sa, dmg 1d6,
  conceal, ammo 3, 500¥}]

\textbf{Beretta 101T} [\textit{range s/m, sa/bf, dmg 1d6, subtle, ammo 2,
450¥}]

\textbf{Ares Lightfire 70} [\textit{range s, sa, dmg 1d6, conceal, ammo 3,
350¥}]


\subsubsection{HEAVY PISTOLS}
\textbf{Ares Predator} [\textit{range s/m, dmg 1d8+1, sa, AP 2, 3 ammo,
675¥}]

\textbf{Colt Manhunter} [\textit{range s/m, dmg 1d8, sa/bf, AP 1, 3 ammo,
560¥}]

\textbf{Ruger Super Warhawk} [\textit{range s/m, dmg 1d10, sa, AP 1, 2
ammo, loud, 560¥}]

\textbf{Browning Max Power} [\textit{range s/m, dmg 2d8b, sa, 3 ammo,
675¥}]


\subsubsection{SUBMACHINE GUNS}
\textbf{HK227} [\textit{range s/m, sa/bf, dmg 1d8, suppressed, ammo 4,
900¥}]

\textbf{AK-97K} [\textit{range s/m, sa/fa, dmg
  1d8, AP 1, ammo 3, 1,000¥}]

\textbf{Ingram Smartgun} [\textit{range s/m, bf/fa, dmg 1d6+1, AP 1,
ammo 3, 950¥}]


\subsubsection{ASSAULT RIFLES}
\textbf{AK-97} [\textit{range s/m/l, 2-hand, sa/fa, dmg 1d10, AP 1, obvious, ammo 3, 800¥}]

\textbf{Ares Alpha} [\textit{range s/m/l, 2-hand, sa/bf/fa, dmg 2d8b, AP 1,
obvious, ammo 4, 1,150¥}]

\textbf{Colt M22A2} [\textit{range s/m/l, 2-hand, sa/bf, dmg 1d10, AP 1,
obvious, ammo 3, 850¥}]

\textbf{FN-HAR} [\textit{range s/m/l, sa/bf, dmg 2d8b, AP 2, obvious, loud,
2-hand, 1,050¥}]


\subsubsection{SHOTGUNS}
\textbf{Remington 990} [\textit{range s/m, sa, dmg 1d10+1, obvious, loud,
forceful, ammo 2, 750¥}]

\textbf{Enfield AS7} [\textit{range s/m, 2-hand, sa/bf, dmg 1d10, obvious,
loud, forceful, ammo 3, 900¥}]


\subsubsection{SNIPER RIFLES}
\textbf{Ranger Arms} [\textit{range l, sa, 2-hand, dmg 1d10+1, AP 3,
ammo 3, 1,150¥}]

\textbf{Walther WA2100} [\textit{range 1, sa, 2-hand, dmg 1d12, AP 2,
ammo 4, 1,100¥}]


\subsubsection{HEAVY WEAPONS}
\textbf{Ingram Valiant LMG} [\textit{range m/l, 2-hand, loud, fa, stabilize,
obvious, loud, messy, dmg 1d12, ammo 4, AP 1,
2,000¥}]

\textbf{Stoner M202 HMG} [\textit{range m/l, 2-hand, loud, bf/fa, stabilize,
obvious, loud, messy, dmg 2d10b, ammo 3, AP 2,
2,500¥}]

\subsubsection{SPECIAL WEAPONS}
\textbf{Narcoject Rifle} [\textit{range s/m, 1d8+1 stun, suppressed, chem,
slow, 700¥}]

\textbf{Taser} [\textit{range s, 1d8 stun, shock,
  slow, 500¥}]


\subsubsection{GRENADES}
\textbf{EMP} [\textit{thrown, area, shock,
  disables electronis, 95¥}]

\textbf{Flash} [\textit{thrown, area, stun, dmg 2d4, +1 to Rock \& Roll/Stay
Frosty, 125¥}]

\textbf{Frag} [\textit{thrown, area, forceful, dmg
  2d6b, 100¥}]

\textbf{Incendiary} [\textit{thrown, area, 2d6b
  dmg, burn, 75¥}]

\textbf{Smoke} [\textit{thrown, area, +1 to Stay
  Frosty, 40¥}]

\textbf{Stun} [\textit{thrown, area, dmg 2d6b, stun, 100¥}]

\subsection{ARMOR}
Armor provides protection against incoming attack, reducing
the damage dealt by the armor value. Armor of the same
type (e.g inherent) does not stack. Armor of differing types
can stack. Armor has the following unique tags:
inherent: this armor is either implanted, or occurs naturally.
Cyberware armor is inherent armor.
worn: this armor is worn on the body
mystic: this armor is magical in nature

\subsubsection{SAMPLE ARMOR}
\textbf{Lined Coat} [\textit{armor 2, obvious,
  worn, 600¥}]


\textbf{Ballistic Vest} [\textit{armor 2, obvious,
  worn, 750¥}]

\textbf{Armorweave Professional Wear} [\textit{armor 1, subtle, worn,
1,500¥}]

\textbf{Chameleon Suit} [\textit{armor 1, conceal, worn, +1 to Stay Frosty, 6,000¥}]

\textbf{Leather Armor} [\textit{armor 1, subtle,
  worn, 250¥}]

\textbf{Armor Charm} [\textit{armor +1, mystic,
  conceal, 400¥}]

\textbf{Light Armor Jacket} [\textit{armor 2,
  subtle, 850¥}]

\textbf{Combat Armor} [\textit{3 armor, obvious,
  2,500¥}]

\textbf{Form-fitting Armor} [\textit{armor 1,
  conceal, 550¥}]

\textbf{Riot Shield} [\textit{armor 2, occupies
  one hand, 700¥}]


\subsection{CYBERDECKS}
Cyberdecks are the essential tool of the hacker. They are the
Hacker’s connection to the Matrix. Cyberdecks have
the following special tags:

\begin{adjustwidth*}{.5cm}{.5cm}
\textit{CPU:} the raw processing power of the deck

\textit{Mask:} the stealthiness of a cyberdeck

\textit{Hardening:} the deck’s resistance to damage; this
acts as armor protecting the hacker

\textit{Storage:} the deck’s capacity for loaded programs
\end{adjustwidth*}

\subsubsection{EXAMPLE DECKS}
\textbf{Allegiance Alpha} [\textit{CPU 1, mask 1, hardening 1, storage 8,
25,000}]

\textbf{Fuchi Cyber-4} [\textit{CPU 1, mask 2, hardening 1, storage 8,
50,000¥}]

\textbf{Fuchi Cyber-7} [\textit{CPU 3, mask 1, hardening 1, storage 8,
75,000¥}]

\textbf{Fairlight Excalibur} [\textit{CPU 3, mask
  2, hardening 1, 100,000¥}]


\subsection{PROGRAMS}
Programs run on a cyberdeck. Hackers don’t need programs
do to their job—they can sling code well enough to bend the
matrix to their will on the fly—but a program can improve
their chances or offer special tricks to help the
hacker.

Programs have the following special tags:
\begin{adjustwidth*}{.5cm}{.5cm}

\textit{routines:} the different routines that make up the program.
See \textbf{Writing Programs}, page 66, for details about routines.

\textit{size n:} the amount of space a program takes up in the cyberdeck’s storage.
\end{adjustwidth*}

Armor or damage tags on programs only work when in the 
Matrix. 

\textbf{RUNNING PROGRAMS 
}
When a program is loaded into the storage on a cyberdeck, 
it is assumed to be running. If the hacker has to change pro- 
grams, they may do so at any time; however, if it would be 
despite risk of some sort (for instance, while in combat with 
IC), then they must \textit{Stay Frosty}. 

\subsubsection{AGENTS}
Hackers can compile separate programs into pseudo-sentient
matrix entities called \textbf{agents}. See the Programs section (page
66) for more information.

\subsubsection{SAMPLE PROGRAMS}
\textbf{Armor} [\textit{armor +2 (matrix only), routines (armor x 2), size 4,
500¥}]

\textbf{Black Hammer} [\textit{dmg 1d6, relocate hostile programs, routines (armor, bounce), size 4, 500¥}]

\textbf{Stealth} [\textit{mask +2, routines(stealth x 2, interference), size 6,
750¥}]

\textbf{Lockpick} [\textit{mask +1, +1 to hack Data nodes, routines(stealth,
decrypt), size 4, 500¥}]

\textbf{Assassin} [\textit{mask +1, dmg 2d6b, armor +1, routines(stealth,
armor, attack x 2), size 8, 1,000¥}]

\textbf{Ghost} [\textit{mask +2, routines (stealth
  x 2), size 6, 500¥}]

\textbf{Tarpit} [\textit{slow alarms and relocate hostile programs, routines
(bounce x 2, interference), size 6, 500¥}]

\textbf{Bloodhound} [\textit{+2 Check the Situation in the matrix, +1 to
hack data nodes, routines (analyze x 2, decrypt), size 6,
750¥}]

\textbf{Medic} [\textit{heal 2 matrix damage, routines(repair x 2), size 4,
500¥}]

\textbf{Codebreaker} [\textit{+2 to decrypt data nodes, routines(decrypt x
2), size 4, 500¥}]


\subsection{VEHICLES}
Vehicles have the following special tags:

\begin{adjustwidth*}{.5cm}{.5cm}
\textit{Power (pwr):} the vehicle’s horsepower,
speed, and acceleration.

\textit{Armor (arm):} the vehicle or drone’s armor
rating.

\textit{Frame (frm):} the vehicle’s or drone’s resilience. This is the
equivalent of a vehicle’s wounds. Remember that small
arms deal half damage to vehicles (see Dealing Damage,
page 9).

\textit{Sensors (ssr):} the quality of the vehicle’s sensors (used
when Checking the Situation while driving or piloting the
vehicle)

\textit{Seats n:} the number of people who can normally occupy
the vehicle, including the driver or pilot

\textit{Fuel:} fuel or battery capacity
\end{adjustwidth*}

\subsubsection{BIKES}
\textbf{Dodge Scoot} [\textit{seats 1, pwr 1, arm 0, frm 4, ssr 0, fuel 4,
1,800¥}]

\textbf{Yamaha Rapier} [\textit{seats 1, pwr 2, arm 0, ssr 1, frm 4, fuel 4,
9,500¥}]

\textbf{Harley Scorpion} [\textit{seats 2, pwr 2, arm 1, frm 7, ssr 1, fuel 2,
17,500¥}]


\subsubsection{CARS \& TRUCKS}
\textbf{C-N Jackrabbit} [\textit{seats 3, pwr 1, frm 6, ssr 0, arm 0, fuel 3,
10,000¥}]

\textbf{Ford Americar} [\textit{seats 4, pwr 1, frm 8, ssr 1, arm 0, fuel 3,
16,000¥}]

\textbf{Eurocar Westwind} [\textit{seats 6, pwr 3, frm 9, arm 1, ssr 1, fuel
3, 200,000¥}]

\textbf{GMC Bulldog} [\textit{seats 8, pwr 2, frm 9, arm 1, ssr 1, 3 fuel, seats
8, 45,000¥}]

\textbf{Ares Roadmaster} [\textit{seats 6, 3 pwr, 11 frm, 2 armor, 2 fuel,
52,000¥}]


\subsection{DRONES}
Drones have most of the same qualities as vehicles, although
they lack the seats tag, and replace it with the
following:

\begin{adjustwidth*}{.5cm}{.5cm}
\textit{Tactical:} the quality of the drone’s tactical expert system,
which comes into play when the drone is in autonomous
mode. Abbreviated tac.
\end{adjustwidth*}

Armed drones also use the \textit{damage} tag, indicating the damage of their built-in weapon systems.

\subsubsection{GROUND DRONES}
\textbf{Aztechnology Crawler} [\textit{pwr 1, frm 5, ssr 2, arm 0, tac 0, fuel
3, 4,000¥}]

\textbf{GM-Nissan Doberman} [\textit{pwr 1, frm 7, arm 1, ssr 1, dmg 1d6,
tac 1, fuel 3, 5,000¥}]

\textbf{Steel Lynx} [\textit{pwr 1, frm 9, arm 2, ssr 1, tac 2, dmg 2d6b, fuel
2, 9,500¥}]


\subsubsection{AIRBORNE DRONES}
\textbf{Lockheed Optic-X} [\textit{pwr 1, ssr 2, arm 0, frm 2, tac 1, fuel 2,
12,500¥}]

\textbf{MCT Roto-Drone} [\textit{pwr 2, frm 5, arm 0, ssr 1, dmg 2d4b, tac
1, fuel 2, 15,750¥}]

\textbf{CD Dalmatian} [\textit{pwr 1, frm 8, arm 1, ssr 0, tac 2, dmg 1d8,
fuel 3, 22,000¥}]


\subsection{CYBERWARE}
The cyberware items in the Archetype’s starting packages are
shown here with all their tags. Cyberware has the following
special tags:
\begin{adjustwidth*}{.5cm}{.5cm}

\textit{add-ons:} this is installed in an existing piece of cyberware,
instead of independently. The item takes up capacity equal
to its essence cost.

\textit{always on:} the implant remains on all the time. If adding
this tag to an item that modifies a move, multiply the cost
of the implant by 2.

\textit{capacity n:} the cyberware item has capacity for n add-on
items.

\textit{device:} this implant is a device of some sort (usually a
weapon or computing tool) that does not offer sensory
modification.

\textit{link (device):} this cyberware must be connected to the
proper kind of device to be effective

\textit{loaner:} this implant was given to you by an organization
lots of money, and they expect you to repay them somehow.

\textit{resist (hazard):} the augmentation protects against particular environmental hazards such as toxins or
electrocution

\textit{sealed:} a sealed implant requires at least an hour and the
proper tools to reload or refill.

\textit{sota:} state of the art; sota cyberware has a lower essence
cost than equivalent standard cyberware

\textit{toggle:} this item is toggled on and off (that is, once activated, it stays on).

\textit{used:} this implant started its life in someone else’s body.
The first time you fail a move related to the implant or
are in a situation where the added capability of the device
comes into play, roll 1d6. On a 3 or better, you’re fine. On
a 2, the implant simply fails gracefully. On a 1, the implant
goes haywire:

\begin{adjustwidth*}{.5cm}{.5cm}
\tcirc{} If the implant modifies a move, that move is glitched
until you get it fixed or shut down

\tcirc{} If the implant provides a capability, that ability suddenly
poses a big problem

\tcirc{} You can shut down a haywire implant by spending a
point of Edge.
\end{adjustwidth*}
\end{adjustwidth*}

\subsubsection{ACTIVATING CYBERWARE}
To gain the benefits of any of the following items, you must
spend a point of Edge to activate the implant. Implants that
offer no mechanical benefit (related to moves or defenses),
such as cyberlimbs, are always on—you don’t have to spend
edge to use them.

\subsubsection{HEADWARE}
EYES
\textbf{Cybereyes} [\textit{always on, capacity 2,
  essence 1}]

\textbf{Thermographic Enhancement} [\textit{ability(thermographic vision), essence 1}]

\textbf{Vision Magnification} [\textit{always on, ability(long distance vision), essence 1}]

\textbf{Low-light enhancement} [\textit{ability(low-light vision), essence
1}]

\textbf{Camera} [\textit{ability(record video or
  images), essence 1}]

EARS
\textbf{Cyberears} [\textit{always on, capacity 2, essence 1}]

\textbf{Damper} [\textit{ability(resist:sound),
  essence 1}]

\textbf{Noise Filter} [\textit{ability(enhanced
  hearing), essence 1}]

\textbf{Recorder} [\textit{ability(record audio or
  video), essence 1}]

\textbf{Ultrasound System}
[\textit{ability(perceive ultrasound), essence 1}]

OTHER
\textbf{Cranial Cushion} [\textit{always on, armor
  +1 vs. stun, essence 1}]

\textbf{Tactical Computer} [\textit{modifies(Check the Situation: use Combat instead of Awareness), essence 1}]

\textbf{Synaptic Hardening} [\textit{armor
  +1(matrix only), essence 1}]

\textbf{Voice Modulator} [\textit{ability(alter
  voice), essence 1}]


\subsubsection{BODYWARE}
\textbf{Active Camouflage} [\textit{special(if you remain motionless, enemies cannot see you), essence 2}]

\textbf{AutoDoc} [\textit{special(gain 1 extra wound box), toggle, essence
3}]

\textbf{Bone Lacing} [\textit{always on, special(deal lethal damage when
unarmed, gain 1 additional wound box), essence 2}]

\textbf{Boosted Reflexes} [\textit{modifies(Stay Frosty: hold 1), special(incompatible with wired reflexes, cannot be upgraded), essence 2}]

\textbf{Cyberarm/Cyberleg} [\textit{always on, device, obvious, capacity
2, essence 3}]

\textbf{Dermal Plating 1} [\textit{armor +1,
  inherent, always on, essence 2}]

\textbf{Dermal Plating 2} [\textit{armor +2,
  inherent, always on, essence 3}]

\textbf{FeatherTouch} [\textit{ability(enhanced
  sense of touch), essence 1}]

\textbf{Gyrostabilizer} [\textit{modifies(Suppression Fire: hold 1), essence
2}]

\textbf{Hand Razors} [\textit{range c, dmg 1d4
  dmg, essence 1, toggle}]

\textbf{Light Cybergun} [\textit{range c/s, 1d6 dmg, toggle, sealed, essence 2}]

\textbf{ReadiMed System} [\textit{modifies(First Aid: hold 1), supply 2,
sealed, special(can also modify relevant Street Doc moves),
essence 2}]

\textbf{Skillsoft} [\textit{link(skillwires), special(required for skillwires to
function; specify area of knowledge when
purchasing)}]

\textbf{Skillwires 1} [\textit{modifies(Drop Science: hold 1), link(skillsoft),
essence 2}]

\textbf{Skillwires 2} [\textit{modifies(Drop Science: hold 2), link(skillsoft),
essence 3}]

\textbf{Shocktrodes} [\textit{range c, dmg 1d4
  stun, essence 1}]

\textbf{Smartlink} [\textit{move(Rock \& Roll: add +1 damage on 10+, on
7-9, don’t mark off ammo), ranged, essence 1}]

\textbf{Spurs} [\textit{range c, dmg 1d6, essence
  2, toggle}]

\textbf{Wired Reflexes 1} [\textit{modifies(Stay
  Frosty: hold 1), essence 2}]

\textbf{Wired Reflexes 2} [\textit{modifies(Stay
  Frosty: hold 2), essence 3}]


\subsection{OTHER EQUIPMENT}
\subsubsection{DRUGS}
Costs listed below are per dose (one dose equals 1 Supply)
\textbf{Bliss} [\textit{take +1 to Gut Check,
  lasts 2 hours, 15¥}]

\textbf{Cram} [\textit{take +1 to Stay Frosty,
  lasts 3 hours, 10¥}]

\textbf{Deepweed:} [\textit{user can perceive
  Astrally, lasts 1 hour, 400¥}]

\textbf{Jazz} [\textit{take +2 to Stay Frosty,
  lasts 30 minutes, 75¥}]

\textbf{Kamikaze} [\textit{take +1 to Rock \& Roll and Gut Check, lasts 1
hour, 100¥}]

\textbf{Long Haul} [\textit{you can go without sleep for four days with no
consequence, 50¥}]

\textbf{Nitro} [\textit{take +2 to Rock \& Roll and +1 to Gut Check, lasts 30
minutes, 75¥}]

\textbf{Novacoke} [\textit{take +1 to Push Someone and Check the Situation, lasts 2 hours, 10¥}]

\textbf{Psyche} [\textit{take +1 to Drop Science,
  lasts 3 hours, 200¥}]

\textbf{Zen} [\textit{take +1 to Stay Frosty,
  lasts 30 minutes, 5¥}]

\textbf{BTLs} [\textit{allow you to experience almost anything virtually, lasts
30 minutes to 3 hours, 20-100¥}]


\subsubsection{MISCELLANEOUS}
\textbf{Medic Patch} [\textit{supply 1, heal 2,
  500¥}]

\textbf{Stimulant Patch} [\textit{supply 1, take +2 to next move, take 1 stun
afterwards, 175¥}]

\textbf{Antidote Patch} [\textit{halts poison
  damage, 200¥}]

\textbf{Trauma Patch} [\textit{supply 1, +1 to
  First Aid Move, 300¥}]

\textbf{Quik-Hax Kit} [\textit{supply 4, bypasses low-grade security locks/
electronic devices, 350¥}]

\textbf{Spy Kit} [\textit{supply 4, +1 to Citation Needed or Check the Situation (assuming bugs haven’t been found), 4000¥}]

\textbf{Countersurveillance Kit} [\textit{supply 4, +1 to Check the Situation to search for bugs, 3000¥}]

\textbf{Infiltrator’s Kit} [\textit{supply 4, +1 to Stay Frosty to infiltrate or
avoid detection, 1,000¥}]

\subsection{MAGICAL SUPPLIES}
\subsubsection{FOCI}
A focus is a mundane item that has been imbued with an astral construct. When used by someone to which it is attuned,
a focus helps them channel astral power greatly enhances
their abilities.

\textbf{ATTUNING}

Before a focus can be used, the user must \textbf{attune} themselves
to it. To do so, they must invest at least one point of essence
into the focus. Essence committed in this fashion remains
spent until the user de-attunes themselves from the focus, or
the focus is destroyed, at which point the essence is recovered.

A mage, adept, or shaman can only be attuned to a number
of foci equal to their Craft rating.

\textbf{TYPES OF FOCI}

\begin{adjustwidth*}{.5cm}{.5cm}
\textbf{Spell Focus:} a spell focus enhances the casting of a specific
spell. When attuned, the mage using the spell focus has
hold equal to the Essence spent attuing the focus. Spend
this hold toward casting that specific spell.

\textbf{Spirit Focus:} a spirit focus enhances the summoning of a
specific type of spirit. When attuned, the shaman has hold
equal to the essence invested in the focus toward summoning that specific spirit type.

\textbf{Weapon Focus:} weapon foci are primarily used by adepts.
When attuned to a weapon focus, the adept using it has
hold equal to the invested Essence to spend on the Rock
\& Roll move or on dealing damage.
\end{adjustwidth*}

\subsubsection{FETISHES}
Fetishes are essentially one-shot magical supplies—small
mundane objects imbued with structure and energy of a spell
or summon a spirit, needing only to be triggered by the
mage or shaman.

\textbf{INVESTING}

To create a fetish, the mage or shaman decides what spell or
spirit to place into the fetish, and then \textbf{invests} the fetish with
power, spending the Essence required for the spell, or the
essence they wish to provide to the spirit. Essence invested
in a fetish in this manner remains spent until the fetish is used,
at which point it immediately returns.

\textbf{ACTIVATING A FETISH}

Normally, to cast a spell or summon a spirit, the mage or
shaman must make the \textit{Cast a Spell} or \textit{Conjure} moves. With
a fetish, this is no longer the case: instead, they can simply
declare that they’re using it (making any other moves that
the fiction would dictate of course, for instance, \textit{Stay Frosty}).
Once triggered, the stored spell or spirit is immediately cast
or conjured. The fetish is good for a single use, after which it
crumbles to dust.

\end{multicols}

\section{SPELLS}
\begin{multicols}{2}
Like other equipment, spells (although they’re not exactly 
  “equipment”) are described in terms of tags. Spells have the 
      following special tags: 

\begin{adjustwidth*}{.5cm}{.5cm}

      \textit{Force:} the minimum Force required to 
      cast the spell. When determining the effects of the spell, use the \textbf{Effective Force}, or \textbf{EF}, value which is the \textbf{(Force Cast - Minimum Force) + 1}.

      \textit{Effect:} describes the actual result of a successful casting of 
      the spell. 
\end{adjustwidth*}

\textbf{RANGE TAGS }


\begin{adjustwidth*}{.5cm}{.5cm}

\textit{Touch:} the spellcaster must touch the target to cast the
spell.

\textit{LOS:} the spellcaster must be within line of sight of the target. Technological vision enhancements (aside from old fashioned optics) do not count for line of sight.

\textit{Linked:} the spellcaster must possess an object of high significance to the target, or a fresh (under 24 hours old) bodily sample. With an appropriate link, the spell has a range of \textbf{EF} kilometers.

\end{adjustwidth*}

\textbf{TARGET TAGS}

\begin{adjustwidth*}{.5cm}{.5cm}

\textit{Self:} the spell only affects the caster

\textit{Metahuman:} the spell only affects metahumans

\textit{Creature:} the spell affects any living creature

\textit{Spirit:} the spell affects only spirit beings

\textit{Object:} the spell affects inanimate objects

\textit{Device:} the spell affects technological devices
\end{adjustwidth*}

\textbf{DURATION TAGS}

\begin{adjustwidth*}{.5cm}{.5cm}

\textit{Instant:} the spell occurs very quickly.

\textit{Short:} the spell lasts long enough for the target to take one
move, more or less.

\textit{Triggered:} this spell is triggered by an outside event (for
instance, taking damage)

\textit{Sustained:} the spell remains in effect for a period determined by the caster. Each sustained spell in effect inflicts a stacking -1 to future spellcasting moves to account for the split concentration of the caster.
\end{adjustwidth*}
\end{multicols}

\subsection{COMBAT SPELLS}

\rowcolors{2}{white}{lightgray}
\begin{tabular}{>{\bfseries}m{.1\linewidth}m{.34\linewidth}>{\bfseries\centering}m{.11\linewidth}m{.35\linewidth}}
Spell& \textbf{Description}&Minimum Force&\textbf{Tags}\\\midrule
Mana Bolt & deals 1d4 damage (bypassing armor) to creatures or spirits at LOS & 2 & LOS, target(creatures, spirits), duration:instant,
dmg 1d4, ignores armor, force 2\\
Fire bolt& deals 1d6 damage and fire effects to creatures at LOS range.& 2 & LOS, target(creatures), instant, dmg 1d6, fire, force 2\\
Taser Hands& deals 1d6+EF damage and shock effects to
creatures at touch range& 2 & touch, target(creatures), object, instant, dmg 1d6+EF, shock,
force 2\\
Acid Stream& deals 1d6 damage and acid effects to targets
and objects at LOS range & 2 &
Tags: LOS, acid, target(creature), object, instant, dmg
1d6, force 2\\
Fireball& deals 1d6+EF damage and fire effects to
all creatures and objects in an area within short
range.& 3 & LOS, fire, area, target(creature), instant, dmg
1d6+EF, obvious, force 3\\
Manaball& deals 1d6 damage (bypassing armor) to creatures
and spirits within the target area&4 & LOS, area, target(creatures, spirits), instant, dmg
1d6, ignores armor, force 4 \\
Knockout& deals 1d6 stun (bypassing armor) to creatures
in touch range& 1 & touch, target(creatures), instant, dmg 1d6 stun, ignores armor,
essence 1\\
\bottomrule
\end{tabular}

\subsection{DETECTION SPELLS}
\rowcolors{2}{white}{lightgray}
\begin{tabular}{>{\bfseries}m{.1\linewidth}m{.34\linewidth}>{\bfseries\centering}m{.11\linewidth}m{.35\linewidth}}
Spell& \textbf{Description}&Essence Cost&\textbf{Tags}\\\midrule
Analyze Device& take +1 to your next move involving the
device being analyzed, or learn what the device does&1& touch, analysis, target(device), duration:short, effect(take +1 to a move involving the device),
force 1\\
Clairvoyance& when you Check the Situation, you can ask
questions about a location you cannot see within the range
of the spell& 2 & LOS, perception, target(self), duration:short, area, effect(Check the Situation in
a remote area), force 2\\
Combat Sense& while you sustain this spell, you cannot be
surprised, and take +1 to your first Rock \& Roll or Stay Frosty
move when combat starts& 2(S) & touch,
perception, target(self), duration:sustained, effect(you cannot be surprised and take +1 to your first Rock \& Roll or Stay
Frosty), subtle, force 2\\
Mind Probe& when you touch the target, you can hold 1 toward Manipulate or Make ‘Em Sweat& 2(S) & touch, telepathy, target(metahumans), duration:sustained, effect(hold 1 toward Negotiate or
Push Someone), force 2\\
Detect Life& when you look for living creatures in an area,
take +2& 2& LOS, perception,
target(self), duration:short, effect(take +2 to look for living
creatures with Check the Situation), force 2\\
\bottomrule
\end{tabular}

\subsection{HEALTH SPELLS}
\rowcolors{2}{white}{lightgray}
\begin{tabular}{>{\bfseries}m{.1\linewidth}m{.34\linewidth}>{\bfseries\centering}m{.11\linewidth}m{.35\linewidth}}
Spell& \textbf{Description}&Essence Cost&\textbf{Tags}\\\midrule
Antidote& when you touch the target, you halt poison or
other toxin effects in the target& 1 & touch, cure, self, target(metahumans), effect(halts poisons and
other toxins), force 1\\
Heal& when you touch the target, heal a number of wounds
equal to the EF&1& touch, heal, self, target(metahumans), effect(heal EF wound boxes), exhausting, force 1\\
Increase Attribute& when you touch the target, choose 1
stat. Moves using that stat take +1 while the spell is sustained. &2& touch, enhance, target(self), metahuman, duration:sustained, effect(choose 1 stat; moves using that stat
take +1 while the spell is sustained), exhausting, force 2\\
\bottomrule
\end{tabular}

\subsection{ILLUSION SPELLS}
\rowcolors{2}{white}{lightgray}
\begin{tabular}{>{\bfseries}m{.1\linewidth}m{.34\linewidth}>{\bfseries\centering}m{.11\linewidth}m{.35\linewidth}}
Spell& \textbf{Description}&Essence Cost&\textbf{Tags}\\\midrule
Chaotic World& when you cast this spell, you can hold 1 to
spend on your or your teammate’s moves&2& LOS,
distraction, target(creatures),area, duration:sustained, effect(hold
1 toward your or your teammate’s moves in combat), force 2\\
Group Invisibility& while you sustain this spell, you conceal
a number of creatures equal to the EF from being
seen by creatures or metahumans& 3(S) & LOS, area, concealment, target(metahumans),
duration:sustained, effect(you cannot be seen by creatures or
metahumans), force 3(S)\\
Silence& while you sustain this spell, all sound is silenced
in the area you specify&3(S)& LOS, area, concealment, target(creatures),
duration:sustained, effect(all sound is silenced in the area),
force 3(S)\\
Stink& while you sustain this spell, all creatures in the affected
area have to either leave the area or use air filters or take
1 stun&3(S)&LOS, area, distraction, target(creatures),
duration:short, effect(enemies must flee, use respirators or
filters, or take 1 damage), force 3\\
\bottomrule
\end{tabular}

\subsection{MANIPULATION SPELLS}
\rowcolors{2}{white}{lightgray}
\begin{tabular}{>{\bfseries}m{.1\linewidth}m{.34\linewidth}>{\bfseries\centering}m{.11\linewidth}m{.35\linewidth}}
Spell& \textbf{Description}&Essence Cost&\textbf{Tags}\\\midrule
Mana \mbox{Barrier}& while you sustain this spell, you create a barrier that blocks living creatures and spirits& 3(S) & LOS, protection, target(creatures,spirits), duration:sustained, effect(create a barrier that blocks living creatures
and spirits), force 3\\
Light& while you sustain this spell, an area you specify is illuminated by bright light& 3(S)& LOS, area,
energy, duration:sustained, effect(generates bright illumination in an area; large areas cost more essence),
force 3\\
Shadow& while you sustain this spell, an area you specify
is cloaked in arcane darkness&3(S) &LOS, area, energy, duration:sustained effect(generates arcane
darkness in an area), force 3\\
Fling& when you cast this spell on a target you are touching,
you hurl the target out of melee range& 1& touch, telekinesis, target(creatures), duration:instant, effect(hurl target out of melee range; target takes 1
stun), force 1\\
\bottomrule
\end{tabular}


\section{SPIRITS}
\begin{multicols}{2}
Spirits are the companions and tools of the Shaman, who
summons them from the astral plane to perform services for
him. Spirits have the following special tags:

\begin{adjustwidth*}{.5cm}{.5cm}
\textit{aspect:} the spirit takes on the appearance of their domain,
and is invisible in their domain unless it chooses to be
seen. Elementals automatically gain this tag, otherwise it
requires 1 spirit point.

\textit{desert:} a spirit of the forbidding landscape of
the deserts

\textit{earth:} a spirit who dwells in the earth, caves, or landscape;
earth spirits are widespread

\textit{elemental:} these spirits represent the basic four elements,
air, earth, fire, and water, and can be summoned
anywhere.

\textit{engulf:} the spirit may enclose a target in the ubstance of its
domain, typically (but not always) dealing damage.

\textit{enthrall:} use this stat for the Enthrall move

\textit{forest:} a spirit of the forests, woods, or similar
areas

\textit{generous:} the spirit will perform one extra move; adding
this tag costs 1 spirit point.

\textit{guard:} use this stat for the Guard move

\textit{harm:} use this stat for the Harm move

\textit{insubstantial:} damage dealt and taken is halved

\textit{mentor:} use this stat for the Mentor move

\textit{mountain:} a spirit that dwell in foothills, crags, ridges, and
other mountainous terrain

\textit{natural:} natural spirits are spirits associated
with particular domains (such as “city spirits” or
“mountain spirits”).

\textit{plains:} a spirit of the open plains, grasslands, fields, and
farms

\textit{robust:} the spirit is particularly resistant to damage; all
damage rolls against it are [w]. Adding this tag costs 1
spirit point.

\textit{search:} use this stat for the Search move

\textit{sky:} a spirit of the open sky

\textit{storm:} a spirit of storms and harsh weather

\textit{swamps:} a spirits of the depths of the swamp, bayou, or
wetlands

\textit{urban:} a spirit dwelling in urban or developed
lands, especially cities

\textit{water:} a spirit of lake, river, or ocean 

\textit{weakness (specify):} the spirit has a weakness to a particular
material or element which ignores insubstantiality, armor,
and robustness. Adding this tag allows the free addition of
another tag.

\textit{wild:} this spirit has an extra spirit point, but the shaman
must take -1 when he or she conjures it
\end{adjustwidth*}

\subsection{SPIRIT MOVES}
Spirits are independent entities, and have thier own moves.
Their moves correspond to the harm, search, guard, enthrall,
and mentor tags.

\textbf{HARM:} when a spirit attacks someone or something,
roll+Harm. On 10+, the spirit deals its damage. On 7-9, the
spirit deals damage, but also takes damage.

\textbf{SEARCH:} when the spirit attempts to locate individuals or
items within its domain, roll+Search. On 10+, the spirit locates the item and can tell the Shaman where it is. On 7-9,
the spirit can tell the shaman whether the item or person is
within its domain, but not it’s specific location. Note: the GM
and player should determine the search range for
elementals.

\textbf{GUARD:} when a spirit stands in defense of its domain or
inhabitants thereof, roll+Guard. On 10+, the spirit prevents
damage or hostile effects from occurring. On 7-9, the spirit
halves damage or the potency of a hostile effect.

\textbf{ENTHRALL:} when a spirit attempts to control someone’s
actions or thoughts, roll+Enthrall. If the target
is a:
\begin{adjustwidth*}{.5cm}{.5cm}
\textbf{An NPC:} On a 10+, the spirit issues two instructions
that the NPC must follow, or take 3 damage. On 7-9,
the spirit may issue one instruction.

\textbf{A PC:} On a 10+, both of the following apply. On 7-9,
only 1 applies:
\begin{adjustwidth*}{.5cm}{.5cm}
\tcirc{} If the character complies, they mark XP

\tcirc{} If the character refuses, they must Stay Frosty
\end{adjustwidth*}
\end{adjustwidth*}

\textbf{MENTOR:} when a spirit imparts knowledge or truth, roll+Mentor. On 10+, the GM provides, in secrete, a useful or
interesting piece of information to the target. On 7-9, the GM
provides an interesting piece of information.

\subsection{EXAMPLE SPIRITS}
There are 5 general spirit natures: Watchers simply observe
and report. Teachers seek to instruct and guide others, but
are reluctant to do harm. Protectors seek to defend their domain and its inhabitants, while Destroyers seek battle, blood,
and vengeance. Finally, Seducer spirits desire control and
devotion.

\subsubsection{ELEMENTALS}
\textbf{Fire Elemental} [\textit{destroyer, aspect, harm 2, search -1, guard
1, enthrall 1, mentor 0, dmg 1d10, armor 2, wounds
9}]

\textbf{Water Elemental} [\textit{seducer, aspect, harm -1, search 2, guard
0, enthrall 3, mentor 1, dmg 1d4, armor 1, wounds
8}]

\textbf{Air Elemental} [\textit{teacher, aspect, harm -2, search 2, guard 0,
enthrall 1, mentor 2, dmg 1d4, armor 2, wounds 7}]

\textbf{Earth Elemental} [\textit{protector, aspect, harm 1, search 2, guard
2, enthrall -1, mentor 0, dmg 1d8, armor 1, wounds
10}]

\subsubsection{NATURAL SPIRITS}
\textbf{Forest Protector} [\textit{natural, forest, harm 1, search 1, guard 2,
enthrall -1, mentor 0, dmg 1d8, aspect, armor 1,
wounds 8}]

\textbf{Forest Watcher} [\textit{natural, forest, search 3, guard 0, enthrall
1, mentor 1, aspect, armor 1, wounds 6, special:may not
Harm}]

\textbf{Sky Watcher} [\textit{natural, aspect, search 3, guard 0, enthrall 0,
mentor 2, armor 1, wounds 6, special:may not
Harm}]

\textbf{Urban Destroyer} [\textit{natural, harm 2, search 0, guard 1, enthrall
1, mentor -1, dmg 1d10, armor 2, wounds 9}]

\textbf{Urban Seducer} [\textit{natural, seducer, harm 0, search 2, guard 0,
enthrall 2, mentor 1, dmg 1d4, armor 1, wounds 7}]

\textbf{Mountain Teacher} [\textit{natural, aspect, harm 0, search 0, guard
2, enthrall 0, mentor 2, dmg 1d4, armor 1, wounds
8}]

\textbf{Swamp Destroyer} [\textit{natural, aspect, harm 2, search 2, guard
0, enthrall 0, mentor -1, dmg 1d10, armor 2,
wounds 9}]


\end{multicols}

\newpage

\invisiblepart{GM GUIDELINES}
\section{GAMEMASTER GUIDELINES}
\begin{multicols}{2}
As mentioned in the introduction to this game, I’m assuming some familiarity with Dungeon World on the part of the
reader. Dungeon World provides a list of important rules for
the GM to follow. Here they are (modified for proper cyberpunk-ness, of course):


\subsection{ALWAYS SAY}
\textbf{What the rules demand:} when a move is triggered, yours
or the players, say what the rules tell you to say. Embellish
and expand, but start from the rules.

\textbf{What the adventure demands:} you know things the players don’t, and you know them ahead of time. If the players
haven’t done anything to change them, stick with
‘em.

\textbf{What honesty demands:} always be honest. If the rules tell
you to give out information, do it. No lies, no half-truths. Be
generous, even. And once it’s set in stone, no going back
on it. Also, if the players achieve something, give it to them
fully.

\textbf{What the principles demand:} use your principles and
agenda as a filter or an inspiration. If you get caught short,
review them to make sure you are abiding by them.
\subsection{YOUR AGENDA}
\textbf{Make the world fantastic:} barf forth cyberpunk! Scenes,
smells, sounds - the glittering height of an arcology, the
stench of a slum hellhole, the scream of turbofans as a GEV
heads toward you, the rrrrrrrrip of a minigun tearing through
your cover - it’s your job!

\textbf{Fill the characters’ lives with adventure:} make the world
they live in exciting, dangerous, full, and epic.

\textbf{Play to find out what happens:} NO. PLOTS. Ideas, yes.
Fronts, sure. But do not come to the table with a story already written in your head, because for sure, the players will
not go where you expect.
\subsection{YOUR PRINCIPLES}
\textbf{Draw Maps, Leave Blanks:} make use of maps, but don’t fill
it all in. Leave holes for imagination.

\textbf{Address the characters, not the players:} never talk to the
players in the fiction. They don’t live in the
\SW{}.

\textbf{Embrace the exotic and fantastic:} the world is a crazy
mesh of man, magic, and machine. Make it breathe.

\textbf{Make a move that follows:} when you make a move, you
are participating in the fiction. The move should follow from
the fiction logically.

\textbf{Never speak the name of your move:} moves aren’t things
in \SW{}. Moves are shorthand for you. Never say the
name of your move.

\textbf{Give every creature life:} monsters and creatures exist and
are real. Give them smells, sounds, personality.
Name every person: everyone has a name. Make sure you
give it to them!

\textbf{Ask questions, and use the answers:} the easiest question
is “What do you do?” Whenever you make a move, end
with “What do you do?” And don’t forget to take opportunities to keep the focus moving from character to
character.

\textbf{Be a fan of the characters:} you are not here to beat them;
this is not a contest. You should cheer their successes,
lament their failures, and mourn their passing.

\textbf{Think with the Front Sight:} nothing in the world you create
for the characters is sacred. Every time you put something
or someone onscreen, think about how destroying them
might affect the story.

\textbf{Begin and end with the fiction:} to do it, do it. Everything
stems from, and leads back to, the conversation you’re having. Transition from fiction to rules and back to
fiction.

\textbf{Think offscreen, too:} make your move elsewhere, and
show the effects to the characters later.

\subsubsection{GM MOVES}
The GM has moves of his or her own to use. Although they’re
given formal names, they’re really just the same things GMs
have always done. For example, “revealing an unwelcome
fact” isn’t an esoteric trick to learn—it could be as simple as
saying “that datastore you just cracked? Yeah, it was really a
honeypot, and security hackers are closing in.”

These moves, just like the players’ moves, stem from, and
return to, the fiction of the game. Let them flow!

\textbf{BASIC MOVES}
\begin{adjustwidth*}{.5cm}{.5cm}
\setlength{\parskip}{.1em}\itshape
Use an NPC, creature, danger, or location move

Reveal an unwelcome fact

Show signs of danger

Deal damage

Use up their resources

Turn their move back on them

Separate them

Give an opportunity to showcase an archetype

Show a downside to their archetype, race, or
equipment

Offer an opportunity - with or without cost

Put someone in a spot

Tell them the requirements and consequences, and
ask
\end{adjustwidth*}

\textbf{LOCATION MOVES}
\begin{adjustwidth*}{.5cm}{.5cm}
\setlength{\parskip}{.1em}
\itshape
Change the environment

Point to a looming threat

Introduce a new faction

Use a threat from an existing faction

Make them backtrack

Present riches at a price

Present a challenge to one character
\end{adjustwidth*}

\end{multicols}


\section{THREATS}
\begin{multicols}{2}
\textbf{Threats} is the general term for the opposition - creatures, 
other runners, security guards, and so forth — that a team of 
  runners might encounter in their adventures. Threats come in 
  many shapes and sizes, and only a few examples are given 
  here, but you can use these examples to expand on the list of 
  threats, and invent your own (you can even use the Monster 
  Creator at http://codex.dungeon-world.com/). 

\subsection{THREATS AND DICE}
If you’re the GM, you should be aware that unlike many
games, \textbf{you never roll dice to make moves} (though you will
roll dice for Threat damage from time to time).
Threats have moves, both the GM moves listed earlier, and
sometimes their own special moves, but you won’t see any
“roll+Stat” instructions here. Threat Moves happen in response to, and flow from the fiction. If something is done by
a player character that would lead to a Threat move, then it
happens. If the player didn’t fail their move, then it’s likely
that what you’ll do is a \textbf{soft move}: show them some danger
coming, make something happen that will trigger a move on
their part, and so forth.
On the other hand, if the player gives you a golden opportunity, usually by completely failing a move, then you can
make a \textbf{hard move}. An easy example of this is in the case of
doing damage. If a PC Rocks \& Rolls with a threat, and fails
(rolls a 6 or less), then in return, that Threat deals its damage
to the player right away. That’s the default outcome for failing
a Rock \& Roll move.
Keep in mind, however, that you only have to make \textbf{as hard
a move \textit{as you like}}. It doesn’t always have to be the ultimate
sanction — sometimes, you might make a soft move to increase the tension of a situation. You don’t have to deal that
damage, if making a different move would be more fun!

\subsubsection{THREAT DAMAGE}
Threats, in general, deal the damage indicated in their entry
whenever they deal their damage. However, sometimes multiple threats mob a single player character and inflict damage
on the PC. In such cases, they do not all deal their damage.
Instead, deal damage for the most dangerous threat, and add
+1 damage for each additional threat involved in
the attack.

\textbf{Example:} \textit{Valentin is facing down a ghoul and four goblins, who all assaulted him more or less simultaneously. He
attempted to dodge away, but failed. Instead of dealing
2d6b for the ghoul, and then rolling 2d4b four more times
(once for each goblin), you would roll 2d6b for the ghoul,
and add an additional 4 damage (+1 for each
goblin).}

\subsubsection{OPTIONAL: INFLICTING CHRONIC INJURY}

If it suits the group, you can allow a threat to inflict chronic injuries (see page 10) if that threat’s damage pushes a character into the bleeding out stage. If so, choose an appropriate
chronic injury from the list. For example, if a ghoul manages
to take a character to the bleeding out stage with a bite, you
can inform the character that unless they stabilize, they will
take the Faded chronic injury, and reduce their Essence by 1.

\subsubsection{THREAT WOUNDS}
Threats make no distinction between stun and wounds for
threats. If you deal stun to a threat, unless it is listed as immune to stun, simply mark the damage on the wound track.

\subsection{THREAT TRAITS}
The traits that follow are primarily intended to help the GM
describe creatures, figure out what a creature might do, set
scenes, and enhance the story. For example, when using a
threat with the Camouflage trait, the GM might leverage that
trait to describe how the threat materializes out of nowhere,
having been hidden against a wall or some other innocuous
place until the PC’s were in just the right spot.

\textit{Amphibious:} threat is at home in water
and on land

\textit{Arcane:} threat is Awakened

\textit{Aspect:} threat shows traits of its domain
or environment

\textit{Bloodthirsty:} the threat will continue to
attack incapacitated opponents

\textit{Camouflage:} threat is difficult to detect and can blend in
with its environment

\textit{Cyber:} this threat is enhanced with cyberware, which increases its performance in some fashion

\textit{Deathwish:} the threat lacks any sense of self-preservation;
this can manifest in relentless attacks, or simple stupidity,
depending on the threat

\textit{Dual Natured:} threat is visible and active both in Astral
Space and in the physical world. Abbreviated dn.

\textit{Fast:} the threat is exceptionally quick

\textit{Fear:} the threat inspires fear or causes
a fear effect

\textit{Fearless:} the threat will often continue
fighting to the death

\textit{Group:} usually seen in groups of 3-6
individuals

\textit{Hoarder:} the threat collects...something. Sometimes good
things, sometimes horrifying things.

\textit{Horde:} threat is typically found in large
groups

\textit{Huge:} colossal, several times larger than
a human

\textit{Immune (type):} threat is immune to a particular type of
damage, for example immune (stun)

\textit{Infected:} threat carries a disease that can be contracted by
the characters

\textit{Insubstantial:} threat takes half damage

\textit{Intelligent:} threat is smart enough to think and plan; most
metahuman threats are intelligent

\textit{Large:} much larger than a human

\textit{Machine:} threat is mechanical in origin

\textit{Medium:} roughly human size

\textit{Movement:} threat has a special movement
mode

\textit{Night Vision:} threat can see in dark environments without
trouble

\textit{Organized:} threat has an organizational structure, and may
have additional allies upon which to call

\textit{Paranormal:} threat is of paranormal origins

\textit{Poison:} threat poison its targets; victims take 1 damage
each time they make a move, until they receive treatment
of some sort)

\textit{Program:} threat is a Matrix program (such as IC)

\textit{Range:} these are the same as the ranges in the equipment
section

\textit{Small:} smaller than a human

\textit{Spirit:} attacking this threat uses the Battle the Arcane move

\textit{Solitary:} usually seen alone

\textit{Stealthy:} threat is naturally difficult to detect

\textit{Summoned:} this is a spirit being, and can be banished

\textit{Tiny:} much smaller than a human

\subsubsection{TAG NOTES}
All paracritters are assumed to have the paranormal tag.

All Intrusion Countermeasures are assumed to have the
fearless and program tags.

Creatures may or may not fight to the death. Many
metahumans will not, since most of them still have
some sense of self preservation. The fearless tag indicates a much greater likelihood of fighting to the death
even without a reason.

\end{multicols}

\section{PARACRITTERS}
\begin{multicols}{2}
All paracritters have the paranormal tag.


\critterspec
{AFANC}
{amphibious, camouflage, group, large}
{Bite (2d6b dmg, c), tail whip (1d6+1, reach)}
{10 Wounds / 2 Armor}
{The Afanc is an awakened crocodile, typically found in Wales
and Eastern Europe. They exist in family groups of 3-6 individuals, and are highly territorial. They have an exceptional
ability to detect nearby prey.}
{to eat}
{\tcirc{} Detect nearby prey

\tcirc{} Death roll
}

\critterspec
{BARGHEST}
{fast, medium, fear, group}
{Bite (1d6+2 dmg, c), howl (2d8b stun, area, c/s/m)}
{6 Wounds / 1 Armor}
{The barghest is an awakened canine found in North America,
Europe, and Asia. A massive mastiff-like creature, the barghest is best known for its unearthly, paralyzing howl which it
uses to freeze its prey in its tracks.}
{to hunt}
{
\tcirc{} stalk the prey
}

\critterspec
{COCKATRICE}
{dual-natured, hoarder, small, solitary}
{Paralytic tail (2d6b+2 stun, c)}
{4 Wounds / 0 Armor}
{The cockatrice resembles an overgrown, semi-reptilian chicken. It is known best for the paralysis a touch of its long tail can induce in a metahuman. It’s also known for its tendency to collect small items -- jewelry, etc. }
{protect its territory.}
{\tcirc{} turn flesh to stone

\tcirc{} collect the shinies}

\critterspec
{BLACK ANNIS}
{fast, fearless, medium, night vision}
{Slam (1d6 dmg, forceful, c), bite (1d8 damage, c)}
{6 Wounds / 1 Armor}
{The Black Annis is an awakened baboon, highly territorial and vicious. Studies also indicate that the Black Annis is capable of creating an overwhelming sense of depression in metahumans, though this has not been confirmed. }
{to dominate.}
{\tcirc{} tear intruders apart

\tcirc{} show a threat display}


\critterspec
{DEATHRATTLE}
{camouflage, medium, poison, solitary}
{Bite (2d6b, poison, c), spit venom (1d8, s)}
{5 wounds / 0 armor}
{The deathrattle is a large awakened rattlesnake, found across North America. The deathrattle has a potent toxin which operates on both a physical and astral basis. It is very difficult to cure, requiring the attentions of both medical professionals and magical expertise.}
{to eat.}
{\tcirc{} strike from hiding

\tcirc{} shake the rattle}

\critterspec
{DEVIL RAT	}
{disease, horde, small}
{Gnaw (1d6 damage, messy, 1AP, c)}
{4 wounds / 0 armor}
{The devil rat is a giant, hairless, loathsome creature found in sewers and sprawls around the world. Devil rats are somewhat dangerous alone, but when they swarm, they can cause catastrophic damage. Stories about mass disappearances in some of the worst slums are sometimes attributed to devil rat swarms.}
{to devour.}
{\tcirc{} swarm of teeth

\tcirc{} avoid the light}

\critterspec
{DRAGON	}
{arcane, dual-nature, huge, hoarder, intelligent}
{Bite (2d10b dmg, 4AP, c), fire breath (2d6 dmg, s/m)}
{30 wounds, 6 to 8 armor}
{Never cut a deal with a dragon. Extremely intelligent and powerful, these creatures have become heads of megacorps, and one was even the President of the UCAS before he was assassinated. They come in many varieties, including western, eastern, feathered and leviathian. Their ultimate purpose is unknown, but whatever it is, they seem to be doing it well.}
{to be the ultimate.}
{\tcirc{} Get rid of opposition

\tcirc{} Scheme from the shadows

\tcirc{} Unleash its wrath}

\critterspec
{GREATER WOLVERINE	}
{bloodthirsty, fearless, large, solitary}
{Bite (1d8 dmg, messy, c), claw (1d6+1 dmg, messy, c)}
{10 wounds / 2 armor}
{The greater wolverine is a massive engine of destruction, with a mean streak a mile wide. }
{to kill.}
{\tcirc{} Abuse the dead

\tcirc{} Eat to excess}

\end{multicols}

\newpage

\section{METAHUMANS}
\begin{multicols}{2}

\critterspec
{CORPORATE SECURITY	 }
{group, intelligent, medium }
{Sidearm (1d8 dmg, 1AP, s/m), stun baton (1d6 stun, c) }
{8 Wounds / 0 Armor }
{This is the run of the mill corporate security guard. Dangerous in groups, and corporations generally have a near-infinite supply.}
{to guard their station. }
{\tcirc{} Call for backup 

\tcirc{} Trigger the alarm }

\critterspec
{ELITE SECURITY	}
{group, cyber, intelligent, medium }
{SMG (2d6b dmg s/m), Hand-to-Hand (1d6+1 dmg, c) }
{8 Wounds / 2 Armor }
{Although not every facility has an elite security contingent protecting it, when you start running the bigger corporations, you may run into these guys. With better training and better gear than your typical security guard, Elite Security is called in when the regular security grunts run into more than they can handle. }
{secure the facility. }
{\tcirc{} Neutralize targets 

\tcirc{} Strike from ambush }

\critterspec
{BEAT COP }
{medium, intelligent, solitary }
{Sidearm (1d8 dmg, 1AP, s/m), baton (1d6 dmg c) }
{8 Wounds / 1 Armor }
{Even in the seemingly lawless 2050s, there are still people out there who serve in the thin blue line, walking a beat and enforcing the law. Whether a member of Knight Errant, Pinkerton, or Lone Star, the beat cop is the most commonly seen law enforcement officer on the streets. }
{to protect and serve. }
{\tcirc{} make an arrest 

\tcirc{} call backup}

\critterspec
{LONE STAR HTR		}
{cyber, medium, intelligent, group}
{Assault Rifle (2d8b dmg, 2AP, s/m/l)}
{8 Wounds / 3 Armor}
{Hostage situations, major crimes, killing sprees, you name it — when a serious crime goes down, the High Threat Response teams are called in. Highly trained, well-equipped, and thoroughly professional, tangling with HTR is no joke.}
{terminate the threat.}
{\tcirc{} Breach, bang and clear

\tcirc{} Take the shot}

\critterspec
{BLOOD MAGE	}
{arcane, medium, solitary}
{Blood bolt (1d8 dmg s/m), death touch (2d4b, ignores armor, c)}
{8 Wounds / 1 Armor}
{Blood magic — the use of blood (usually not your own) to fuel magical spells and rituals — is illegal almost everywhere in the \SW{}. However, that doesn’t stop people from using it.}
{to gather power.}
{  \tcirc{} Inflict bleeding wounds}

\critterspec
{CYBERZOMBIE	}
{dual-natured, medium, intelligent, cyber}
{Arm Cannon (2d6b dmg, 2AP, s/m/l), arm blade (1d6 dmg, c)}
{15 wounds / 3 armor}
{The cyberzombie is an unfortunate soul, a cyborg who has pushed himself too far with cybernetics and died. A cybermancer has managed to reconnect his soul to the body, and now the creature lives a tortured life. }
{to pass on.}
{\tcirc{} Destroy for the creator

\tcirc{} Find a way to end the suffering}


\critterspec
{COMBAT MAGE	}
{arcane, cautious, medium, solitary}
{Manabolt (1d6+1 dmg, s/m), flamethrower (1d6+1 dmg, burn, s/m), confusion (targets take -2, s)}
{8 Wounds / 2 Armor}
{The Awakened are statistically rare in the \SW{}, but shadowrunners tend to deal with them considerably more frequently than your average wageslave. One of the more feared foes on the battleground is the Combat Mage, a mage who has devoted his abilities to deadly combat magic.} 
{to see who’s best.}
{\tcirc{} Display their power

\tcirc{} Burn everything}

\critterspec
{SECURITY HACKER	}
{cyber, intelligent, medium, solitary}
{Black hammer (2d6b dmg, c), blackout (1d6+1 dmg, stun c), slow (-1 forward, c)}
{8 Wounds / 2 Armor (matrix only)}
{Any corporation worth its salt employs security hackers to protect its precious data. A corporate hacker is often equipped with excellent gear and has the benefit of being able to navigate a corporate grid easily, since they belong there.}
{to track ‘em and smack ‘em.}
{\tcirc{} Initiate a trace

\tcirc{} Deploy IC}

\critterspec
{STREET THUG}
{group, intelligent, medium}
{Spiked bat (1d6+1 dmg, c), cheap but powerful pistol (2d8w dmg, s/m)}
{9 Wounds / 1 Armor}
{Gangs plague the sprawls, and turf is everything. During a shadowrun, it’s often a good idea to know whose turf you’re on, who the leaders are, and what kind of crime they’re into. If you run afoul of a gang, you might run into someone like the Street Thug.}
{to guard their turf.}
{\tcirc{} Issue a beatdown

\tcirc{} Gather the crew
}

\critterspec
{GHOUL}
{blind, group, infected, intelligent, medium}
{Bite (2d6b dmg, disease, c), talons (1d6 dmg, 1AP, c)}
{6 Wounds / 0 Armor}
{Ghouls are humans infected with HMHVV, which has modified their genetics such that they have an insatiable hunger for human flesh. Intelligent, and often found in packs in sewers, back alleys, and the squats and slums of the \SW{}. Despite their physical blindness, they can be a dangerous enemy indeed.}
{to feed the hunger.}
{\tcirc{} consume essence}

\critterspec
{GOBLIN	}
{horde, infected, small}
{Claw (1d4+1 dmg, c), knife (1d6 dmg, c)}
{4 Wounds / 1 Armor}
{Goblins are the result of a dwarf being infected with HMHVV, resulting in a small, twisted, nocturnal creature that tends to run in large packs. Stumbling across a goblin colony can really ruin your day.}
{to scavenge and collect.}
{\tcirc{} ambush}

\end{multicols}

\newpage
\section{INTRUSION COUNTERMEASURES}
\begin{multicols}{2}
Intrusion countermeasures all possess the fearless and program tags. Use these threats in conjunction with matrix nodes
and armored nodes (see page 33).

\critterspec
{ACID}
{}
{Burnout (reduces hardening by 1), chip burn (reduce CPU by1)}
{4 Wounds / 0 Armor}
{Acid is a version of IC designed to damage cyberdecks, opening holes for other more dangerous IC to use to make the attack.}
{burn through defenses.}
{}

\critterspec
{BLASTER}
{}
{Jolt (1d6 dmg, stun)}
{4 Wounds / 1 Armor}
{Blaster IC is designed to inflict nonlethal damage on a hacker, hopefully knocking him or her out and forcing them to disconnect from the grid. Blaster is fairly common, since it is nonlethal, and can be found even in generally lower-security systems. }
{to knock ‘em out.}
{}

\critterspec
{BLACK IC}
{Intelligent, organized}
{Lethal biofeedback (2d8b dmg)}
{6 Wounds / 2 Armor}
{Black IC is the most feared of all intrusion countermeasures. Used by high-security installations, Black IC is designed for one purpose: to kill intruding hackers. Capable of delivering a lethal burst of biofeedback, the victim of a black IC attack is usually found dead in their rig, bleeding from eyes, ears, nose, and mouth. Black IC is not to be trifled with.}
{to kill.}
{\tcirc{} Finish them off}


\critterspec
{CRASH}
{}
{Segfault (crash one program in your deck)}
{3 wounds / 1 Armor}
{A simple countermeasure designed to shut down unauthorized programs, crash is designed to do one thing: corrupt a running program until it crashes. }
{to mess things up.}
{}

\critterspec
{BINDER}
{camouflage}
{Overload (reduce CPU by 1)}
{4 Wounds / 0 Armor}
{Binder is another simple countermeasure, designed to place extra processing load on a cyberdeck’s CPU to decrease its efficiency. }
{to slow down the intruder.}
{}

\end{multicols}

\newpage
\section{SPIRITS}
\begin{multicols}{2}
\textbf{Note:} given the wide array of spirits and their specific manifestations, the GM is encouraged to tweak these entries as
needed!

\critterspec
{SPIRIT OF MAN	}
{aspect, medium, spirit}
{confusion (targets take -2 forward, s), slam (2d6b dmg, forceful)}
{1 armor / 5 Wounds}
{Spirits of Man include spirits of street, hearth, and field, domains intimately linked to the activities of humankind. Known more for their desire to guard and protect an area rather than their innate hostility, they are nonetheless dangerous when their ire is provoked.}
{to guard what man has made.}
{\tcirc{} prevent threats from entering

\tcirc{} cause an accident}

\critterspec
{SPIRIT OF EARTH	}
{aspect, spirit variable size}
{hurl rock (1d8 dmg, forceful), punch (2d6b dmg, forceful)}
{4 Armor / 7 Wounds}
{Spirits of Earth dwell in the very soil and mountain and rock on which life takes root. They usually manifest as beings of rock and dirt, their aspects making them tough to injure. Their powers vary, but as all natural spirits they are motivated to guard their domain.}
{to protect the land.}
{\tcirc{} engulf an intruder

\tcirc{} surge up from the ground}

\critterspec
{SPIRIT OF AIR	}
{aspect, spirit, small, medium}
{fling (1d6+1 dmg, forceful, c), noxious cloud (1d6 dmg, area, poison)}
{3 Armor / 6 Wounds}
{Spirits of Air are capricious beings who dwell in the domain of air. They manifest as howling winds, cold gusts, and vaguely humanoid clouds. Their insubstantial nature makes injuring them difficult.}
{to trick.}
{\tcirc{} move at blinding speed

\tcirc{} toy with an enemy}

\critterspec
{SPIRIT OF WATER	}
{aspect, spirit, small, medium}
{slam (2d8b dmg, c)}
{2 Armor / 7 Wounds}
{Spirits of Water are methodical and inexorable, and take pride that the world will eventually return to the water whence it came. They can be summoned anywhere there is a body of water or river, and they are powerful enemies indeed.}
{to flow}
{\tcirc{} drown the threat

\tcirc{} flow through and around}

\critterspec
{INSECT SPIRIT	}
{aspect, spirit, small/medium/large}
{bite (1d8 dmg, poison, c), strike (2d6b dmg, c)}
{3 Armor / 6 Wounds}
{Insect Spirits are summoned by Insect Shamans, who must “invest” a living host with the spirit (since it lacks the capability to materialize). This process is generally done to involuntary hosts, and the results are horrific. Insect Shamans and Insect Spirits are never something to willingly “get to know.”}
{to breed.}
{\tcirc{} summon the swarm

\tcirc{} scuttle just out of sight}

\critterspec
{TOXIC SPIRIT	}
{aspect, spirit, small/medium/large}
{throw toxin (2d6b, poison, c), poison punch (1d6+1 dmg, poison, c)}
{2 Armor / 10 Wounds}
{Toxic spirits are summoned by toxic shamans from domains that have been corrupted by pollution and other manmade evils. These spirits are as twisted as the domains from which they come.}
{to pollute.}
{\tcirc{} corrupt the environment

\tcirc{} leave their mark}


\end{multicols}

\newpage

\invisiblepart{SPRAWLS}
\section{SPRAWLS}
\begin{multicols}{2}
You could look at shadowrunning as a series of discrete missions, episodes in an ongoing story of quasilegal adventuring.
Ideally, however, the story you weave when you play and/
or GM this game will take place in a world that feels like it’s
alive and breathing, full of real people with realistic motivations, and happening in a place with its own character and
appropriately cyberpunk feel.

Obviously, your adventures have to happen somewhere,
and in the Awakened world of the 2050’s, most of the time
“somewhere” is one of the vast urban regions that grew up
around the cities of the early 2000: the \textbf{Sprawl}.

Whether through urban growth, massive construction projects
by the megacorporations, mergers, or political realignmen,
many cities have grown so large that they a single coherent
“city plan” is laughable. Because of this, the environments
within a single city are wildly varied: you can go from glittering financial sector to funky entertainment districts to rumbling industrial zones to blasted near-wastelands of poverty
and deprivation from the comfort of mass transit.

Some things don’t change, though. Every sprawl has it’s own
character, it’s own particular vibe. There are always factions
fighting for something, always people looking for an edge.
People like to have influence, and they’ll use the tools at their
disposal to get it. And frequently, you will be one of those
tools.

\subsection{CREATING A SPRAWL}
In \SW{}, we use a system quite similar to creating a
Front in Dungeon World to characterize a Sprawl. Of course,
since Shadowrun takes place in a future version of our own
world, you’re welcome to use this system to decide how a
real-world city (for instance, oh, let’s say Seattle). However,
nothing is stopping you from making one, if you want to
place a new city in the world. You’re in control!

The big difference between Dungeon World Fronts and Sixth
World Sprawls is that Sprawls have the added element of
geography and locale. A Sprawl helps the GM keep track of
both individual forces at work in the world (as with a Front),
but also lets the GM and group define the broad conflicts that
exist over a particular location.

The basic process for creating a Sprawl is as follows (each step
will be explained in more detail):
\begin{adjustwidth*}{.5cm}{.5cm}
1.	Allocate 5 points among the three main Influences:
\textbf{Man}, \textbf{Magic}, and \textbf{Machine}.

2.	For each point assigned to an Influence, pick a Peril (you
can pick the same Peril twice).

3.	For each Peril, choose a Crisis, and describe how it will
manifest.
\end{adjustwidth*}

\subsection{INFLUENCES}
\textbf{Influences} are the broad forces acting on a city, which exist
in a constantly shifting equilibrium. There are three influences:

\textbf{Man} is the influence of humanity and its organizations. In
this sense, man represents the influence of people and the
organizations they run on the city: corporations, criminals,
politicians (but I repeat myself), syndicates, religions, celebrities, and so forth.

\textbf{Magic} is the influence of the Awakened and the Astral upon
a city. Often this is tied to the astral beings that populate the
land on which the city stands, but it also includes the desires
and activities of the magically active beings who dwell there
(or who might wish to): mages, dragons, spirits, even paranormal creatures may all exercise their influence on the city.

\textbf{Machine} is the influence of technology, the Matrix, and the
reality of human augmentation. In this modern world, machines and technology are a powerful an influence on the way
people think and feel.

\subsubsection{ALLOCATING INFLUENCE}
The first step of the City Creation process is to allocate influence. The GM should allocate 5 points among the three
Influences, representing the balance or relative weight of that
Influence on the Sprawl in general.
\begin{adjustwidth*}{.5cm}{.5cm}
\textbf{Example:} \textit{Tanner is creating a Sprawl for Buffalo, NY. He
chooses to allocate 3 to Man and 1 each to Magic and Machine. Buffalo, right now, is the prize in a struggle between
organized crime and megacorporations, while magic and
machine have a subtler influence.}
\end{adjustwidth*}
\subsection{PERILS}
Each Influence on a city is characterized by one or more Perils: the specific entities, organizations, and creatures that embody the influence in question. Perils vary widely, and are
selected by the group as the city is being created. Creating a
Peril is as simple as one group member suggesting it. Several
categories of perils are presented below, as inspiration.
Choose one peril for each point assigned to an influence (so a
city with Magic 2 would need 2 perils associated with Magic).
You can assign multiple points to the same Peril, representing
competing interests from the same category of danger.
Example: Tanner’s Buffalo Sprawl is coming along. The
next step is identifying Perils for each Influence area. For
Man’s influence, he needs to assign 3 points to perils of
Man. He assigns one to Megacorporation once and two to
Syndicate (he’s thinking about a mob war brewing).

\subsubsection{PERILS OF MAN}
\textbf{Megacorporations} \textit{(impulse: to boost the bottom line)}

Be it one of the Big 10 megacorps, or some poor little rank A,
all corporations need as much help as they can get. What that
help is may be sketchy, but you have no problem with that.

\textbf{Leagues} \textit{(impulse: to influence you)}

Leagues are groups of people with political agendas, be they
either good or misplaced. Policlubs, local governments, merc
squads, terrorist cells, religions, shadow groups, presidents
and more are trying to spread their own version of reality.
Sometimes quietly, other times with a bang.

\textbf{Syndicates} \textit{(impulse: to control the streets)}

As long as there has been crime, someone has tried to organize it. From street gangs to the Triads, the Yakuza, and the
Mafia, organized and not-so-organized crime eyes the sprawl
with hungry and calculating eyes.

\subsubsection{PERILS OF MAGIC}

\textbf{Energies} \textit{(impulse: to empower)}

We pretend that magic is a science to be studied in the halls
of academia, but the wild and unpredictable power of the Astral and Metaplanes, power sites, ley lines, mana surges and
mana storms make a mockery of our learning.

\textbf{Orders} \textit{(impulse: to achieve eldritch ends)}

Orders are those groups of people with a strong interest in
magic. They can range from noble universities and research
organizations to fanatical cults of dark magic. Be it Atlantean
artifacts to Blood Magic, they want to push, discover and
convert.

\textbf{Awakened} \textit{(impulse: to survive and thrive)}

Not all people affected by the Awakening are metahumans.
In fact, most aren’t. There’s a whole world out there of paracritters, free spirits, dragons and metasapients such as centaurs. Some are in power, some want to be in power, and
some simply want to survive.

\subsubsection{PERILS OF MACHINE}

\textbf{Matrix} \textit{(impulse: to absorb and accumulate)}

The Matrix is just a network of 0’s and 1’s...right? Not if you
ask a Hacker. The Matrix is a living, breathing, evolving entity
that we’ve come to take for granted. But in its unvisited or
forgotten corners and gleaming graphical citadels, what feeds
on the information we produce?

\textbf{Technology} \textit{(impulse: to connect and isolate)}

From ubiquitous surveillance, tailored marketing, and betterthan-life virtual reality to orbital space stations, underwater
compounds, and teeming arcologies, it’s hard sometimes to
tell whether we’re using technology, or it’s using us.

\textbf{Advancement} \textit{(impulse: to relentlessly improve)}

New cyberware, robotics, AI, cloning and more are all coming down the pipeline. Some people are afraid that metahumanity is starting to evolve past its tipping point. Some think
it’s already happened. Whatever the case, it pays to be wary.

\subsection{CRISIS}
\textbf{Crisis} is what happens when a particular Peril accomplishes
its primary aims (which are, obviously, determined by the
GM). Left unchecked, a Peril will always progress toward its
goal—the world lives and breathes, and things happen even
when the player characters aren’t around to witness them.
The progress a Peril makes toward its goals is tracked on the
\textbf{Doom Bar} (more on that later), and when it reaches the end,
whatever Crisis was selected for the Peril goes into effect.
There are five main Crises; when you come up with a Peril,
you must also decide on a Crisis for it, and specify the exact
form it will take.
\begin{adjustwidth*}{.5cm}{.5cm}

\textbf{Control:} insidious influence, strings being pulled, and puppets dancing to the puppetmaster

\textbf{Destruction:} disaster and mass death befall the city

\textbf{Havoc:} the breakdown of order, law, and control

\textbf{Conquer:} unopposed power, and the freedom to enact
any agenda

\textbf{Corruption:} a blight of some sort—crime, graft, or something dark and unnatural—spreads through the Sprawl
\end{adjustwidth*}
\subsection{DOOM BAR}
At the end of this document is a reference sheet to help you
record notes about your Sprawl. You’ll note on the Sprawl
Sheet that the section for each Peril has five boxes next to it.
These bars are known as the \textbf{Doom Bar}.

The Doom Bar represents how close the Peril is to fulfilling its
desire. At 1 box, they are in the initial phases of construction
and planning, while at 5 they are moments away from unleashing their plan.

At the start of a campaign, every Doom Bar starts at 1. A GM
then has 3 points to divide between the Perils to modify the
initial state of their Doom Bars.

As the campaign progresses, the action (and inaction) of the
player characters will influence changes in a Peril’s Doom Bar.
For example, blowing a run, helping an enemy accidently, or
not stopping some plan in time are likely to increase a Peril’s
Doom Bar.

When the runners can’t stop a Peril, or when the DM deems
it appropriate, you mark a Doom Box under the appropriate
Peril. During the next adventure, the DM should state as a
side-bar what the results of the increased Doom are.
\begin{adjustwidth*}{.5cm}{.5cm}

\textbf{For example:}\textit{ Two weeks ago, the team barely escaped
a botched run on a corporate arcology that is performing
strange and dangerous experiments on its citizens without
their knowledge. The failed run caused the corporation to
raise security and step up their project’s timeline, dooming
the citizens now trapped inside.}
\end{adjustwidth*}
The GM could even choose to increase the Doom on multiple
Perils if it makes sense.

\subsubsection{THE END OF THE DOOM BAR}
If a Peril has 5 boxes, and the GM goes to mark another one,
it’s too late: the Peril has accomplished what they were trying
to do, and their Crisis goes into effect. This could have major
impacts on both the Sprawl and the world.

\subsubsection{REDUCING DOOM BAR}
Runners can, believe it or not, reduce the Doom Bar for a
Peril. If they do something that hampers the Peril, the GM
should erase one Doom Box. If the runners do something
really significant to strike a blow to the Peril, such as blowing
up a Renraku datacenter, the GM reduces the Doom Bar by
two boxes.

A minor setback won’t reduce the Doom, but it will prevent
it from increasing.

If runners ever reduce a Peril’s Doom Bar to 0, the Peril goes
into \textbf{remission}. Remission means the Peril may be gone, or
perhaps it’s just licking its wounds. Either way, a Peril in remission does not show up for 2 adventures. Once that time
is over, the GM can either bring back the Peril at 1 Doom, or
bring in a totally new Peril. If a Peril is ever redudced to 0, it
is a good idea to give the players a free advance to award
them for their skill.
\begin{adjustwidth*}{.5cm}{.5cm}

\textbf{Example:} \textit{the team pulled off a run that culminated in blowing up the Renraku datacenter mentioned earlier. Renraku
had been slowly subsidizing Matrix usage, trying to cut the
Sprawl off from the main Matrix grids (and thereby achieve
Control). That Peril stood at 2 Doom before the run, but
the GM decides to remove both Doom boxes—reducing
the Doom to 0— due to the success of the run. Renraku
decides to back off the Matrix control plan.}
\end{adjustwidth*}
However, two sessions later, the team gets word of Renraku
performing some sketchy genetic experiments on Awakened
rats. Looks like Renraku’s back with a new plan.

\subsection{SPRAWL DISTRICTS}
Sprawls are a way to get an idea of the large influences at
work in a particular area, giving you an idea of whch entities
are the movers and shakers of a given city.

\textbf{Districts}, on the other hand, are areas within a Sprawl where
a runner might find him- or herself. Districts are a shorthand
way to record basic descriptive information about different
neighborhoods, areas, and communities within a Sprawl.

The word “district” should be interpreted broadly—a small
neighborhood, a glittering financial sector full of high-rise
buildings, and a sprawling industrial zone can all be Districts.

\subsubsection{CREATING A DISTRICT}
A District is described by tags (like equipment and threats),
which provide some descriptive information to help players
and the GM get a handle on an important area.

Creating a district is very simple:
\begin{adjustwidth*}{.5cm}{.5cm}

1.	Name the District

2.	Determine the core tags of the district (type, economy,
population, and trust)

3.	Determine any other special tags the district may have.

\textbf{Example:} \textit{the GM wants to create an industrial area for
some of the action of this latest run to happen in. He pictures an oil refinery area, full of containers, pits, fences, low
warehouse buildings, tall processing plants, and pipelines
of all sizes crisscrossing the district. Economically, it’s active, though not exactly a “glittering rich” place. It’s isolated due to the industry, and polluted with leavings. It’s
also owned by Ares. The tags for this district are} industrial,
average, stable, cooperative, corporate, polluted, isolated.
\end{adjustwidth*}
\subsubsection{DISTRICT TAGS}
There are four basic or core tags that describe a district, which
are, in order, Type, Economy, Population, and Trust.

\textbf{Type} identifies the general type of district, what kind of things
happen there, and its role in the Sprawl.
\begin{adjustwidth*}{.5cm}{.5cm}

\textit{Residential:} this district is a place where people live,
whether in housing projects, suburbs, apartments, rowhouse, etc.

\textit{Commercial:} this district is primarily occupied by retail and
service businesses of varying size

\textit{Financial:} this district is primariily occupied by financial institutions such as brokerages, stock markets, banks, and
investment firms

\textit{Industrial:} this district is primarily occupied by heavy industry such as construction, manufacturing, and shipping
firms.

\textit{Entertainment:} this district is primarily occupied by entertainment businesses such as casinos, theaters, clubs, bars,
and sports venues.
\end{adjustwidth*}

\textbf{Economy} indicates the general financial strength of the district.
\begin{adjustwidth*}{.5cm}{.5cm}

\textit{Rich:} this district is extremely wealthy, with a great deal of
financial pull in the Sprawl. Examples include high-stakes
financial districts and upper-crust residential areas.

\textit{Affluent:} this district is well-off, with some financial sway.
Examples include luxury residential areas and gated communities, or ritzy entertainment districts.

\textit{Middle-class:} this district has only a modicum of financial
pull, being primarily a middle-class / median income area;
housing is small and efficient, businesses (if there are any)
small as well.

\textit{Poor:} this district is struggling, with little to no resources.
Residences are tiny and shabby, employment is minimal,
and businesses are struggling.

\textit{Slum:} this district is a wasteland, with abandoned buildings, no jobs to speak of, failing (or failed) businesses, and
no monetary influence whatsoever.
\end{adjustwidth*}

\textbf{Population} describes the size (and growth or decline) of the
inhabitants of a district (or the people employed there, if it is
a business district).
\begin{adjustwidth*}{.5cm}{.5cm}

\textit{Booming:} the population is large and getting larger fast;
people are moving there, or businesses are expanding
there at breakneck pace.

\textit{Growing:} the population is large and growing, with a
steady (but not explosive) increase in population.

\textit{Stable:} the population is moderate and steady, with only
minor increases and decreases that tend to even out over
time.

\textit{Dwindling:} people are leaving for some reason, whether
because of abandonment by the city, or failing businesses,
or redevelopment. The current population is small, with
numerous abandoned buildings and businesses.

\textit{Abandoned:} this district has been largely abandoned by
businesses and/or residents. The legitimate population
is tiny, and most buildings are empty and decaying. The
largest population by far is likely to be criminals and the
outcast.
\end{adjustwidth*}

\textbf{Trust} is the final core tag, indicating the districts view of authority, including politicians, law enforcement, and organizations. Remember that this is relative to the 2050’s, where
trust is a little harder to come by anyway.
\begin{adjustwidth*}{.5cm}{.5cm}

\textit{Cooperative:} the community tends work closely with authority.

\textit{Neutral:} the community is neutral toward authority.

\textit{Reserved:} the community is not inclined to trust authority
figures, though it will not actively hamper their work

\textit{Wary:} the community instinctively suspects authority figures and will not cooperate unless compelled.

\textit{Hostile:} the community is openly hostile to authority figures; law enforcement may avoid the area and it may be
“written off” by politicians and organizations
\end{adjustwidth*}

Other tags can be used to add additional description as necessary or for special features of a particular district:
\begin{adjustwidth*}{.5cm}{.5cm}

\textit{Big name:} a person of significant renown (the GM determines to whom) lives or works in this zone

\textit{Corporate:} this neighborhood is owned, managed, and
serves one of the megacorporations or a subsidiary

\textit{Dense:} tight streets, densely packed homes/businesses,
and narrow passages.

\textit{Despair:} the district is blighted and collapsing, and the despair of the people is palpable.

\textit{Highrise:} this area is predominantly high-rise office and/
or residential buildings with few open areas, but well-organized streets

\textit{Infestation:} there is an infestation of some creature in this
area (e.g. goblins, devil rats, etc.). It generally remains hidden inside buildings and underground. Note that this may
be a natural infestation, or something worse

\textit{Isolated:} although uncommon in the \SW{}, there
are some districts that are still difficult to get to, or cut off
from other areas by construction, road modification, and so
forth. Police and emergency response is slowed.

\textit{Lawless:} police presence in this district is absent, and
crime is rampant and unchecked except by the criminals
themselves

\textit{Open:} this area is remarkably devoid of construction, and
has open (perhaps even green) space and room to move
easily (or to move large vehicles)

\textit{Outbreak:} there is a disease outbreak of some sort in this
District; medical services may be present, depending on
the neighborhood’s economic value. If not, quarantine
may be in place.

\textit{Policed:} the neighborhood is regularly patrolled by law enforcement, and response time is short

\textit{Prejudice:} this is a dislike, dismissal, bigotry, or hatred
against a particular category of individuals (perhaps another District, or the police, or orks, or ethnicity)

\textit{Prize:} there’s something in the neighborhood or the land it
sits on that is desired by multiple factions

\textit{Protected:} the neighborhood is protected by some group
(for example, a gang, or a cult)

\textit{Rot:} something poisons this neighborhood, perhaps physically or mentally or spiritually

\textit{Religious:} a religion, cult, or other spiritual movement
holds sway here

\textit{Turf (gang):} this zone is the turf of the indicated gang
\end{adjustwidth*}

\end{multicols}

\section{WILDS}
\begin{multicols}{2}
Most of the action in \SW{} games will take place somewhere in the byzantine environment of a Sprawl. However,
there are plenty of adventure-ready wild spaces left in the
world. In fact, with the upheaval of the early 2000’s, there’s
quite a lot of new wilderness out there, and at some point or
another, you’ll likely end up crossing through it.

If you want to create a \textbf{Wild}, the process is identical to the
creation of a Sprawl: allocate points among the influence of
Man, Magic and Machine, and then determine appropriate
Perils and Crises to accompany those influences.

\subsection{WILDERNESS ZONES}
Just like Sprawls, a single Wild can contain multiple smaller areas with specific characteristics. These smaller areas are
called \textbf{Zones} (since the word “district” doesn’t quite fit). Creating a zone, however, is done the same way as a District:
think of a Zone you want to create, give it a name, and select
the appropriate tags to describe it.
\begin{adjustwidth*}{.5cm}{.5cm}

\textbf{Example:} \textit{the GM creates a region near Lily Lake, deep in
one of the former National Parks. The GM imagines this
to be a thickly forested area, with steep slopes and deep
gullies. Remnants of some park services buildings (mainly
huts and SAR bivouacs) can be found. It’s mostly populated by small animals band birds, althoug a mated pair of
Piasma call this area home. The tags selected for the Zone
are} forest, rugged, typical, ruins, predator.
\end{adjustwidth*}

The tags for the zone are explained below.

\subsubsection{WILD ZONE TAGS}
Because many of the tags for Sprawl Districts wouldn’t necessarily apply, some new tag options are presented below.
Wild Zones have the following tag types: \textbf{type}, \textbf{terrain}, and
\textbf{wildlife}.

(The categorizations that follow—which were greatly trimmed
and simplified for game purposes—may cause painful grimacing in ecologists, forestry experts, geographers, and zoologists; I apologize sincerely).

\textbf{Type} describes the general type of biome and climate of the 
zone. 
\begin{adjustwidth*}{.5cm}{.5cm}

\textit{Plains:} characterized by low rolling hills, open fields of 
grass or scrub, high visibility and winds. Climate varies per 
season. 

\textit{Desert:} characterized by aridity, heat, rolling or rocky ter- 
rain. Deserts may be arctic, but this tag primariliy deals 
with the “hot deserts” of the world. 

    \textit{Aquatic:} a water-based zone, either riverine, limnic, or 
    oceanic. Depending on specifics could be hostile (if sub- 
    aquatic) 

    \textit{Forest:} characterized by a high density of trees of vari- 
    ous types (different categories of forest will have differing 
    dominant tree types); terrain varies 

    \textit{Jungle:} a land area covered with thick, dense vegetation, 
    typically in a tropical area 

\textit{Polar:} cold northern or southern lands in the polar latitutdes, including arctic regions
\end{adjustwidth*}

\textbf{Terrain} describes the zone’s physical features and topography, and how difficult or easy it may be to traverse.
\begin{adjustwidth*}{.5cm}{.5cm}

\textit{Flat:} little to no change in elevation, with only small hills
and depressions

\textit{Rolling:} smoothly transitioning hills, with at times sizable
changes in elevation.

\textit{Wetland:} an area saturated with water, such as a bayou,
delta, swamp, fen, or bog

\textit{Rugged:} terrain with sudden changes in elevation, rocky
outcrops, or thick vegetation that is difficult to navigate
directly or maneuver through

\textit{Mountainous:} rough terrain in a mountainous region, with
large changes in elevation; tiring, demanding terrain

\textit{Broken:} the land is shattered and extremely rugged, very
difficult to cross (almost impassable), and full of blind runs,
rocky outcrops, sharp ridges and technically demanding
terrain.

\textit{Exotic:} the terrain is unusual in some way and not generally encountered; deep subaquatic regions, highly unusual
rock formations, strange caves, and so forth would be examples of exotic terrain
\end{adjustwidth*}

\textbf{Wildlife} describes the flora and fauna of the area, as well as
the relative biodiversity of the zone.
\begin{adjustwidth*}{.5cm}{.5cm}

\textit{Limited:} the zone’s biodiversity is low, marked by only a
few kinds/categories of plants and animals

\textit{Typical:} the zone’s biodiversity is typical for the Sixth
World, having several types of animal and plant species
represented

\textit{Diverse:} the zone is populated by a fairly varied number of
different species, both flora and fauna; edible species are
reasonably easy to find

\textit{Rich:} the zone is rich in different animal and plant species;
it is a busy place

\textit{Hotspot:} the zone is a biodiversity hotspot, teeming with
highly varied species of plants and animals
\end{adjustwidth*}

Other tags may come into play to describe a particular wilderness zone. In addition to the tags below, the tags prize,
protected, and infestation are also applicable.
\begin{adjustwidth*}{.5cm}{.5cm}

\textit{Awakened:} this zone is heavily imbued with magic, whether it be from ley-lines, artifacts, ritual, or other unknown
reason, magic is almost tangibly present.

\textit{Blasted:} some cataclysmic event happened here, and the
scars remain visible.

\textit{Extreme:} the zone is an extreme representative of its
type—a fiercely hot desert, bitterly cold polar region (e.g.
 Antarctica), a dense jungle.

\textit{Megafauna:} the zone contains a relatively high population
of megafauna (animals exceeding 45kg/100lb) such as
deer, large paranimals, and the like.

\textit{Polluted:} this zone is heavily polluted; water is likely undrinkable without treatment and animals and plants dangerous to eat.

\textit{Predator:} there is an apex predator (or mated pair) that
considers this zone its hunting grounds. Be sure to identify
the predator (because your players will ask about it, and
you may have to answer!)

\textit{Remote:} the zone is a long way from civilization. You’re
on your own.

\textit{Ruins:} this zone is composed of, or contains, the abandoned remnants of (meta)human construction.

\textit{Seismic:} this zone is prone to seismic activity, which may
pose a threat

\textit{Storms:} this zone is prone to storms of some sort: electrical, rainstorms, windstorms, snowstorms. These may lead
to related events (fire, flood, etc.)

\textit{Territory:} this zone is the territory of a particular individual
or pack; intruders may be met with extreme aggression.
Make sure to identify the type of creature.

\textit{Wasteland:} this zone is essentially dead—native fauna and
flora has mostly died, water may be scarce or toxic, the
ground poisonous. Inhabitants of this zone (if any) may be
twisted mutants, odd Awakened creatures, strange infestations, or desperate squatters
\end{adjustwidth*}

\end{multicols}

\invisiblepart{CREATING WEAPONS \& GEAR}
\section{CREATING GEAR}
\begin{multicols}{2}
\SW{} uses a “template-based” gear model for most
equipment used in the game: rather than provide extensive
lists of individual items, such as firearms, there are basic templates for broad categories of item, and rules to modify the
templates to suit the player’s needs and desires.

For example, rather than a list of ten heavy pistols, there is one
template for \textit{Heavy Pistol}, with certain basic tags. From there,
the player may add or remove tags based on the guidelines
for doing so. Use these entries to come up with your own, or
modify these as needed.

The rules that follow are optional and experimental (so they’re
not guaranteed to be completely balanced, and you may end
up using the time-tested practice of “make the item using the
rules, then, when it doesn’t feel right, change stuff”).

\begin{adjustwidth*}{.5cm}{.5cm}

\textbf{Bonus Limits:} in general, with the exception of tags that
are the equivalent of wounds, no quality of a piece of gear
may have a value higher than +3.
\end{adjustwidth*}

\subsection{GENERAL TAGS}
As explained in the \textbf{Gear} section, all gear has one or more descriptive tags (not including its price) describing its particular
qualities. Tags may be descriptive (to aid with the fiction), or
have mechanical import. The following tags apply to multiple
types of equipment.
\begin{adjustwidth*}{.5cm}{.5cm}

\textit{2-hand:} this item must be used with both hands

\textit{Armor +n:} grants a +n bonus to existing armor

\textit{Armor n:} grants n Armor (for vehicles or drones, indicates
armor rating, and is abbreviated arm)

\textit{Arcane:} can only be used by magical archetypes

\textit{Area:} affects multiple targets

\textit{+Bonus:} grants a bonus to a particular move; e.g. +1 to
Stay Frosty

\textit{Conceal:} this weapon or item is easily hidden and will not
be spotted by enemies

\textit{Damage n:} the amount of damage a weapon or other item
deals. Abbreviated dmg

\textit{Heal n:} restores n wounds

\textit{Loud:} noisy and audible to anyone with functionin hearing;
for weapons, it means the weapon cannot be suppressed

\textit{Messy:} deals damage in a particularly gruesome way

\textit{Obvious:} cannot be concealed, or is immediately visible to
any observer

\textit{Range:} the range(s) at which the weapon or other attack is
effective. Ranges are close (c), short (s), medium (m), and
long (l).

\textit{Special (description):} if the effect of the item requires explanation, use this tag.

\textit{Stun:} this weapon or attack deals Stun damage only

\textit{Subtle:} not easily noticed (as opposed to conceal, which
means it is unnoticeable)

\textit{Supply n:} the amount of supplies or uses you can get out
of an item. Each use of the item consumes 1 supply (unless
otherwise stated).
\end{adjustwidth*}

\subsection{CREATING WEAPONS}
The templates below represent a starting point to begin customizing a weapon. There are only a few templates, since
most of the rest of the process can be handled through customizing and modifying the item’s price. The basic weapon
templates are:

\textbf{melee weapon} \textit{[range c, dmg 1d6, 150¥]}

\textbf{light pistol} \textit{[range s/m, sa, dmg 1d6, ammo 3, 300¥]}

\textbf{heavy pistol} \textit{[range s/m, sa, dmg 1d8, ammo 2, 450¥]}

\textbf{submachine gun} \textit{[range s/m, sa/bf, dmg 1d8, ammo 3,
700¥]}

\textbf{longarm} \textit{[range s/m/l, sa, dmg 1d10, AP 1, obvious,
ammo 4, 600¥]}

\textbf{heavy weapon} \textit{[range m/l, fa, dmg 1d12, AP 2, loud,
obvious, stabilize, messy, ammo 4, 2,500¥]}

\subsubsection{DAMAGE EXPRESSIONS}
Damage expressions can be put in order from the smallest
damage die (1d4) through the largest (1d12), with modifications in between. Here’s how the damage options in Sixth
World progress:

\begin{center}
\begin{tabular}{c}
\toprule
Average Damage (low to high)\\
\midrule
1d4 \\
2d4b \\
1d4+1 \\
d6 \\
2d6b \\
1d6+1 \\
1d8 \\
1d8+1 \\
1d10 \\
2d8b \\
1d10+1 \\
1d12 \\
2d10b \\
1d12+1 \\
2d12b \\
\bottomrule
\end{tabular}
\end{center}

\textbf{Notes:}
\begin{adjustwidth*}{.5cm}{.5cm}
1. No [w] rolls. The
“worst” roll modifier is
a significant penalty,
especially as the die type
gets bigger. Save it for
broken gear and things
that interfere with the
characters.

2. The progression isn’t
nicely ordered, because
the [b] roll gets progressively better as the dice
type gets higher.
\end{adjustwidth*}

\subsubsection{WEAPON TAGS}
Weapons use the following tags (in addition to the general
tags from the preceding page):
\begin{adjustwidth*}{.5cm}{.5cm}

\textit{AP n:} this weapon ignores n points of armor.

\textit{Auto:} this weapon can fire in full auto mode (take +1 to
suppression fire). Treat as burst otherwise. Abbreviated fa.

\textit{Burst:} this weapon fires in burst mode (mark off 1 ammo to
deal +1 damage). Abbreviated bf.

\textit{Chem:} this weapon delivers a chemical agent of some kind
to the target; depending on the delivery mechanism, armor may be ignored.

\textit{Forceful:} when this weapon deals damage, it also deals 1
stun

\textit{Fuzed:} this weapon cannot be used at less than the shortest range increment listed

\textit{Reload:} after using this weapon, it takes more than a moment to reload it.

\textit{Semiauto:} this weapon fires one shot every time the trigger is pulled. Abbreviated sa.

\textit{Stabilized:} this weapon cannot be fired except from a bipod, tripod, or supported position.

\textit{Suppressed:} this weapon makes little to no noise when
fired

\textit{Thrown:} this item can be throw. If thrown, the range is
short.

\textit{Vented:} the weapon has recoil venting, granting +1 to
Suppression Fire
\end{adjustwidth*}

\subsection{CUSTOMIZING WEAPONS}
To build a custom weapon, follow these steps:

\begin{adjustwidth*}{.5cm}{.5cm}
1.	Choose base template.

2.	If creating the weapon during character creation, you
have 3 points to spend on customizations. If you’re buying it, the only limit is how much nuyen you’ve got on
your credstick.

3.	Modify the base template as you like: adjust damage,
rate of fire, ammo, and other tags by spending points or
adjusting the final price of the weapon.

4.	If you like, give your new weapon a name.
\end{adjustwidth*}

\subsubsection{WEAPON CUSTOMIZATIONS}
\textbf{HI-POWER}

Increasing the power of a weapon raises the damage expression (and, if the damage expression becomes a [b] roll,
also increases the consistency of that damage somewhat, reflecting an “in-world” improvement in control). Up-gunning
a weapon raises the damage expression one step (use the
table on the preceding page to figure out the new damage).
You can increase a weapon by a maximum of 3 steps (e.g.,
1d6 to 1d8); each increment costs 1 point or adds 50¥ to the
base cost.

\textbf{LOW POWER}

The opposite of increasing power. You can reduce a weapon’s
damage expression by up to 2 steps to gain points for other
options, or to reduce the price. Each decrement provides 1
point or reduces the cost by 25\%.

\textbf{EMBEDDED}

The gun is built into an otherwise unremarkable non-cyberware object (such as a camera or briefcase). Doing so makes
it undetectable, but reduces accuracy. Subtract 1 from the
damage. Cost: -50\% / -1 point.

\textbf{CHANGING FIRE MODES}

You can add or remove firing modes from a weapon. Adding
a fire mode is a positive, while removing fire modes from a
weapon that already has them is negative. Note that if you
restrict a weapon to burst or full-auto mode, it always costs
ammunition to use, which can be a fairly significant penalty.

\textbf{PRICE REDUCTION}

When building a new weapon using the point by system, if
you have unused points you can use them to reduce the final
price of the weapon. Drop 50¥ from the price per point spent.

\subsubsection{MODIFYING TAGS}
You can add or remove tags from weapons, paying for (or
getting rebates back) depending on the tag. Positive tags cost
build points or more nuyen, while negative tags grant more
points or reduce the price of the weapon. The table below
lists the tags as well as their cost.

\textbf{Note:} positive and negative is relative to the tags the weapon already has. In other words, adding burst fire mode to
a pistol is a positive thing. If you removed it from an SMG
instead, then it would be a negative modification. The table
below simply indicates the value of the tag in points or nuyen added or subtracted when modifying a base template.

\begin{center}
\begin{tabular}{lp{5.5cm}}
\toprule
Tag Type& Tags\\
\midrule
1 pt / 50 ¥ & 2-hand, add/remove range increments, add/reduce ammo, additional fire modes, suppressed, vented, +bonus, subtle, stabilized, loud, messy, stun, chem, smart\\
2 pts / 100¥ & AP, forceful, ignores armor(e), 2-hand, fuzed, obvious, reload, conceal\\
\multicolumn{2}{p{5.5cm}}{e) - exceptional tag, twice the normal value

(m) - melee weapon tag
}\\
\bottomrule
\end{tabular}
\end{center}


\subsection{CREATING CYBERDECKS}
Cyberdecks are the essential tool of the hacker. They are the
Hacker’s connection to the Matrix, his weapon, his instrument, his toolbox, and his armor when he’s throwing down
with serious Matrix security.

\textbf{TAGS
}
\begin{adjustwidth*}{.5cm}{.5cm}

\textit{CPU:} the raw processing power of the deck

\textit{Mask:} the stealthiness of a cyberdeck

\textit{Hardening:} the deck’s resistance to damage

\textit{Storage:} the deck’s capacity for loaded programs
\end{adjustwidth*}

\subsubsection{DECK TEMPLATES}
Each template below provides a number of Gear Points (gp)
to distribute among the four tags listed above. Lower end
decks offer fewer points to play with, while the high-end
dream decks can be powerful rigs indeed. All decks start with
a base of 8 storage, and no deck can have a tag higher than 3.

\textbf{Entry Level} [\textit{3 gp, 25,000¥}]

\textbf{Mid-Range} [\textit{4 gp, 50,000¥}]

\textbf{High-End} [\textit{5 gp, 75,000¥}]

\textbf{Elite} [\textit{6 gp, 100,000¥}]

\subsection{CREATING VEHICLES}
Vehicles have the following tags describing their capabilities:

\begin{adjustwidth*}{.5cm}{.5cm}

\textit{Power (pwr):} the vehicle’s horsepower, speed, and acceleration.

\textit{Armor (arm):} the vehicle or drone’s armor rating.

\textit{Frame (frm):} the vehicle’s or drone’s resilience. This is the
equivalent of the vehicle’s wounds. Remember that vehicles take half damage from small arms, and none from
melee weapons.

\textit{Sensors (ssr):} the quality of the vehicle’s sensors (used
when Checking the Situation while driving or piloting the
vehicle)

\textit{Seats n:} the number of people who can occupy the vehicle,
including the driver or pilot

\textit{Fuel:} fuel capacity
\end{adjustwidth*}

\subsubsection{VEHICLE TEMPLATES}
When designing a vehicle, select a template below, distribute
the indicated Gear Points (gp) among the 4 core stats as desired. Power, armor and sensors may not have a value higher
than 3.

The base fuel and frame of the vehicle will be indicated in
each template. You can spend as many gear points as you
wish to increase those tags.

\textbf{BIKES}

\textbf{Scooter} [\textit{3 gp, 3 fuel, frm 3, seats 1, 1,800¥}]

\textbf{Street Bike} [\textit{5 gp, 3 fuel, frm 4, seats 2, 6,500¥}]

\textbf{Racer} [\textit{4 gp, 3 fuel, frame 3, +1 pwr, seats 1, 9,500¥}]

\textbf{Offroader} [\textit{5 gp, 3 fuel, frm 4, seats 2, 4,850¥}]

\textbf{Hog} [\textit{6 gp, 2 fuel, frm 5, seats 2, 17,500¥}]

\textbf{CARS}

\textbf{Economy} [\textit{4 gp, 3 fuel, frm 5, seats 3, 10,000¥}]

\textbf{Standard} [\textit{5 gp, 3 fuel, frm 6, seats 4 16,000¥}]

\textbf{Sports} [\textit{6 gp, +1 pwr, frm 5, 2 fuel, seats 2, 36,000¥}]

\textbf{Luxury} [\textit{6 gp, +1 ssr, frm 6, 3 fuel, seats 5, 85,000¥}]

\textbf{Exotic} [\textit{7 gp, +1 arm, frm 6, 2 fuel, seats 6, 200,000¥}]

\textbf{TRUCKS}

\textbf{Van} [\textit{6 gp, frm 8, 2 fuel, seats 8, 35,000¥}]

\textbf{Light Truck} [\textit{6 gp, +1 pwr, frm 10, 2 fuel, seats variable,48,000¥}]

\textbf{Heavy Truck} [\textit{7 gp, frm 12, +1 pwr, 2 fuel, seats variable,125,000¥}]

\textbf{ROTORCRAFT / VTOL}

\textbf{Helicopter} [\textit{6 gp, +1 ssr, frm 10, 3 fuel, seats 6, 100,000¥}]

\textbf{VTOL} [\textit{7 gp, +1 ssr, 4 fuel, frm 10, seats 8, 355,000¥}]

\subsection{CREATING DRONES}
Drones are built the same way as vehicles, and have most
of the same qualities. However, drones have the following
additional stats:

\begin{adjustwidth*}{.5cm}{.5cm}
\textit{Tactical (tac):} the quality of the drone’s tactical expert system, which comes into play when the drone is in autonomous mode. Tac may not have a value higher than 3.
\end{adjustwidth*}

\subsubsection{DRONE TEMPLATES}
\textbf{Ground Surveillance} [\textit{3 gp, +1 ssr, frm 4, 2 fuel, 1,800¥}]

\textbf{Ground Sentry} [\textit{4 gp, +1 arm, 1d6 dmg, frm 6, 2 fuel, 4,500¥}]

\textbf{Ground Combat} [\textit{4 gp, +1 tac, 2d6b dmg, frm 8, 3 fuel, 8,000¥}]

\textbf{Air Surveillance} [\textit{3 gp, +1 ssr, frm 3, 2 fuel, 2,500¥}]

\textbf{Air Sentry} [\textit{4 gp, +1 ssr, 2d4b dmg, frm 4, 2 fuel, 12,000¥}]

\textbf{Air Combat} [\textit{5 gp, +1 tac, 1d8 dmg, frm 6, 3 fuel, 22,000¥}]

\end{multicols}


\invisiblepart{CREATING CYBERWARE}
\section{CREATING CYBERWARE}
\begin{multicols}{2}
Cyberware, like other equipment in \SW{}, can be described using a set of tags. Generally cyberware augments
a character either by providing capabilities that the character did not have (nor could have naturally) such as a direct
connection to a device or foot-long razors on their wrists, or
enhances an existing capability such as their reaction time or
toughness.

Since it’s possible to describe cyberware in terms of tags,
it is also possible to perform some customization of cyberware devices (although they’re usually pretty fixed in their
performance). The most typical customization possible is in
the cyberware’s grade, which indicates the general level of
enhancement it provides, and in its damage capability (for
cyberweapons).

\subsubsection{ACTIVATION CYBERWARE}
Cyberware is activated by spending Edge. By default, a cyberware system requires the user to spend 1 Edge to activate
it, each time they wish to use it (that is, each time the user
wishes to gain its benefits). The \textit{toggle} and \textit{always on} tags
modify this general rule, as described in the \textbf{Other Cyberware Tags} section.

\subsubsection{INSTALLING CYBERWARE}
Installation of cyberware is an advanced surgical procedure
that must be taken during downtime or legwork time due
to recovery time. There are two general types of cyberware.

\textbf{Implants} are cyberware that are installed inside the recipient’s body. The extent of the installation and the amount of
Essence lost varies; a datajack is a relatively trivial installation,
while wired reflexes involve an extensive whole-body procedure and a considerable amount of recovery time. Any cyberware item aside from, obviously, a replacement part can be
installed as an implant (for example, you don’t need to have
cybereyes to get cybernetic low-light vision).

\textbf{Full Replacements} are cyberware that fully replaces an
equivalent part of the recipient, such as eyes, ears, or limbs.
Like implants, the invasiveness of such a procedure varies,
but replacements are in general more invasive than implants.
By themselves, replacements offer no additional capability.
However, full replacements have the following benefits:

\begin{adjustwidth*}{.5cm}{.5cm}

\tcirc{} full replacments can have optional components installed
into them with no further essence cost; instead, the
component takes up capacity equal to its essence cost.
\end{adjustwidth*}
\subsubsection{ESSENCE COST}
Every time cyberware is installed in a metahuman, it costs a
bit of their essence. This loss depends on the invasiveness of
the surgery required, the biological systems modified, and
the grade of the cyberware. State of the art cyberware has a
significantly decreased essence cost, but is also significantly
more expensive. A character may not reduce their Essence
below 0.

The tag for Essence cost is simply essence n, where n is the
amount of essence the item costs to install.

\subsubsection{PRICE}
Unfortunately, there’s no “generic” piece of cyberware, so
there’s no “standard price” to start from when customizing
cyberware. The cost of the implant is based on a lot of factors:
how invasive it is, how technically complex, what exactly it
does, and how much the legal system and corporations frown
upon John Q. Citizen having something that does that. Your
standard datajack is an innocuous device, and might cost you
around a thousand nuyen. On the other hand, having a pistol
hidden inside your arm is probably going to cost a lot more,
because no matter how convincing you are, most people
won’t believe you when you say it’s just for target practice.

One of the jobs of the GM will be, if you use these customization rules, to figure out the base prices for different items.
Some very loose (essentially guesswork) guidelines are given
below:

\textbf{Common Legal Items}

\begin{adjustwidth*}{.5cm}{.5cm}


\tcirc{} Minimally invasive: 1,000 - 5,000¥

\tcirc{} Moderately invasive: 7,500 - 40,000¥

\tcirc{} Highly invasive: 50,000 - 100,000¥
\end{adjustwidth*}
Restricted or regulated items will be more pricey. A premium
of 25-50% over the cost of an equivalently invasive legal item
might be appropriate.

For flat-out illegal ‘ware, the sky’s the limit. It’s illegal to have
the augmentation in the first place, so the black market can
pretty much ask whatever it wants.

\subsection{CREATION RULES}
Although there is a list of “typical” cyberware in this section,
it is possible to create or customize cyberware items using
the rules in this sectionThe steps below describe how to create a new piece of cyberware (these are explained in more
detail below):

\begin{adjustwidth*}{.5cm}{.5cm}


1.	Select either \textit{standard} or \textit{sota} grade.

2.	Select the item’s general function.

3.	Decide how invasive the augmentation is, noting the
base essence and price, as well as selecting the appropriate benefit based on function.

4.	Choose additional tags, adjusting the final essence and
nuyen cost as necessary.

5.	Write out the tags, and name the item.
\end{adjustwidth*}
For items installed in full replacements, after you figure out
the final cost and stats, you may want to record the individual
components down, and simply note that they’re installed in
the containing implant, rather than jam everything into one
endless and unintelligible stream of tags.

\subsubsection{CYBERWARE QUALITIES}

\textbf{\large GRADE}

\textbf{Standard:} this is your basic “off the shelf” augmentation,
and is the default grade. Standard cyberware has the following characteristics
\begin{adjustwidth*}{.5cm}{.5cm}

\tcirc{} Essence cost of 1, 2, or 3, depending on invasiveness

\tcirc{} Full replacements can have 2 add-on components
\end{adjustwidth*}
\textbf{State of the Art:} state of the art (sota) cyberware uses the
latest technology to improve performance and customize
it to your specific physiology and genetic makeup, reducing its essence cost. SOTA cyberware has the following
characteristics:
\begin{adjustwidth*}{.5cm}{.5cm}

\tcirc{} Essence cost is 0, 1, or 2 (for replacements with addons, add up the total cost for all components, then multiply). Yes, minimally invasive implants cost no essence.

\tcirc{} Base cost is multiplied by 3

\tcirc{} Full replacements can have 3 add-on components (these
must also be SOTA-grade)
\end{adjustwidth*}
\textbf{\large FUNCTION}

Cyberware is highly varied, but has two general mechanical
functions in the game: \textbf{modify a move}, or \textbf{grant a new capability}. Therefore, a cyberware item may have one of the
following two tags:
\begin{adjustwidth*}{.5cm}{.5cm}

\textit{modifies:} many enhancements affect a specific move or
moves; this tag describes the specific modification. For example, a smartlink alters the Rock \& Roll move, so the
tag list will contain modifies(Rock \& Roll), along with a
description of the specific benefit.

\textit{ability:} the implant adds a new ability the recipient did
not previously have (for example, armor, low-light vision,
sound damping, a gun hidden in their toe, etc.). The ability
added is usually evident from the name of the item (e.g.
“Thermographic Vision Implant”), but if not, put the specific ability in parenthesis after this tag. Use the \textit{special} tag to
describe specific effects, as needed.
\end{adjustwidth*}

\textbf{\large INVASIVENESS}

The extent of the surgery required to install cyberware dictates both its base essence cost and its base cost in nuyen. In
general, the more substantial the augmentation or the more
fundamental or sensitive the systems being modified, the
more invasive the surgery.

\textbf{Level 0:} this level of cyberware is minimally invasive, requiring little essence loss. This type of cyberware has the following characteristics:
\begin{adjustwidth*}{.5cm}{.5cm}

\tcirc{} Base essence cost of 1

\tcirc{} Typical Systems: device links, vision enhancement,
hearing enhancements, replacement eyes, replacement
ears, installed devices, small compartments, implanted
light blade, implanted holdout pistol

\tcirc{} \textbf{Benefit (choose 1):} new ability, 1d4 damage, special
effect
\end{adjustwidth*}
\textbf{Level 1} cyberware requires a bit more surgical intervention
to install, a longer recovery time, and has more of an impact
on the recipients system. This level of augmentation has the
following characteristics:
\begin{adjustwidth*}{.5cm}{.5cm}

\tcirc{} Base essence cost of 2

\tcirc{} Typical Systems: armor implants, hazard protection,
wired reflexes, skillwires, compartments, implanted
medium blade, implanted light pistol

\tcirc{} \textbf{Benefit (choose 1):} new ability, 1d6 damage, +1 Armor, Hold 1, special effect
\end{adjustwidth*}
\textbf{Level 2} cyberware is highly invasive and complex, requiring
considerable modification of the recipient. It brings with it a
correspondingly high monetary and essence cost. This level
of augmentation has the following characteristics:
\begin{adjustwidth*}{.5cm}{.5cm}

\tcirc{} Base essence cost of 3

\tcirc{} Typical Systems: replacement limbs, wired reflexes, armor implants, skillwires, move-by-wire system, cybertorso, hazard protection, implanted heavy pistol

\tcirc{} \textbf{Benefit (choose 1):} new ability, 1d8 damage, +2 Armor, Hold 2, special effect
\end{adjustwidth*}
\subsubsection{OTHER CYBERWARE TAGS}
Cyberware can use many of the same general tags that other
equipment use, such as \textit{armor}, \textit{range}, or \textit{obvious}. The tags
below are unique to cyberware.
\begin{adjustwidth*}{.5cm}{.5cm}

\textit{add-ons:} this is installed in an existing piece of cyberware,
instead of independently. The item takes up capacity equal
to its essence cost. \textbf{Note:} components \textbf{do not} inherit the
always on or toggle tags from the item in which they are
installed.

\textit{always on:} the implant remains on all the time. If adding
this tag to an item that modifies a move, multiply the cost
of the implant by 2. Full replacements always have this tag,
but their components do not inherit it.

\textit{n capacity (cap):} the cyberware item has capacity for n
add-on items. If add-ons are listed, this tag should show
the remaining capacity. Only full replacements can have
the cap tag. Capacity can be added in increments of 0.5 by
increasing the base cost of the item by 25%.

\textit{device:} this implant is a device of some sort (usually a
weapon or computing tool) that does not offer sensory
modification. If installed as an add-on, it must be installed
in a replacement with the device tag.

\textit{link (device):} this cyberware must be connected to the
proper kind of device to be effective (for example, a smartlink must be connected to a weapon with a smartgun system)

\textit{loaner:} this implant was given to you by an organization
lots of money, and they expect you to repay them somehow. This tag can reduce or eliminate the financial cost for
an implant, but it comes with a different sort of price tag.

\textit{resist (hazard):} the augmentation protects against particular environmental hazards such as toxins or electrocution

\textit{sealed:} a sealed implant requires at least an hour and the
proper tools to reload or refill. Reduce the base cost by
25%.

\textit{toggle:} this item is toggled on and off (that is, once activated, it stays on). For items that modifies a move, multiply
the cost of the item by 1.5.

\textit{used:} this implant started its life in someone else’s body,
and it shows. The first time you fail a move related to the
implant or are in a situation where the added capability
of the device comes into play, roll 1d6. On a 3 or better,
you’re fine. On a 2, the implant simply fails gracefully. On
a 1, the implant goes haywire:
\begin{adjustwidth*}{.5cm}{.5cm}

\tcirc{} If the implant modifies a move, that move is glitched
until you get it fixed or shut down

\tcirc{} If the implant provides a capability, that ability becomes
a big problem (for example, if your used thermographic
vision goes haywire, you may be temporarily blinded)

\tcirc{} You can shut down a haywire implant by spending a
point of Edge.
\end{adjustwidth*}
\end{adjustwidth*}
\subsubsection{CHANGING MOVES}
When a cyberware item modifies a move, the basic version of
it always modifies a core or secondary moves, so it’s useful
to all of the different archetypes. However, if you want to
change the move modified by the item to one of your archetype moves, go right ahead. There’s only one rule: you can’t
double up. If you have an archetype move that grants a bonus
or grants Hold, you can’t change a cyberware item to grant
more Hold for that move. Just take the highest amount.

\subsubsection{MODIFYING TAGS}
If you add a beneficial tag, increase the cost of the item. If
you add a negative tag (such as obvious, or used), reduce the
overall price to reflect this.

\end{multicols}

\invisiblepart{CREATING PROGRAMS}
\section{CREATING PROGRAMS}
\begin{multicols}{2}
Programs act as a Hacker’s weapons, tools, and enhancements in the matrix. They may alter the stats of a cyberdeck,
or enhance your ability to damage enemy code, or help you
pull off moves. A program loaded into a cyberdeck’s storage
is assumed to be running. Changing programs is done by declaring it, or via a move, as the situation demands.

Now, no self-respecting codeslinger buys off-the-shelf software, for a couple reasons: one, there usually isn’t a shrinkwrapped program out there for the things the Hacker wants
to do; and two, if there was, you certainly don’t want anyone
to know you bought it.

So what is the Hacker to do? Well, write code, of course!
Here are rules for creating your own tools for bending the
matrix to your will.

\subsection{CODING}
Programs consist of one or more routines, which are appended to the program name as tags. Each routine offers a different effect or benefit; multiple routines can be combined into
a single piece of software.

Writing programs follows a simple procedure:
\begin{adjustwidth*}{.5cm}{.5cm}

1.	Name the program (I encourage you to come up with
suitably Zero Cool names for programs)

2.	Add routines to the program, spending the required
time or money to develop them.

3.	Calculate the size of the program, which is how much
storage it occupies. A program’s size is equal to the
\textbf{number of routines x 2.}

\textbf{Example:} \textit{Blitz is writing a new program for her deck for
an upcoming run. She hopes to slip in, crack the datastore,
and get out. She calls the program NinjQk, and gives it
the routines analyze, stealth, and decrypt. This program
has size 6.}
\end{adjustwidth*}
\subsubsection{PROGRAM ROUTINES}

\begin{adjustwidth*}{.5cm}{.5cm}

\textit{Analyze:} this routine lets the hacker roll+Matrix to Check
the Situation while in VR.

\textit{Attack:} deal 1d6 damage to targeted node, program, or
hacker

\textit{Bounce:} temporarily relocate a hostile program to another
node in the system

\textit{Armor:} this routine increases a cyberdecks Hardening by 1

\textit{Stealth:} this routine increases a cyberdeck’s Mask rating
by 1

\textit{Scan:} this provides +1 ongoing to Awareness-based Stay
Frosty

\textit{Repair:} corrects errors and restores crashed code; heal 1
matrix damage

\textit{Interference:} slows hostile program alarm triggers

\textit{Decrypt:} take +1 to hacking Datastore nodes

\textit{Interface:} take +1 to hack or use Control nodes

\textit{Backdoor:} allows the hacker to automatically gain access
to a specific node at some point in the future.
\end{adjustwidth*}
STACKING ROUTINES

You can add up to two copies of a single routine to a program. Doing so doubles its effect or the number of times you
can use the routine. For example, Harden can be stacked,
raising the bonus to hardening to +2. \textbf{Note:} when Attack is
doubled, it becomes 2d6b damage.

ON TIME, UNDER BUDGET
When creating programs (with the exception of during character creation), Hackers will need to devote time to writing,
debugging, and perfecting their code. Creating a program
requires the Hacker to spend one day per routine.

Of course, shadowrunners don’t always have the luxury of
time. If a hacker doesn’t have the time to write his or her own
code, he or she can work their contacts to purchase black
market bits. The average cost for a single routine is 250¥.
\begin{adjustwidth*}{.5cm}{.5cm}

\textbf{Example:} \textit{Blitz’s new program, NinjQk, needs to be done
pretty quick. She has one day free, so she spends that
writing the analyze routine. However, she’s out of time
by then, so she calls up a couple buddies and snags some
stealth and decryption libraries from them. Since they were
friends, they cut her a break, and she scored the two routines for about 400¥.}
\end{adjustwidth*}
\subsection{AGENTS}
As programs are assembled from multiple routines, it is possible to compile multiple programs into an autonomous expert system called an \textbf{agent}, virtual companions to a hacker
that act independently of the hacker but in accord with his or
her wishes.

Only one agent can be in operation at once. Agents have the
following characteristics:
\begin{adjustwidth*}{.5cm}{.5cm}

\textbf{CPU:} this is the primary stat of the Agent, and is used
when executing its moves

\textbf{Wounds:} a Agent’s wounds are equal to the combined
size of its constituent programs

\textbf{Moves:} Agents use the Sling Code and Born Digital moves

\textbf{Other Stats:} any other stats an Agent depend on its constituent programs (e.g., if a constituent program has the
Armor routine, the Agent has Armor 1)
\end{adjustwidth*}
To create an Agent:
\begin{adjustwidth*}{.5cm}{.5cm}

1.	Choose up to 6 storage worth of programs already running on your cyberdeck to compile together.

2.	Allocate at least 1 point from your cyberdeck’s CPU to
the Agent’s CPU stat. A cyberdeck whose CPU is reduced to 0 in this fashion is not destroyed; it simply has
all of its primary power devoted to the agent, and CPU
cannot be added to the result of any Hacker moves.

3.	Determine the Agent’s wounds and other characteristics per the information above.
\end{adjustwidth*}
\end{multicols}

\invisiblepart{CREATING SPELLS}
\section{CREATING SPELLS}
\begin{multicols}{2}
Spell creation in \SW{} is relatively simple, and requires
only that you name the spell, and then assign it the appropriate tags to describe how it works, based on the Spell Templates presented in the next section.

Every spell must have all core tags assigned; additional tags
may be assigned (see Other Spell Tags) as necessary (or when
required in the rules that follow).

\begin{adjustwidth*}{.5cm}{.5cm}

\textbf{Example:} \textit{Lynn, playing the Mage, wants a spell that
shoots a jet of acid at the target. She calls it Acid Spray,
and gives it the following tags:} close/short/medium, creature, instant, Force 3, 1d8+EF dmg, element:acid,
obvious.
\end{adjustwidth*}

\begin{adjustwidth*}{.5cm}{.5cm}

\textbf{Example 2:} \textit{Lynn’s not all about hurting people; sometimes she needs to protect herself too! She creates a spell
she calls Astral Armor. It is a Manipulation spell affecting
only her, triggered by any incoming damage, and not obvious to casual observers. She starts with some basic tags:}
touch, self, triggered, Force 1, effect:+1 armor. \textit{Since it’s
a protective Manipulation spell, it gets the protection tag
as well. She wants it to be a bit more potent, so she decides to add the} exhausting \textit{tag to increase effect to +EF armor.
Finally, she wants to add the} subtle\textit{ tag, which requires an
extra point of Force. The final spell, then, is} Astral Armor
[touch, self, triggered, Force 2, protection, subtle, exhausting, effect:+EF armor against one attack]. \textit{It’s a costly
spell, but a nice way to have some low-profile protection
against surprise attacks.}
\end{adjustwidth*}

\subsection{SPELL FORCE}
All spells start with a minimum Force of 1. This is the force at which the spell must be cast to gain any effect. Some tags increase this minimum. No combination of tags can reduce a spell's minimum below 1. When determining the effects of the spell, use the \textbf{Effective Force}, or \textbf{EF}, value which is the \textbf{(Force Cast - Minimum Force) + 1}. 

\begin{adjustwidth*}{.5cm}{.5cm}
\textbf{Example:} \textit{Lynn wants to cast her Acid Spray spell which has a minimum Force of 3. Casting this spell at Force 3 would yield an EF of 1. Lynn wants more damage potential, so she decides to cast a more powerful Force 5 Acid Spray, which yields an EF of 3.}
\end{adjustwidth*}


\subsubsection{SPELL TEMPLATES AND TAGS}

All spells share a core set of tags describing their \textbf{Range, Targets, Duration, Essence}, and \textbf{Effect}.

\textbf{Range} describes the effective range over which the spell can
be cast. Remember that most spells require line of sight to the
target. Combat spells, by default, have the \textit{LOS} tag. By default, a spell can only have one \textbf{Range} tag.
\begin{adjustwidth*}{.5cm}{.5cm}

\textit{Touch:} the spellcaster must touch the target to cast the
spell.

\textit{LOS:} the spellcaster must be within line of sight of the target. Technological vision enhancements (aside from old fashioned optics) do not count for line of sight.

\textit{Linked:} the spellcaster must possess an object of high significance to the target, or a fresh (under 24 hours old) bodily sample. With an appropriate link, the spell has a range of \textbf{EF} kilometers.

\end{adjustwidth*}
\textbf{Target} indicates the valid targets for the spell. Spells are by
default single target, though they may have multiple valid
target types.
\begin{adjustwidth*}{.5cm}{.5cm}

\textit{Self:} the spell only affects the caster

\textit{Metahuman:} the spell only affects metahumans

\textit{Creature:} the spell affects any living creature

\textit{Spirit:} the spell affects only spirit beings

\textit{Object:} the spell affects inanimate objects

\textit{Device:} the spell affects technological devices
\end{adjustwidth*}
\textbf{Duration} specifies how long the effects of a spell normally
last. \textbf{Note:} wound or stun damage removed by a spell does
not come back when the spell’s duration is up, unless that is
specified in the spell effect itself. For ease of play, those sorts
of effects are permanent.
\begin{adjustwidth*}{.5cm}{.5cm}

\textit{Instant:} the spell occurs very quickly. \textbf{All Combat spells
have instant durations.}

\textit{Short:} the spell lasts long enough for the target to take one
move, more or less (this is common for spells that boost a
single move or enhance a Stat temporarily). Triggered (see
\textit{Other Spell Tags}) can replace this tag at the caster’s discretion. \textbf{All spells except Combat spells have a default
duration of short.}

\textit{Sustained:} the spell remains in effect for a period determined by the caster. Each sustained spell in effect inflicts a stacking -1 to future spellcasting moves to account for the split concentration of the caster. Common for spells that
grant ongoing bonuses.

\textit{Specified:} the spell lasts for a specific amount of time (e.g.
5 minutes, 30 minutes, 1 hour).
\end{adjustwidth*}
\textbf{Force} indicates the minimum Force expenditure required to cast the spell. No customizations can reduce a
spell’s minimum Force below 1.

\textbf{Effect} describes the actual result of a successful casting of
the spell. Spell effects are extremely varied, but generally do
such things as enable previously impossible abilities (breathing underwater, or perceiving remote events), enhance existing abilities (offering bonuses or Boosts to moves or Stats), or
healing or inflicting damage. \textbf{Note:} the effect of combat spells
is almost always, of course, to inflict damage.

\subsection{CUSTOMIZING SPELLS}
Using the basic tags as well as tags specific to certain spell
categories (if any), spells can be modified in order to meet the
caster’s needs. Most modifications simply require the caster
to commit more essence to power the spell.

The following modifications are common:
\begin{adjustwidth*}{.5cm}{.5cm}

\textit{More Targets:} additional valid target types or additional
targets can be added to a spell. For each target type added, increase the minimum Force by 1.

\textit{Discreet Casting:} all spells are assumed to have the obvious tag, indicating that you can’t miss the mage going
through the motions to cast the spell. To add the subtle tag
to hide the casting process, increase the minimum Force by 1.

\textit{Increased Range:} to add an additional range tag, increase
the minimum Force by 1. By default, Combat spells start with
LOS; Health spells start
with touch, and other spells with touch, LOS.

\textit{Decreased Range:} in some cases you may wish to decrease
the effective range of a spell in order to decrease its minimum Force. Remove the longest range increment and either reduce minimum Force, or (for damaging spells) stage
the damage die type down one step.

\textit{Potent Effect:} you may double the potency of a non-combat spell’s effect, by adding the exhausting tag (modifying
the effect of combat spells is described in that section).

\textit{Increase Duration:} some spells (usually Health, Illusion,
and Detection spells) have durations longer than instant.
Increasing the duration of the spell by one step increases the minimum Force by 1.
\end{adjustwidth*}

\subsubsection{TYPE-SPECIFIC SPELL TAGS}

\paragraph{COMBAT}

Combat spells have the following specific customization options:
\begin{adjustwidth*}{.5cm}{.5cm}

\textit{Damage:} instead of an effect tag, combat spells deal damage (similar to weapons). All combat spells start with a
base damage value of \textbf{1d6}. Spell damage can be upgraded
in a couple ways, each with a cost:

\textit{Scaling (add EF to damage):} either remove the highest range increment
from the spell, or add the obvious tag

\textit{Upgrade damage die:} increase the minimum Force
of the spell by 1, and add the obvious tag

\textit{Downgrade damage die:} reduce the damage die by 1 step to reduce minimum Force by 1 

\textit{Modify the damage to a “best” roll:} add the \textit{exhausting} tag

\textit{Ignore Armor:} reduce the damage die by 1 step

\end{adjustwidth*}
\paragraph{DETECTION}

Detection spells have the following specific tags:
\begin{adjustwidth*}{.5cm}{.5cm}

\textit{Analysis:} the spell is designed to analyze the workings of
an object, device, or similar target

\textit{Perception:} the spell enhances the target’s perceptive capability or to enable otherwise impossible feats of perception (such as clairvoyance)

\textit{Telepathy:} the spell affects the target’s mind, allowing the
caster to read surface thoughts or intentions, or glean other information
\end{adjustwidth*}
\paragraph{ILLUSION}

Illusion spells have the following specific tags:
\begin{adjustwidth*}{.5cm}{.5cm}

\textit{Concealment:} the spell’s purpose is to conceal its targets
from detection by others

\textit{Distraction:} the spell creates illusions that distract and confuse the target, enhancing your actions or hampering theirs
\end{adjustwidth*}
\paragraph{MANIPULATION}

Manipulation spells have the following specific tags.
\begin{adjustwidth*}{.5cm}{.5cm}

\textit{Protection:} the spell’s focus is protecting the target(s)
against threats

\textit{Telekinesis:} the spell enables the caster to move physical
objects

\textit{Energy:} the spell manipulates energy to create effects
(such as igniting material or generating light)

\textit{Mental:} the spell manipulates the mind of the target
through direct magical force
\end{adjustwidth*}
\paragraph{HEALTH}

Health spells have the following specific tags:
\begin{adjustwidth*}{.5cm}{.5cm}

\textit{Heal:} the spell mends wounds and eases trauma

\textit{Cure:} the spell counteracts the effects of disease, toxins,
and similar threats.

\textit{Enhance:} the spell enhances the physiology of the target
in some, such as increasing a Stat or enabling otherwise
impossible feats
\end{adjustwidth*}
\subsubsection{OTHER SPELL TAGS}
\begin{adjustwidth*}{.5cm}{.5cm}

\textit{Area:} the spell covers an area of effect, within its specified
range, and affects all valid targets in the area. Adding the
area tag to a combat spell reduces the damage die by 1
step (to a minimum of 1d4); adding this to another kind of
spell increases its minimum Force by 1.

\textit{Element:} this spell has an elemental aspect (e.g. acid, fire,
ice, electricity, water, air) with corresponding additional
effects; increase its minimum Force by 1.

\textit{Scaling (add EF to an effect):} add the \textit{exhausting} tag and add the \textit{obvious} tag. Only available for non-Combat spells.

\textit{Exhausting:} this spell is especially draining; the caster must
take at least 1 stun damage (in addition to any other drain damage incurred) when casting this spell (this stun ignores armor, although it can otherwise be healed normally).

\textit{Subtle:} this tag means much the same as it does with other
activities, except that for spells, it indicates that the preparations to cast the spell are subtle; the spell effect itself
may or may not be (for example, a fireball can be subtle,
but only insofar as nobody notices the mage forming the
spell; once it goes off, it’s certainly obvious).

\textit{Triggered:} this spell is triggered by a particular event (often
a move); it remains in effect until the individual in question makes the triggering move or action. This tag is a
replacement for the Short duration tag at the spellcaster’s
discretion.
\end{adjustwidth*}
\subsubsection{THE MAGE'S SANCTUM}
Mages, unfortunately, cannot simply borrow another mage’s
spell to use. The creation of a spell is a very personal event,
and you wouldn’t want to have someone else’s formulas “go
down the wrong pipe,” as it were. As a result, it requires time
(and sometimes money) to develop a spell.

Mechanically, development of a new spell requires the Mage
to spend at least 72 hours in study, preparing reagents,
studying tomes, and inscribing strange symbols. Once done,
of course, the spell is added to the mage’s repertoire; a Mage
never forgets her spells.

It is possible to shorten this process somewhat by obtaining
help from outside sources. Talismongers, for instance, might
be able to locate items or suggest pronunciations; other mages may be able to explain certain concepts to the uninitiated;
and spending time in pure study (using the Initiate move) can
reduce the time required.

\end{multicols}


\invisiblepart{CREATING SPIRITS}
\section{CREATING SPIRITS}
\begin{multicols}{2}
Instead of crafting spells like mages, shamans familiarize
themselves with the denizens of Astral Space, learning to
make bargains and offer wagers in order to secure the aid
and services of these ethereal beings. A practiced shaman is
adept at “wheeling and dealing” with spirits and elementals.
There is a dizzying array of different spirits in the astral world.
\SW{} lets the Shaman create the spirits they wish to
summon.

\subsection{SPIRIT BONDING}
Although the rules here provide a mechanical way to make
your own custom spirits, remember that spirits are independent entities, not “on the fly” creations of the Shaman. In
the game world, the shaman has met, negotiated with, and
bonded with a spirit, developing a relationship (the \textbf{spirit
bond}) with the entity.

\subsubsection{JUST BUSINESS}
It is important to recognize that the relationship between the
Shaman and the spirits to whom he or she has bonded is
not necessarily (or even \textit{usually}) one of friendship or altruism.
Rather, the relationship is more akin to a contract or pact—it
is a business relationship, with consideration promised and
mutually agreeable terms established. Spirits do not, as a
rule, love being randomly yanked out of the astral plane to
perform work for people, and if uncontrolled, are as likely to
turn on their summoner as they are to simply vanish back into
Astral Space.

\subsubsection{RULES}
Use the following procedure to develop the spirits with which
you’ve formed a Spirit Bond.
\begin{adjustwidth*}{.5cm}{.5cm}

1.	Choose the spirit’s \textbf{Type:} elemental or natural.

2.	Choose the spirit’s \textbf{Domain}, and record the base Armor
and Wounds.

3.	Choose the spirit’s \textbf{Nature}, and modify the basic spirit
tags as needed.

4.	Distribute 4 spirit points among spirit’s Moves, adjusting for the spirit’s purpose. No spirt move may have a
modifier higher than +3.

5.	Add additional tags if desired (see \textit{Other Spirit Tags}).

6.	Name your spirit.

\textbf{Example:} \textit{Pam is playing a Shaman named Chert, and is
developing the initial three spirits Chert can summon. Pam
decides the first one will be a natural forest spirit, a protector of the dwindling unspoiled lands.}

\textit{With those decisions made, the spirit’s qualities so far are}
natural, forest, protector, armor 1, wounds 10, dmg 1d8,
guard 1, enthrall -1.

\textit{Pam also wants the spirit to blend in with the forest, and
to an excellent guardian of its inhabitants. She spends
one spirit point (out of 4) to gain the} aspect \textit{tag, and then
spends the remaining three to boost the Guard move
twice, and the Harm move once. The final spirit looks like
this:} natural, forest, protector, harm 2, guard 3, search 0,
enthrall -1, mentor 0, armor 1, wounds 10.
\end{adjustwidth*}
\subsubsection{SPIRIT TYPES}
\begin{adjustwidth*}{.5cm}{.5cm}

\textbf{Elemental:} these spirits represent the basic four elements,
air, earth, fire, and water, and can be summoned anywhere.

\textbf{Natural:} natural spirits are spirits associated with particular domains (such as “city spirits” or “mountain spirits”).
Natural spirits may enter other domains freely, but they
can only be summoned within their own, and if they cross
domains, there’s always a chance they attract unwanted
attention from other spirits who don’t like intruders.
\end{adjustwidth*}
\subsubsection{BASIC SPIRIT TAGS}
\textbf{Domain} represents the spirit’s preferred environment or the
area in which it may be summoned. A natural spirit summoned in its domain always has the generous tag. The domain of an elemental is considered to be the same as its element (though they gain no benefit from being within their
domain).
\begin{adjustwidth*}{.5cm}{.5cm}

\textit{Urban:} spirits that dwell in urban or developed lands, especially cities

\textit{Plains:} spirits that dwell in open plains, grasslands, open
fields, and farms

\textit{Forest:} spirits that dwell in forests, woods, and similar areas

\textit{Mountain:} spirits that dwell in foothills, crags, ridges, and
other mountainous terrain

\textit{Earth:} spirits that dwell underground or in caves; the domains of earth spirits are widespread.

\textit{Deserts:} spirits that dwell in the sere, forbidding landscape
of the deserts

\textit{Sky:} spirits dwelling in the open skies.

\textit{Storm:} spirits of storm and disruption

\textit{Swamps:} spirits who dwell where earth and water are one

\textit{Water:} spirits of the water, be it lakes, rivers, or the open
sea
\end{adjustwidth*}
There are two things to be aware of regarding domains. First,
domains are relatively confined—a mountain spirit’s domain
is not all mountains, nor even all of a specific mountain. Rather, it is usually a region with a radius of around 500 meters,
within a mountainous region. Overlap among domains is possible, and the byzantine negotiations that take place between
spirits defy understanding even by the most gifted shamans.

Also remember that multiple domains may exist within a
larger area that seems uniform. In other words, city spirits
(for example) are the only kind of spirit you’ll run across in a
city—a park within a city may be the home of a forest spirit,
and you may find a river spirit fighting to protect it’s home
from polluted runoff in some industrial area.

\textbf{Armor} represents the spirit’s innate magical resistance to
damage; spirit armor cannot be ignored, nor reduced by
weapons with the AP tag. All spirits have 1 armor.

\textbf{Wounds} simply represent the spirit’s innate health; all spirits,
by default, have 8 wounds.

\subsubsection{SPIRIT NATURE}
Every spirit has a \textbf{nature}, which indicates its sense of purpose
and the activities to which it is drawn. A spirit’s nature also
affects its basic tags and moves (see Spirit Moves, below) in
various ways.

\textbf{Watcher} spirits observe, find, and note. They are incapable of
dealing harm to anyone or anything. Watcher spirits have the
following modifiers: \textit{Search +2, Wounds -2, may not Harm.}

\textbf{Teacher} spirits wish to inform and instruct, and find it difficult to inflict damage upon those they could otherwise teach.
Teacher spirits have the following modifiers: \textit{Mentor +2,
Harm -2, dmg 1d4.}

\textbf{Protector} spirits preserve, defend, and support their domain.
They are unconcerned with influencing intruders, preferring to
throw them out instead. Protector spirits have the following
modifiers:\textit{ Guard +1, Enthrall -1, Wounds +2, dmg 1d8.}

\textbf{Destroyer} spirits are warrior spirits who revel in combat and
bloodletting. They are fearsome enemies, though somewhat
limited in imagination. Destroyer spirits have the following
modifiers:\textit{ Harm +2, Mentor -1, Search -2, Wounds +1, Armor +1, dmg 1d10.}

\textbf{Seducer} spirits wish to influence, to inspire love, andto acquire servants, though they do not typically enjoy directly
harming others. Seducer spirits have the following modifiers:
\textit{Enthrall +2, Harm -1, Wounds -1, dmg 1d4.}

\subsubsection{SPIRIT MOVES}
Spirits and elementals summoned by player characters are
individual beings that have their own set of moves. While
summoned, spirits may perform a number of moves equal to
their Force. Each use of a move below counts
toward that limit.

When creating a spirit, the Shaman may spend up to 4 spirit
points to increase the value of a spirit’s moves. However,
remember that some additional tags cost spirit points, so use
them wisely!

\textbf{HARM:} when a spirit \textbf{attacks someone or something,}
roll+Harm. On 10+, the spirit deals its damage. On 7-9, the
spirit deals damage, but also takes damage.

\textbf{SEARCH:} when the spirit \textbf{attempts to locate individuals or
items within its domain,} roll+Search. On 10+, the spirit locates the item and can tell the Shaman where it is. On 7-9,
the spirit can tell the shaman whether the item or person is
within its domain, but not it’s specific location. Note: the GM
and player should determine the search range for elementals.

\textbf{GUARD:} when a spirit \textbf{stands in defense of its domain or
inhabitants thereof,} roll+Guard. On 10+, the spirit prevents
damage or hostile effects from occurring. On 7-9, the spirit
halves damage or the potency of a hostile effect.

\textbf{ENTHRALL:} when a spirit \textbf{attempts to control someone’s
actions or thoughts,} roll+Enthrall. If the target is a:
\begin{adjustwidth*}{.5cm}{.5cm}

\tcirc{} An NPC: On a 10+, the spirit issues two instructions
that the NPC must follow, or take 3 damage. On 7-9,
the spirit may issue one instruction.

\tcirc{} A PC: On a 10+, both of the following apply. On 7-9,
only 1 applies:
\begin{adjustwidth*}{.5cm}{.5cm}

\tcirc{} If the character complies, they mark XP

\tcirc{} If the characer refuses, they must Stay Frosty
\end{adjustwidth*}
\end{adjustwidth*}
\textbf{MENTOR:} when a spirit \textbf{imparts knowledge or truth}, roll+Mentor. On 10+, the GM provides, in secrete, a useful or
interesting piece of information to the target. On 7-9, the GM
provides an interesting piece of information.

\subsubsection{OTHER SPIRIT TAGS}
\begin{adjustwidth*}{.5cm}{.5cm}


\textit{Robust:} the spirit is particularly resistant to damage; all
damage rolls against it are \textbf{[w]}. Adding this tag costs 1
spirit point.

\textit{Aspect:} the spirit takes on the appearance of their domain,
and is invisible in their domain unless it chooses to be
seen. All spirits have this tag.

\textit{Generous:} the spirit will perform one extra move; adding
this tag costs 1 spirit point.

\textit{Insubstantial:} damage dealt and taken is halved

\textit{Weakness (specify):} the spirit has a weakness to a particular
material or element which ignores insubstantiality, armor,
and robustness. Adding this tag allows the free addition of
another tag.

\textit{Engulf:} the spirit may enclose a target in the ubstance of its
domain, typically (but not always) dealing damage.

\textit{Wild:} this spirit has an extra spirit point, but the shaman
must take -2 whenever he or she conjures it.
\end{adjustwidth*}
\subsection{MAKE NEW BONDS}
As with weaponry, spells, or programs, it takes time and effort to develop a relationship with a spirit. The spirit creation
rules here are, as already said, not intended for “on the fly
summoning,” rather they are intended to help Shaman players create a list of spirits that the shaman is accustomed to
summoning, and that fit the player’s desired concept for their
character.

If the Shaman wants to develop a relationship with a new
spirit, the character must spend at least two full days of
downtime meditating and communing, meeting and negotiating with spirits in the Astral realm. At the conclusion of this
time, the Shaman’s player may create a new spirit with whom
the Shaman has formed a bond.

\subsubsection{INTRODUCTIONS}
A shaman can reduce the time spent in bargaining with a
new spirit in a very simple way—have another spirit “make
introductions.” To do so, a Shaman must be mentored by another spirit (one he or she has summoned). If the mentoring
is successful (use the Mentor move), reduce the time required
by one day.

\end{multicols}

\invisiblepart{TOTEMS}
\section{TOTEMS}
\begin{multicols}{2}
Shaman characters must select a totem, representing their
connection to one of the great spirits.

\paragraph{BEAR}
\begin{adjustwidth*}{.5cm}{.5cm}
\textbf{Boons:} +1 to summoning Protector spirits.

\textbf{Flaw:} when injured, roll 1d6. On 1 or 2, the shaman goes
berserk).
\end{adjustwidth*}
\paragraph{CAT}
\begin{adjustwidth*}{.5cm}{.5cm}
\textbf{Boons:} gain low-light vision; you cannot be surprised

\textbf{Flaw:} you cannot deal lethal damage to your enemy
\end{adjustwidth*}


\paragraph{COYOTE}
\begin{adjustwidth*}{.5cm}{.5cm}
\textbf{Boons:} take +1 to conjure Teacher spirits

\textbf{Flaws:} destroyer spirits summoned lose 1 spirit point
\end{adjustwidth*}

\paragraph{DARK KING}
\begin{adjustwidth*}{.5cm}{.5cm}
\textbf{Boons:} when you check the situation, you are boosted

\textbf{Flaw:} take -1 ongoing to gut checks
\end{adjustwidth*}

\paragraph{DOG}
\begin{adjustwidth*}{.5cm}{.5cm}
\textbf{Boons:} and take +1 to conjure protector spirits or city
spirits

\textbf{Flaw:} your moves are glitched if you have left an ally
behind or in danger
\end{adjustwidth*}


\paragraph{DRAGONSLAYER}
\begin{adjustwidth*}{.5cm}{.5cm}
\textbf{Boons:} take +1 to stay frosty

\textbf{Flaw:} if you break a promise, your moves are glitched until you fulfill the promise or otherwise atone
\end{adjustwidth*}

\paragraph{GATOR}
\begin{adjustwidth*}{.5cm}{.5cm}
\textbf{Boons:}take +1 to conjure water spirits.

\textbf{Flaw:} You are exceptionally greedy
\end{adjustwidth*}


\paragraph{EAGLE}
\begin{adjustwidth*}{.5cm}{.5cm}
\textbf{Boons:} take +1 to conjure watcher spirits or air elementals

\textbf{Flaw:} you have an allergy to something relatively common, and take -1 ongoing when exposed
\end{adjustwidth*}

\paragraph{FIRE-BRINGER}
\begin{adjustwidth*}{.5cm}{.5cm}
\textbf{Boons:} take +1 to conjure fire spirits.

\textbf{Flaw:} when facing a difficult choice, you invariably choose to aid others, even at your own expense
\end{adjustwidth*}


\paragraph{LION}
\begin{adjustwidth*}{.5cm}{.5cm}
\textbf{Boons:} take +1 to conjure protector or plains spirits

\textbf{Flaw:} Take -1 on Gut Checks
\end{adjustwidth*}

\paragraph{MOON MAIDEN}
\begin{adjustwidth*}{.5cm}{.5cm}
\textbf{Boons:} take +1 ongoing to manipulate others

\textbf{Flaw:} take -1 to casting combat spells
\end{adjustwidth*}

\paragraph{MOUNTAIN}
\begin{adjustwidth*}{.5cm}{.5cm}
\textbf{Boons:} take +1 to conjure earth spirits

\textbf{Flaw:} once you've decided on a plan of action, you stick to it - even it means going alone
\end{adjustwidth*}


\paragraph{OWL}
\begin{adjustwidth*}{.5cm}{.5cm}
\textbf{Boons:} gain low-light vision, take +1 to conjure teacher
spirits

\textbf{Flaw:} During the day, the minimum force to cast spells is increased by 1
\end{adjustwidth*}


\paragraph{RACCOON}
\begin{adjustwidth*}{.5cm}{.5cm}
\textbf{Boons:} and take +1 to conjure watcher spirits

\textbf{Flaw:} must Stay Frosty to avoid letting his curiosity get to
him
\end{adjustwidth*}


\paragraph{RAT}
\begin{adjustwidth*}{.5cm}{.5cm}
\textbf{Boons:} take +1 to conjure city spirits

\textbf{Flaw:} when combat starts, you must Stay Frosty, or flee
\end{adjustwidth*}


\paragraph{RAVEN}
\begin{adjustwidth*}{.5cm}{.5cm}
\textbf{Boons:} take +1 to conjure watcher spirits

\textbf{Flaw:} you must take advantage of others’ misfortune
when you can
\end{adjustwidth*}

\paragraph{SEDUCTRESS}
\begin{adjustwidth*}{.5cm}{.5cm}
\textbf{Boons:} take +2 when manipulating someone

\textbf{Flaw:} when given the opportunity to indulge in a vice, roll 1d6: on 1, 2, or 3, the shaman gives into the vice (drugs, btls, etc...)
\end{adjustwidth*}


\paragraph{SHARK}
\begin{adjustwidth*}{.5cm}{.5cm}
\textbf{Boons:} take +1 to conjure destroyer spirits

\textbf{Flaw:} when injured or when you injure another, roll 1d6: on 1, 2, or 3, the shaman
goes into a frenzy. The shaman may continue to attack their last victim instead of moving on to nearby opponents. 
\end{adjustwidth*}


\paragraph{SNAKE}
\begin{adjustwidth*}{.5cm}{.5cm}
\textbf{Boons:} take +1 to conjure seducer spirits

\textbf{Flaw:} take -1 ongoing to Rock \& Roll
\end{adjustwidth*}

\paragraph{THUNDERBIRD}
\begin{adjustwidth*}{.5cm}{.5cm}
\textbf{Boons:} take +1 on when making them sweat

\textbf{Flaw:} when insulted, roll 1d6: on 1, 2, or 3, the shaman can't resist but to respond in kind
\end{adjustwidth*}

\paragraph{WISE WARRIOR}
\begin{adjustwidth*}{.5cm}{.5cm}
\textbf{Boons:} take +1 ongoing to Rock \& Roll

\textbf{Flaw:} when you have acted dishonorably, you are glitched until you are able to atone
\end{adjustwidth*}


\paragraph{WOLF}
\begin{adjustwidth*}{.5cm}{.5cm}
\textbf{Boons:} take +1 to conjure protector spirits

\textbf{Flaw:} you must Stay Frosty to retreat from combat
\end{adjustwidth*}

\end{multicols}

\invisiblepart{Compendium Classes}
\section{Compendium Classes}
\begin{multicols}{2}

In \SW{}, a \textbf{compendium class} is an mixin class that augments your existing class. Similar to cross-archetype moves, compendium classes provide a set of moves a player may take for their character upon advancement. However, compendium classes differ in two respects:

\begin{adjustwidth*}{.5cm}{.5cm}

1. All compendium classes possess a requirement. The nature of the requirement varies, but in generally they represent a special milestone in a character's life which marks a significant change in attitude, social standing, or another context.

2. There are no restrictions on the number of moves that be chosen from a compendium class. Typically the number of moves per class is low and the existence of the entrance requirement helps to focus the character down a particular path.

\end{adjustwidth*}

Compendium classes typically fulfill two roles in \SW{}. Most commonly, compendium classes allow characters to explore niches that are not fully represented by standard archetypes. A good example would be the \textit{Assassin} class - while the \textit{Covert Ops} archetype could be played as an assassin, for example, their class moves aren't especially specialized for wetwork jobs. At the same time, within the world of \SW{}, assassins are not common enough to be a general character archetype -  they certainly exist, but those individuals that market themselves as professional assassins are rare.

The other role fulfilled by compendium classes are campaign specializations. Every setting has its special groups and a compendium class is an excellent method of describing membership. Whether that means a character is part of a setting's organization, where class moves represent perks of the position, or has undergone an experience unique to the setting, where the moves represent the effects of the experience.

Several example compendium classes are listed below:

\end{multicols}

\newpage

\invisiblepart{Compendium Class: Assassin}
\section{Assassin}
\begin{multicols}{2}


\texttt{>>> I get it. No one ever wants to be the guy pulling the trigger. Maybe they think they have a conscious, or whatever. Or maybe they just don't want to deal with the mess. I don't really care. Because one thing is for certain: no matter their reasons for not doing the job themselves, there is a never ending stream of people who have no problem hiring me to do it for them. <<<}

\textbf{The Assassin} is a cold blooded killer for hire. That is the thin line that separates the assassin from the common murderer. As long as the money is good, it doesn't matter much who has to die. The assassin is a master of eliminating targets with minimal fuss, whether at range or up close. 

\subsection{Assassin Moves \& Requirements}

When you have \textbf{accepted a kill contract and successfully completed the job by eliminating your target and only your target}, the next time you advance, you may take this move:

\textbf{CONTRACT KILLER:} when you \textbf{have downtime and put out word that you’re looking to take on a contract}, roll+Presence. On 10+, someone approaches you with a job - they’ll give you a name and maybe a description. Roll 2d6b: that’s what the job is worth to them, in thousands of nuyen; take it or leave it. On a 7-9, the job has strings attached - they want you to kill the target in a specific way or place, by a specific time, etcetera.

 Either way, once the job is done, they'll find a way to pay you. If you fail to complete a contract, take -1 ongoing to \textbf{Contract Killer} until you prove yourself again.

If you have taken the \textbf{Contract Killer}
move, you are eligible to take any of the
following moves next time you advance:

\textbf{PROFESSIONAL:} when you \textbf{are
  approached for a kill contract}, instead roll
1d10 to determine the price of the contract in
thousands of nuyen. 

\textbf{MEASURE TWICE, KILL ONCE:} when you
\textbf{have time to study the environment of your
  target beforehand for at least an hour}, you are
never considered surprised when operating within
that environment for the next 24 hours.

\textbf{HEARTSEEKER:} when you \textbf{Rock \&
  Roll}, on a 10+ you may spend 1 Edge to specifically target and
destroy a vital organ of your choice.

\textbf{GRIM EXPERIENCE:} when you \textbf{take damage
from combat}, you may optionally take 1 additional
point of damage. If you do so, mark off an
additional point of XP.

\textbf{SERIES OF IMPROBABLE EVENTS:} when you
\textbf{doctor a scene to hide your involvement},
roll+Craft. On 10+, you can create an impression
that the murder was an one in a million
accident. On 7-9, you hide your tracks, but
careful investigation will reveal the murder.


\end{multicols}

\invisiblepart{Compendium Class: Smuggler}
\section{Smuggler}
\begin{multicols}{2}

\texttt{>>> Everyone wants something. And when
  there is want, someone will be selling. That's
  capitalism for you. Of course, selling requires
  supply and that's where I come in. Guns, drugs,
  whores, even the occasional creepy magic
  artifact, I don't much care. Just give me the
  cash, the cargo, and the destination - I'll get
  it there in record time, no questions asked. <<<}

\textbf{The Smuggler} is an expert at moving past
borders and checkpoints all while keeping their
cargo hidden and safe. Good smugglers know all the
tricks to avoid border patrols and where to hide
their potentially illicit goods. Its dangerous
work, but for those who enjoy the adrenaline
rush, a smuggler's life pays well. There's always
someone who wants something moved.

\subsection{Smuggler Moves \& Requirements}

When you have \textbf{successfully transported a
  significant amount of contraband past a border
  checkpoint at great personal risk}, the next
time you advance, you may take this move:

\textbf{STICK IT WHERE?!:} when you \textbf{hide a
  small item (the size of your fist or smaller) on
  your person}, roll+Craft. On 10+ the item is
undetectable. On 7-9, you successfully hide the
item, but it is very awkward for you.

If you have taken the \textbf{Stick it Where?!}
move, you are eligible to take any of the
following moves next time you advance:

\textbf{ITS BIGGER ON THE INSIDE:} when you \textbf{have
  downtime or legwork time}, you may modify a
vehicle or drone you own to contain a hidden
compartment. The vehicle or drone cannot contain
more hidden compartments than half its Frame,
rounded down.

\textbf{I DON'T HAVE ANY:} when you \textbf{Stay
  Frosty and lie
  about what you are carrying}, if you have not
looked at looked at your cargo's contents, you are
boosted for the move.

\textbf{EAT MY DUST:} when you \textbf{attempt to
  flee from authorities in a vehicle},
roll+Craft. On 10+, you can execute some quick
maneuvers to throw them off your trail. On 7-9,
you succeed temporarily, but your pursuers will pick up your
trail again soon.

\textbf{TRANSPORT SPECIALIST:} you are an expert
at transporting a particular kind of cargo. Choose
a type of cargo from below - when carrying that
cargo, take +1 ongoing to all rolls.
\begin{moveoptions}
\moveoption{Weapons}
 
\moveoption{Drugs}

\moveoption{People}

\moveoption{Animals}

\moveoption{Magical}

\moveoption{Very Large Objects}

\end{moveoptions}

\textbf{HIDDEN SHORTCUT:} when \textbf{traveling
  in a vehicle or by foot}, roll+Awareness. On
10+, you know of a shortcut that cuts your normal travel
time in half. On 7-9, your shortcut works, but you
encounter trouble on the way.

\end{multicols}

\invisiblepart{Compendium Class: Gang Lord}
\section{Gang Lord}
\begin{multicols}{2}

\texttt{>>> The streets are a hard place. Full of
  dangerous, nasty people. And I'm the worst of
  all. By blood, guile, and ruthless
  determination, I've climbed to the top of this
  social ladder of human trash. But there's no
  rest for the wicked - every week there seems to
  be a new challenger. But I'd be lying if I told
  you I was tired of crushing those bitches skulls. <<<}

\textbf{The Gang Lord} is the ruler of his own little
corner of hell. Outside of the major cities, slums
and barrens crawl with roaming gangs of thugs. And
each gang has its leader. Gang lords, while often
short lived, command respect in their local
communities and have their fingers in all the
local crime. 

\subsection{Gang Lord Moves \& Requirements}

When you \textbf{personally depose an existing
  gang lord in a fashion that indisputably
  demonstrates your power}, the next
time you advance, you may take this move:

\textbf{LEADER OF THE PACK:} you gain command of a
local gang containing 5 members. They follow your commands and, while
you don't lead your gang into hard times, are
loyal to you. Your gang's turf, or sphere of
influence, is a number of kilometers equal to the
number of members in your gang. Choose a gang attribute from the
following list:
\begin{adjustwidth*}{.5cm}{.5cm}
\textbf{Big} Your gang is rather large. Roll 2d6
and add that number of extra members to your gang.

\textbf{Savage} Your gang is known to be extra
savage in combat. Gain +1 towards \textbf{Make
  Them Sweat} if your target knows of your gang.

\textbf{Bikers} Your gang is a biker gang. Your
gang's turf is now a number of kilometers equal to
twice the number of members in your gang.

\textbf{Wizkids} Your gang has Awakened
members. For every five gang members, choose one
member to be a mage, shaman, or adept.

\textbf{Contraband} Your gang traffics in some
form of contraband. Gain d6 hundred nuyen a week
as your cut as leader.

\textbf{Tight} Your gang is tight-knit by a shared
bond. Your members are more loyal than usual and
will go the extra mile for you.

\textbf{Dug In} Your gang possesses a hidden and
fortified safehouse. 

\textbf{Chromed} Your gang has numerous cyberware
enhancements. When they join you in combat, you
gain +1 towards \textbf{Rock \& Roll} moves.

\textbf{Techheads} You gang has hackers and
riggers. For every five gang members, choose one
member to be a hacker or rigger.

\end{adjustwidth*}

\critterspec
{GANG MEMBER}
{group, intelligent, medium}
{Spiked bat (1d6+1 dmg, c), cheap but powerful pistol (2d8w dmg, s/m)}
{9 Wounds / 1 Armor}
{A typical member of your gang. They're not
  especially dangerous to a well armed group, but
  in the local neighborhood they are a force to
  reckoned with.}
{to guard their turf.}
{}

If you have taken the \textbf{Leader of the Pack}
move, you are eligible to take any of the
following moves next time you advance:

\textbf{STREET LESSONS:} Choose an additional gang attribute.

\textbf{NEIGHBORHOOD SNITCH:} Your gang had eyes
and ears everywhere within your gang's turf. When
operating within your turf, you cannot be
surprised. 

\textbf{GOT YOUR BACK:} when you \textbf{encounter
  trouble within your gang's turf}, you can call
on your gang and home turf advantage for help and roll+Presence. On 10+,
choose 2. On 7-9, choose one.
\begin{moveoptions}
\moveoption{A gang member is close by and appears
  instantly to aid you}
 
\moveoption{d6 (or total number of gang members,
  whichever is smaller) arrive in 5 minutes to
  aid you. }

\moveoption{Your knowledge of your turf grants you
 +3 hold for subsequent \textbf{Rock \& Roll} and
 \textbf{Check the Situation} moves.}

\end{moveoptions}

\textbf{ARMS SHIPMENT:} you arrange underworld
deals to upgrade your gang's gear. Increase your
gang members' armor by 1 and replace their pistols
with either a SMG (range s/m, sa/fa, dmg 1d8, AP
1) or a shotgun (range s/m, sa, dmg 1d10+1, obvious, loud,
forceful).

\textbf{LEGION:} when you \textbf{defeat a rival
  gang member in combat} you may spare their life
and roll+Presence. On 10+, you convert them into
your gang as a new loyal member. On 7-9, you convert
them, but their loyalty is suspect. 

\end{multicols}


\end{document}
%%% Local Variables: 
%%% mode: latex
%%% TeX-master: t
%%% End: 
