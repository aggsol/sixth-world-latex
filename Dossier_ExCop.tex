\invisiblepart{DOSSIER : EX-COP}

\section{THE EX-COP}
\begin{multicols}{3}
\setlength{\parskip}{.05cm}

\texttt{>>>Years on the job, and now what am I doing? Running
the shadows. Shit, I used to throw skels like myself in jail ev-
ery day. On the other hand, the pay is better than anything
I made on the force, I get to meet interesting people, and it
beats corporate rent-a-cop work.}

\texttt{Some of these folks, they think because they’ve got the wires,
or the mojo, they can walk circles around me. And yeah,
maybe so, if I ever let them have a level playing field. But I
still think like a cop, and I know the system. People still on the
job are happy to help an old buddy.}

\texttt{And while the badge may not be entirely official anymore,
there’s always the gun.<<<}

\textbf{The Ex-Cop} comes from Lone Star, Knight Errant, the
military police, or any one of many law enforcement
agencies in the confused landscape of the 2050’s.
Possessed of a keen investigative mind, brutally effec-
tive combat skills, experience with the best and worst
of humanity, and connections deep into “the system,”
the ex-cop is a valuable asset.


\subsection{CREATING AN EX-COP}

\paragraph{1.  Choose your Metatype}

You may choose \textbf{Human}, \textbf{Dwarf}, \textbf{Elf}, \textbf{Ork}, or
\textbf{Troll}. Each metatype offers a selection of meta-
type moves. Choose one metatype move from
the options presented.

\paragraph{2.  Choose your look}

\textit{Cold eyes, tired eyes, wary eyes}

\textit{Close cropped hair, shaggy hair, bald}

\textit{Cheap suit, street clothes, hawaiian shirt}

\textit{Heavy body, fit body, injured body}

\paragraph{3.  Choose your name and street name}

Make up a name and street name or pick a real
name and street name from the lists and name
generators starting in the \textbf{GM Resources} section.

\paragraph{4.  Assign your stats}

You have 5 stats: Awareness, Combat, Stamina,
Craft, and Presence. Important stats for you are
Craft, Presence, and Combat.

You have 4 \textbf{Build Points} to distribute among
your stats. To increase a stat by 1 point costs 1
Build Point. You may increase a stat to a maxi-
mum of +2 as a starting character. If you wish,
you may lower 1 stat to -1 in order to have an
additional point to spend.

\paragraph{5.  Choose your Equipment}

Choose from the lists below, or customize your
own gear using the rules in \textbf{Creating Gear} on
page 60.

\textbf{Armor:} \textit{armor vest, form-fitting armor}

\textbf{Service Pistol:} \textit{Ruger Super Warhawk, Colt
Manhunter}

\textbf{Additional Weapon:} \textit{HK 227,
  Remington 990}

\paragraph{6.  Choose your cyberware}

You may start with one of the following cyber-
ware kits (descriptions of these items are on
page 45):

\textbf{Kit 1 (3 essence):} \textit{smartlink, bone lacing}

\textbf{Kit 2 (3 essence):} \textit{cybereyes with low-light
and flare compensator, level 1 skillwires}


\paragraph{7.  Set your Essence and Edge.}

To determine your starting Essence, subtract the
essence cost of your cyberware (if any) from 6.

You start with 3 Edge.

\paragraph{8.  Choose 3 Contacts}

Confidential informant (CI), precinct secretary,
gang leader, prosecutor, journalist, former part-
ner, defense attorney

\paragraph{9.  Establish debts and favors}

Place one of your fellow runners’ names in at
least one of the blanks in the \textbf{Debts \& Favors}
section of your playbook. Each time a name
appears in a debt or favor, it counts as 1 Bond
with that character. The more people you have
Bond with, the better.

\paragraph{10.  Starting Funds}

You start play with 3d6 x 250¥ immediately
available.

\paragraph{11.  Starting Moves}

You know all the Core and Secondary Moves.
You also know the \textbf{Gumshoe} move, and
one other Ex-Cop move.

\end{multicols}

\newpage


\begin{dossier}
\dossierstatbar{THE EX-COP}
\hspace{.5cm}%
\vrule width 2pt
\hspace{.3cm}%
\begin{dossiermovebar}
\fontsize{9pt}{1em}\selectfont
\setlength{\parskip}{.05cm}


\selectedMove{Gumshoe:} when you examine the scene of an event, or interrogate someone about an
event, roll+Craft. On 10+, pick two of the following to learn (relevant to what you’re investi-
gating). On 7-9, pick one:
\begin{moveoptions}
\moveoption{ Scene: when the events happened; whether magic was involved; how many individuals
were involved; if this is the primary scene of the
event}

\moveoption{ Person: if they’re connected to the event; whether they’re hiding something; what they
stood to lose or gain; a useful personal detail
(e.g, a tic, handedness, etc.)}
\end{moveoptions}

\unselectedMove{Work the System:} when you use your ex-LEO status to get help, roll+Presence. On
10+, you have an old pal jam somebody up or cut them a break. On 7-9, you get the de-
sired result, but (choose 1):
\begin{moveoptions}
\moveoption{ the person knows who helped or hindered
them}

\moveoption{ your buddy got in trouble}

\moveoption{ your name got mentioned to the wrong ears}
\end{moveoptions}

\unselectedMove{Takedown:} when you take control of a person physically, roll+Combat. On 10+, they
are under your complete control, and you are both unharmed. On 7-9, you gain control of
them, but either you or your target must take 2
damage.

\unselectedMove{Interrogation:} when you attempt to make someone sweat, you may roll+Craft instead
of +Presence.

\unselectedMove{Deep Cover:} when you Stay Frosty to blend in to a criminal environment, you are
boosted.

\unselectedMove{Good Cop, Bad Cop:} when you aid someone you have bond with during an interroga-
tion, roll+Craft instead of +Bond.

\unselectedMove{Gun Cage:} when you need some specialized weapons or hardware fast, you can bor-
row it from a buddy on the force. When you return it, roll+Presence (subtracting 1 for every
day beyond the first that you’ve had the gear). On 10+, everything’s fine. On 7-9, you’ve
raised suspicions and can’t use this move for your next run. On 6 or less, you owe your
buddy a favor before you get trusted with the keys
again.

\unselectedMove{The Feds:} you have a contact in federal law enforcement. Roll+Presence. On 10+, pick
2. On 7-9, pick 1.
\begin{moveoptions}
\moveoption{ You get a tip-off on a big operation so
you can steer clear}

\moveoption{ You gain interesting and useful
information about your current run}

\moveoption{ You get access to federal data on an
individual}

\moveoption{ You are listed as a “consultant” on a
case}
\end{moveoptions}

\unselectedMove{Doorkicker:} when you lead the team in an assault on the enemy, roll+Combat. On 10+,
designate up to 3 enemies who are surprised. On
7-9, designate up to 2 enemies.

\unselectedMove{Hostage Negotiator:} when you negotiate in a dangerous situation, you may
roll+Combat instead of +Presence. You must still have leverage to negotiate.

\end{dossiermovebar}%
\end{dossier}

%%% Local Variables: 
%%% mode: latex
%%% TeX-master: "sixth_world"
%%% End: 
