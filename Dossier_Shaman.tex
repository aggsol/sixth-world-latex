\invisiblepart{DOSSIER : SHAMAN}

\section{THE SHAMAN}
\begin{multicols}{3}
\setlength{\parskip}{.05cm}

\texttt{>>>My partner over there likes blasting lightning from his
hands. That’s cool, you know? I mean seriously - it’s cool.
And scary. I’d be jealous, but...I have this other trick. See, in-
stead of channeling power through my hands and poring
over dusty tomes, I just have a quick look-see into the unseen
world around us, locate a friend, and ask ‘em for a hand.}

\texttt{You’re looking at me like you’ve got no idea what I’m talking
about. Lemme break it down for you. All around you, right
now, is the world of astral energy. It’s like our world, but...
not. Okay, not really at all but let’s not get off-topic. Dwelling
there are spirits. Some are called elementals, but what’s nec-
essary to grok is this: I can talk to ‘em, and I can bring them
here, and I can make them do things.}

\texttt{So remember to thank me the next time a being of pure fire
appears and saves your ass from getting geeked.<<<}

\textbf{The Shaman} is a master of conjuring: summoning the
spirits that dwell in the astral realm and compelling
them to do the shaman’s bidding. The shaman’s spir-
its provide many services, from devastating combat
abilities to protection from hostile intent to informa-
tion and reconnaissance impossible for a mundane.



\subsection{CREATING A SHAMAN}

\paragraph{1.  Choose your Metatype}

You may choose \textbf{Human}, \textbf{Dwarf}, \textbf{Elf}, \textbf{Ork}, or
\textbf{Troll}. Each metatype offers a selection of meta-
type moves. Choose one metatype move from
the options presented.

\paragraph{2.  Choose your look}

\textit{Heterochromic eyes, wise eyes, sunglasses}

\textit{Long hair, dreadlocks, shaved head}

\textit{Street clothes, anachronistic clothes, biker gear}

\textit{Wiry body, thin body, round body}

\paragraph{3.  Choose your name and street name}

Make up a name and street name or pick a real
name and street name from the lists and name
generators starting in the \textbf{GM Resources} section.

\paragraph{4.  Assign your stats}

You have 5 stats: Awareness, Combat, Stamina,
Craft, and Presence. Important stats for you are
Craft, and Stamina.

You have 4 \textbf{Build Points} to distribute among
your stats. To increase a stat by 1 point costs 1
Build Point. You may increase a stat to a maxi-
mum of +2 as a starting character. If you wish,
you may lower 1 stat to -1 in order to have an
additional point to spend.

\paragraph{5.  Choose your Totem}

Choose a totem from the list on page 71, or
make up one of your own.

\paragraph{6.  Choose your Equipment}

Choose from the lists below, or customize your
own gear using the rules in \textbf{Creating Gear} on
page 60.

\textbf{Armor:} \textit{Leather jacket, defensive charm, riot
shield}

\textbf{Weapon:} \textit{Ruger Super Warhawk, Colt Man-
hunter, AK-97, combat axe, crossbow}

\paragraph{7.  Bond with your Spirits}

You start the game able to summon 3 spirits. Ei-
ther choose 3 spirits from the examples on page
43, or create these spirits using the rules in the
\textbf{Creating Spirits} section starting on page 69.

\paragraph{8.  Set your Essence and Edge.}

You start with 6 Essence and 3 Edge.

\paragraph{9.  Choose 2 Contacts}

Wage mage, ork underground, gang thug, street
cop, herbalist, university professor, diner owner,
fetishmonger, art dealer, hedge wizard, houngan

\paragraph{10.  Establish debts and favors}

Place one of your fellow runners’ names in at
least one of the blanks in the \textbf{Debts \& Favors}
section of your playbook. Each time a name
appears in a debt or favor, it counts as 1 Bond
with that character. The more people you have
Bond with, the better.

\paragraph{11.  Starting Funds}

You start play with 3d6 x 150¥ immediately
available.

\paragraph{12.  Starting Moves}

You know all the Core and Secondary Moves.

You also know the \textbf{Conjure} and
\textbf{Commune} moves.

\end{multicols}

\newpage

\begin{dossier}
\dossierstatbar{THE SHAMAN}
\hspace{.5cm}%
\vrule width 2pt
\hspace{.3cm}%
\begin{dossiermovebar}
\fontsize{9pt}{1em}\selectfont
\setlength{\parskip}{.1cm}

\selectedMove{ Conjure:} When you summon a spirit, choose the spirit's force and roll. What stat you add
depends on the spirit’s nature:
\begin{moveoptions}
\moveoption{ \textbf{Destroyer:} roll+Combat}

\moveoption{ \textbf{Teacher:} roll+Craft}

\moveoption{ \textbf{Protector:} roll+Stamina}

\moveoption{ \textbf{Seducer:} roll+Presence}

\moveoption{ \textbf{Watcher:} roll+Awareness}
\end{moveoptions}

For every 3 points of force, take -1 to the
roll. The force can not exceed twice your essence.

On 10+, the being is summoned as expected, and may perform a number of Spirit Moves
equal to it's force. On 7-9, the being is summoned, but (choose 1):
\begin{moveoptions}
\moveoption{ it causes drain; take the spirit's force divided by 2 as stun. If the force is greater than your current essence, the damage is physical. }

\moveoption{ you only manage to summon it at half
  the desired force}

\moveoption{ You must expose yourself to danger or an attack to summon the spirit}
\end{moveoptions}
On a failure, the spirit does not manifest, and
you take 1 stun as drain. If you roll a natural 2
(that is, “snake eyes”), the spirit is summoned in an uncontrolled state, and the GM will control
its actions until it is exhausted or banished.

\selectedMove{Commune:} when you mentally commune with your totem, you may gain its boons and
flaws.

\unselectedMove{Banish:} when you attempt to banish a spirit, roll+Stamina. On 10+, you reduce the spirit’s
available moves by 1. On 7-9, you reduce the spirit’s moves by 1, but it deals 1 damage to
you (ignoring armor). If you reduce the spirit’s available moves to 0, it vanishes immediately.

\unselectedMove{ Binding:} when you know a free spirit’s true name and attempt to bind it, roll+Presence.
On a 10+, the true spirit falls under your control and can be called upon later. On a 7-9, the
spirit is controlled, but only for the remainder of the scene.

\unselectedMove{ Favored Spirit:} choose 1 spirit type (Watcher, Teacher, Protector, Destroyer, Seducer).
Take +1 when conjuring spirits of that type.

\unselectedMove{ Aura Mask:} you may conceal your magical nature. Roll+Craft. On 10+, you appear to
be a mundane individual to anyone or anything that examines you. On 7-9, you appear
mundane, but must spend 1 Edge to do so.

\unselectedMove{ Spirit Master:} whenever you
would summon a spirit of force greater than 1, you
may instead conjure multiple spirits, dividing the force among them.

\unselectedMove{ Ally:} choose one of your spirits. This spirit becomes your ally, and when summoned,
always performs one Spirit Move for free for the Shaman. The Spirit also develops a
telepathic link with the Shaman, becoming a new contact. If you ever roll snake eyes while
summoning your ally, it becomes a free spirit.

\unselectedMove{ Great Spirit:} when you conjure a
spirit, if you may take -2 to the roll and the spirit is summoned as a Great Spirit. The Great Spirit is immune
to non-magical attacks, and it has 2 more spirit points increase its Moves for as long as it is
summoned. If you take drain from the summoning,
increase the amount by 1.

\unselectedMove{Spirit Hunter:} when you battle a spirit, you can spend 1 Edge to 
                  ignore the spirit’s armor. 

\end{dossiermovebar}%
\end{dossier}

%%% Local Variables: 
%%% mode: latex
%%% TeX-master: "sixth_world"
%%% End: 
